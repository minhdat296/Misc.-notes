\input{article preambles}

\setcounter{section}{-1}

\input{commands}

\begin{document}

    \title{Geometric unramified abelian global class field theory}
    
    \author{Dat Minh Ha}
    \maketitle
    
    \begin{abstract}
        The Langlands Programme is a network of many deep conjectures (and recently, some outstanding theorems!) with far-reaching consequences, notably in the realms of algebraic number theory and representation theory. Its starting point is what we nowadays call \textbf{class field theory}, which we shall approach via algebraic geometry. This is a non-traditional approach, but it will help us understand why the study of Galois groups naturally requires representation theory, which as a consequence, helps us makes sense of why class field theory is the same as the Langlands Correspondence for the group $\GL_1$.
    \end{abstract}
    
    {
      \hypersetup{} 
      %\dominitoc
      \tableofcontents %sort sections alphabetically
    }

    \section{Introduction}
        Our starting point is the classical following result, which hints at why one might even suspect any sort of involvement of algebraic geometry in the first place:
        \begin{proposition}[Curves and function fields] \label{prop: curves_and_function_fields}
            \cite[\href{https://stacks.math.columbia.edu/tag/0BY1}{Tag 0BY1}]{stacks} For any perfect field $k$, one has an equivalences of categories:
                $$\{\text{Field extensions $K/k$ of transcendence degree $1$ and $k$-algebra homomorphisms}\}^{\op}$$
                $$\cong$$
                $$\{\text{Smooth projective curves $X/k$ and \href{https://stacks.math.columbia.edu/tag/01RI}{\underline{dominant}} \href{https://stacks.math.columbia.edu/tag/01RR}{\underline{rational}} maps}\}$$
        \end{proposition}
        From this proposition, one infers that the study of global function fields over perfect fields $k$ (i.e. field extensions $K/k$ of transcendence degree $1$, such as $k(t)$) is actually the same as the study of smooth projective curves\footnote{The reader might also wish to consult \cite[Sections I.13 and I.14]{neukirch_2010_algebraic_number_theory}.} over $\Spec k$, such as $\P^1_k$. As a result, one might believe that at least for global function fields, there should exist a purely geometric version of class field theory. The idea is that such a version of global class field theory would be much more intuitive than the classical formulation of the theory, as various relevant notions from algebraic number theory - such as places, ramification, or even what it means to be \say{global} itself - have natural interpretations in the language of schemes. 
        
        Now, while such a version of global class field theory indeed exists, it is not without its shortcomings. Most prominently is the fact that due to technical inadequacies (e.g. the lack of a Frobenius endomorphism in characteristic $0$), our ground field $k$ can only be allowed to be a finite field should we want to recover the classical version of class field theory from the geometric version. However, for curves over finite fields, we can manage to obtain a rather complete story, neatly encapsulated in the following commutative diagram wherin the arrows are all isomorphisms of (abelian) groups:
            $$
                \begin{tikzcd}
                    {\Shv_{\bar{\Q}_{\ell}}^{\lisse, 1}(X)} && {\Eig\Shv_{\bar{\Q}_{\ell}}^1(\Bun_{\GL_1}(X))} \\
                    {\Rep^1_{\bar{\Q}_{\ell}}(\pi_1(X_{\fet}))} && {\scrA_{\GL_1}(X, \bar{\Q}_{\ell})}
                    \arrow["{\text{Theorem \ref{theorem: unramified_abelian_geometric_class_field_theory}}}", from=1-1, to=1-3]
                    \arrow["{\text{Theorem \ref{theorem: galois_representations_are_lisse_sheaves}}}"', from=1-1, to=2-1]
                    \arrow["{{\text{Proposition \ref{prop: hecke_characters_from_hecke_eigensheaves}}}}", from=1-3, to=2-3]
                    \arrow["{\text{Theorem \ref{theorem: artin_reciprocity_for_function_fields_over_finite_fields}}}", from=2-1, to=2-3]
                \end{tikzcd}
            $$
        In the diagram above, the top arrow is the geometric version of global class field theory, and the two side arrows tells us how the \say{decategorification} process through which we might obtain the classical version is done. As for the bottom arrow, it expresses Artin's Reciprocity Law for global function fields $K_X$ over finite fields (which we know to arise from smooth projective curves $X$ over finite fields, thanks to proposition \ref{prop: curves_and_function_fields}). This in turn implies the classical statement of class field theory (again, for global function fields over finite fields), which asserts that there is an equivalence of categories as below, relating finite unramified abelian extensions of $K_X$ and finite-index subgroups of the so-called id\`ele class group $\GL_1(K_X)\backslash\GL_1(\A_X)$ that contain $\GL_1(\bbO_K)$:
            $$
                \begin{tikzcd}
                    {\{\text{Finite unramified abelian extensions $L/K_X$ and $K_X$-algebra homomorphisms}\}^{\op}} \\
                    {\{\text{Finite-index subgroups of $\GL_1(K_X)\backslash\GL_1(\A_X)/\GL_1(\bbO_{X})$}\}}
                    \arrow["{\text{Theorem \ref{theorem: class_field_theory_for_global_function_fields_over_finite_fields}}}", from=1-1, to=2-1]
                \end{tikzcd}
            $$ 
        Now, in addition to the fact that it is possible to obtain the classical version of global class field theory via purely geometric methods, it should also be noted that the geometric formulation is arguably more fundamental, and that the classical version ought to be thought of as a corollary thereof and not the other way around (evident from the fact that theorem \ref{theorem: unramified_abelian_geometric_class_field_theory} implies theorem \ref{theorem: artin_reciprocity_for_function_fields_over_finite_fields}, which in turn implies theorem \ref{theorem: class_field_theory_for_global_function_fields_over_finite_fields}). Furthermore, this version of global class field theory, as it is naturally phrased in tems of sheaves instead of the underlying spaces, demonstrates to us the importance of representation theory in the study of Galois groups.

    \section{The Galois Side}
        \subsection{\'Etale fundamental groups}
            \subsubsection{Construction of \'etale fundamental groups}
                \'Etale fundamental groups of schemes as in \cite[Expos\'e V]{SGA1} are constructed using so-called \textbf{finite Galois categories}: in particular, for any fixed scheme $X$, the category of finite \'etale $X$-schemes is an instance of a finite Galois category, for whom the fundamental group is the group of natural automorphisms of the \textbf{fibre functor} based at a geometric point $\bar{x} \in X$:
                    $$\Fib_{\bar{x}}: (\Sch_{/X})_{\fet} \to \Fin\Sets$$
                    $$Y \mapsto Y(\bar{x})$$
                Below shall be a brief discussion of the construction (remark \ref{remark: finite_etale_schemes}) and key properties of \'etale fundamental groups of schemes (propositions \ref{prop: etale_fundamental_groups_do_not_depend_on_base_points} and \ref{prop: etale_homotopy_exact_sequence}), accompanied by certain relevant examples (cf. examples \ref{example: etale_fundamental_group_of_a_field} and \ref{example: etale_fundamental_group_of_a_curve}). For the sake of brevity, we have delegated the fine technicalities of this construction to \cite{}. The reader might also wish to consult \cite[Expos\'e V]{SGA1}, which was where \'etale fundamental groups of schemes were first discussed, as well as the more modern account in \cite[\href{https://stacks.math.columbia.edu/tag/0BQ6}{Tag 0BQ6}]{stacks}.
                
                We begin with the notion of finite Galois categories. Heuristically speaking, these are categories which are equivalent to $\pi\-\Fin\Sets$, the category of finite sets with actions of some profinite group $\pi$, which as we shall see, serves as the \say{fundamental group} of its associated finite Galois category. Due to this phenomenon, the fundamental group of a Galois category serves the same purpose as the classical fundamental groups of topological spaces, that being as a topological invariant. 
                \begin{definition}[Finite Galois categories] \label{def: finite_galois_categories}
                    \noindent
                    \begin{itemize}
                        \item \textbf{(Finite Galois categories):} A \textbf{finite Galois category} is defined via the data contained in a pair $(\calG, F)$ consisting of:
                        \begin{itemize}
                            \item a \textit{finitely complete and finitely cocomplete} small category $\calG$, wherein objects can all be written as finite coproducts of \textit{connected} objects\footnote{Objects $Y \in \calG$ such that the copresheaf $h^Y: \calG \to \Sets$ preserves all coproducts.}, and
                            \item a \textit{pro-representable} functor $F: \calG \to \Fin\Sets$, called the \textbf{fibre functor}, which we shall require to be exact and to reflect isomorphisms.
                        \end{itemize}
                        \item \textbf{(Galois objects):} An object $Y$ of a finite Galois category $\calG$ is a \textbf{Galois object} if and only if $Y/\Aut(Y)$ is terminal as an object of $\calG$ (which exists as $\calG$ is finitely complete by definition).
                        \item \textbf{(Galois functors):} A \textbf{Galois functor} is an exact functor $\Phi: \calG \to \calG'$ between finite Galois categories $(\calG, F), (\calG', F')$ which preserves connected objects and commute with the fibre functors in the following manner:
                            $$
                                \begin{tikzcd}
                                	\calG && {\calG'} \\
                                	& \Fin\Sets
                                	\arrow["F"', from=1-1, to=2-2]
                                	\arrow["{F'}", from=1-3, to=2-2]
                                	\arrow["\Phi", from=1-1, to=1-3]
                                \end{tikzcd}
                            $$
                    \end{itemize}
                \end{definition}
                \begin{definition}[Fundamental groups of finite Galois categories] \label{def: fundamental_groups_of_finite_galois_categories}
                    The \textbf{fundamental group} of a given finite Galois category $(\calG, F)$, denoted by $\pi_1(\calG, F)$, is defined to be $\Aut(F)$.
                \end{definition}
                \begin{proposition}[Fundamental groups of finite Galois categories are profinite] \label{prop: fundamental_groups_of_finite_galois_categories_are_profinite}
                    The fundamental group of any finite Galois category is profinite. In fact, given a finite Galois category $(\calG, F)$, its fundamental group arises as the limit of the cofiltered diagram $\{\Aut(F(Y_i))\}_{i \in \calI}$, where $\calI$ is any cofinal diagram of Galois objects in $\calG$.
                \end{proposition}
                \begin{proposition}[The Finite Galois Correspondence] \label{prop: finite_categorical_galois_correspondence}
                    \footnote{See also: \cite[\href{https://stacks.math.columbia.edu/tag/0BN4}{Tag 0BN4}]{stacks}.}\cite[Theorem 1.1.1]{} For any finite Galois category $(\calG, F)$, the functor $F: \calG \to \pi_1(\calG, F)\-\Fin\Sets$ is a Galois equivalence (we equip the latter with the forgetful functor to $\Fin\Sets$).
                \end{proposition}
                
                Now, the \'etale fundamental group is literally the fundamental group of a Galois category by construction. Namely, it is the fundamental group of the finite Galois category of pointed schemes that are finite \'etale over a pointed base scheme $(X, \bar{x})$ (cf. remark \ref{remark: finite_etale_schemes}). This, however, is not the only interesting feature that the \'etale fundamental group displays; one of the main reason why Grothendieck's \'etale fundamental group gained attention in the first place is that by taking the \'etale fundamental group of an algebraic variety, one actually obtains the absolute Galois of the global function field of said variety. As such, the \'etale fundamental group is a great tool that we can make use of in order to understand Galois groups; in fact, even the use of finite \'etale morphisms has a very concrete field-theoretic reason (cf. \cite[\href{https://stacks.math.columbia.edu/tag/00U3}{Tag 00U3}]{stacks}).
                
                Let us also note that because any topological space with a generic point is contractible \textit{a priori}, and since most interesting schemes (such as varieties) are integral and hence have generic points\footnote{This is more or less due to the fact that the zero ideal is prime in integral domains. For more details, see \cite[\href{https://stacks.math.columbia.edu/tag/01IS}{Tag 01IS}]{stacks}.}, the usual topological fundamental group is very unlikely to be able to distinguish the underlying topological spaces of two different schemes. Furthermore, because schemes are actually determined by their structure sheaves as opposed to their underlying topological spaces (e.g. if $\Nil(R)$ denotes the nilradical of a commutative ring then $\Spec R$ and $\Spec R/\Nil(R)$ will be non-isomorphic schemes whose underlying topological spaces are homeomorphic), we should not expect the topological fundamental group to be able to tell two schemes apart from one another. This is not to mention that in most cases, the underlying topological space of a scheme is not path-connected, so a fundamental group defined as the group of homotopy equivalence class of loops would not have made much sense.
                
                At the end of the section on Grothendieck's \'etale fundamental group, we will be giving several arithmetically interesting examples (notably, in corollary \ref{coro: unramified_galois_representations_induced_by_the_abel_jacobi_map}), such as computations of the \'etale fundamental groups of the prime spectrum of a field and the projective line $\P^1$.
                \begin{remark}[Finite-\'etale schemes] \label{remark: finite_etale_schemes}
                    For any given by scheme $X$, the small category $(\Sch_{/X})_{\fet}$ of finite-\'etale $X$-schemes is a category wherein:
                        \begin{itemize}
                            \item all finite limits and all finite colimits exist, and
                            \item all objects can be written as a (possibly empty) finite coproduct of connected objects, which happen to be schemes that are \'etale over $X$.  
                        \end{itemize}
                    (for a detailed proof, see \cite[\href{https://stacks.math.columbia.edu/tag/0BN9}{Tag 0BN9}]{stacks}) so should we be able to define a fibre functor $(\Sch_{/X})_{\fet} \to \Fin\Sets$, we will have succeeded in putting a finite Galois category structure on $(\Sch_{/X})_{\fet}$. As a matter of fact, such a well-defined fibre functor has good reasons to exist: it is an easy consequence of \cite[\href{https://stacks.math.columbia.edu/tag/00U3}{Tag 00U3}]{stacks} that for any fixed geometric point $\bar{x} \in X$ (corresponding to an algebraic closure\footnote{Certain sources consider geometric points to correspond to separable closures. For us, however, geometric points are algebraically closed fields $K$ so that $\Spec K$ be a Galois object of $(\Sch_{/\Spec K})_{\fet}$. In practice this choice usually does not matter, since we will mostly work over perfect field, and separable closures of perfect fields are algebraically closed.} $\bar{\kappa}_x$ of the residue field of $x \in X$), one has:
                        $$(\Spec \bar{\kappa}_x)_{\fet} \cong \Fin\Sets$$
                    (the forward direct simply involves taking the underlying set, and the inverse functors is given by $I \mapsto \coprod_{i = 1}^{|I|} \Spec \bar{\kappa}_x$) and so for any $k$-scheme $X$, one has the following canonical defined functor:
                        $$(\Sch_{/X})_{\fet} \to (\Sch_{/\Spec \bar{\kappa}_x})$$
                        $$Y \mapsto Y_{\bar{x}}$$
                    where $Y_{\bar{x}} \cong Y \x_X \Spec \bar{\kappa}_x$; one can then take the underlying set of $Y_{\bar{x}}$ to get the following trivially left-exact\footnote{... and hence pro-representable (cf. \cite[Proposition 3.1]{grothendieck_fga_2}).} functor:
                        $$\Fib_{\bar{x}}: (\Sch_{/X})_{\fet} \to \Fin\Sets$$
                        $$Y \mapsto |Y_{\bar{x}}|$$
                    We should also verify that the sets $|Y_{\bar{x}}|$ are indeed finite. To this end, let us first apply the fact that pullbacks of \'etale morphisms are \'etale to see that if $Y$ is affine over $X$ then $Y_{\bar{x}}$ will have to be the spectrum of an \'etale $\bar{\kappa}_x$-algebra; however, according to \cite[\href{https://stacks.math.columbia.edu/tag/00U3}{Tag 00U3}]{stacks}, this means that $Y_{\bar{x}} \cong \Spec (\bar{\kappa}_x)^{\oplus N}$ for some finite $N$. The locality of \'etale-ness and the finiteness of $Y$ as an $X$-scheme then tells us that in general, $Y_{\bar{x}}$ must be a finite disjoint union of affine schemes of the form $\Spec (\bar{\kappa}_x)^{\oplus N}$, meaning that $Y_{\bar{x}} \cong \Spec (\bar{\kappa}_x)^{\oplus N'}$ for some finite $N'$. The set $|Y_{\bar{x}}|$ is therefore always finite. One also sees that an immediate consequence of this proof is that $\Fib_{\bar{x}}$ necessarily \textit{reflects isomorphisms} and is \textit{right-exact}. 
                \end{remark}
                
                The discussion above allows us to make the following important definition.
                \begin{definition}[\'Etale fundamental groups] \label{def: etale_fundamental_groups}
                    For any scheme $X$ with a fixed geometric point $\bar{x}$, the pair $((\Sch_{/X})_{\fet}, \Fib_{\bar{x}})$ as in remark \ref{remark: finite_etale_schemes} defines a finite Galois category in the sense of \cite[Section 1]{}. Its fundamental group is commonly denoted by $\pi_1(X_{\fet}, \bar{x})$ and called the \textbf{\'etale fundamental group} of $X$ based at $\bar{x}$.
                \end{definition}
                
                We can then apply proposition \ref{prop: finite_categorical_galois_correspondence} directly to get the following topological characterisation of connected schemes via their \'etale fundamental groups.
                \begin{theorem}[The Geometric Galois Correspondence] \label{theorem: geometric_galois_correspondence}
                    For $(X, \bar{x})$ a pointed connected scheme, the functor $\Fib_{\bar{x}}: (\Sch_{/X})_{\fet} \to \Fin\Sets$ as in remark \ref{remark: finite_etale_schemes} above is a Galois equivalence of finite Galois categories. 
                \end{theorem}
                \begin{corollary}
                    Should $H$ be a finite-index (hence open, since $\pi_1(X_{\fet}, \bar{x})$ is profinite \textit{a priori}; cf. proposition \ref{prop: fundamental_groups_of_finite_galois_categories_are_profinite}) normal subgroup of $\pi_1(X_{\fet}, \bar{x})$ and $(X^H, \bar{x}^H)$ be the corresponding Galois $X$-scheme with a choice of base point $\bar{x}^H$ lying over $\bar{x}$, then $\pi_1(X^H_{\fet}, \bar{x}^H) \cong H$. In fact, for any connected pointed scheme $(X, \bar{x})$, there is an equivalence of categories:
                        $$\{\text{Finite \'etale $X$-schemes $Y$ with base points $\bar{y}$ lying over $\bar{x}$}\}$$
                        $$\cong$$
                        $$\{\text{Finite-index subgroups of $\pi_1(X_{\fet}, \bar{x})$}\}$$
                    which restricts down to:
                        $$\{\text{Finite \'etale Galois $X$-schemes $Y$ with base points $\bar{y}$ lying over $\bar{x}$}\}$$
                        $$\cong$$
                        $$\{\text{Finite-index \textit{normal} subgroups of $\pi_1(X_{\fet}, \bar{x})$}\}$$
                \end{corollary}
                \begin{example}[The \'etale fundamental group of a field] \label{example: etale_fundamental_group_of_a_field}
                    As a sanity check, note that if $K$ is a field then finite-\'etale Galois schemes over $\Spec K$ shall be of the form $\Spec L \to \Spec K$, where $L/K$ is a finite Galois extension, and as a consequence, there are the following equivalences of categories\footnote{In fact, these categories are lattices with meets being intersections of fields/subgroups and joins being the constructions of composite fields/subgroups.}, which demonstrate that theorem \ref{theorem: geometric_galois_correspondence} directly generalises the classical Galois Correspondence:
                        $$\{\text{Finite-index normal subgroups of $\pi_1((\Spec K)_{\fet})$}\}$$
                        $$\cong$$
                        $$\{\text{Finite \'etale Galois schemes over $\Spec K$}\}$$
                        $$\cong$$
                        $$\{\text{Finite Galois extensions of $K$}\}^{\op}$$
                        $$\cong$$
                        $$\{\text{Finite-index normal subgroups of $\Gal(\bar{K}/K)$}\}$$
                \end{example}
                \begin{example}[The \'etale fundamental group of a curve] \label{example: etale_fundamental_group_of_a_curve}
                    Let $k$ be a field. If $X$ is a connected non-singular projective curve over $\Spec k$ with function field $K$, then there is a canonical equivalence between the categories of finite Galois extensions of $K$ and finite \'etale Galois $X$-schemes, which are precisely dominant rational maps whose associated function field extensions are Galois thanks to proposition \ref{prop: curves_and_function_fields}. Through this, it is easy to see that:
                        $$\pi_1(X_{\fet}) \cong \Gal(\bar{K}/K)$$
                    For instance, we have:
                        $$\pi_1((\P^1_k)_{\fet}) \cong \Gal(\bar{k}/k)$$
                    (since the function field of $\P^1_k$ is $k(t)$), which tells us that $\P^1_k$ is simply \'etale-connected if and only if $k$ is algebraically closed (since $\Gal(\bar{k}/k)$ is \textit{a fortiori} trivial in that case). 
                \end{example}
            
            \subsubsection{Properties of \'etale fundamental groups}
                Now, let us make sure that the \'etale fundamental group $\pi_1(X_{\fet}, \bar{x})$ as defined in definition \ref{def: etale_fundamental_groups} is meaningful as a formal construction. Namely, we would like to know the behaviours of $\pi_1(X_{\fet}, \bar{x})$ when we change the base point and when we base-change (cf. proposition \ref{prop: etale_fundamental_groups_do_not_depend_on_base_points}), as well as whether or not \'etale fibrations induce homotopy exact sequences of fundamental groups (cf. proposition \ref{prop: etale_homotopy_exact_sequence}). 
                \begin{proposition}[\'Etale fundamental group do not depend on base points] \label{prop: etale_fundamental_groups_do_not_depend_on_base_points}
                    \cite[\href{https://stacks.math.columbia.edu/tag/0BQA}{Tag 0BQA}]{stacks} Let $f: Y \to X$ be a morphism of connected qcqs\footnote{quasi-compact and quasi-separated} schemes such that the base change functor $X' \mapsto X' \x_X Y$ is an equivalence of Galois categories between $(\Sch_{/X})_{\fet}$ and $(\Sch_{/Y})_{\fet}$. Then, for any choice of geometric points $\bar{x} \in X$ and $\bar{y} \in Y$, one has the following isomorphism of \'etale fundamental groups $\pi_1(X_{\fet}, \bar{x}) \cong \pi_1(Y_{\fet}, \bar{y})$.
                \end{proposition}
                \begin{corollary}[Uniqueness of \'etale fundamental groups] \label{coro: etale_fundamental_group_uniqueness}
                    For any connected qcqs scheme $X$ and any pair of possibly distinct geometric points $\bar{x}, \bar{x}' \in X$, one has any isomorphism of \'etale fundamental groups $\pi_1(X_{\fet}, \bar{x}) \cong \pi_1(X_{\fet}, \bar{x}')$, and therefore it makes sense to only speak of \textit{the} fundamental group of $X$, which we shall denote by $\pi_1(X_{\fet})$.
                \end{corollary}
                
                \begin{proposition}[The \'etale homotopy exact sequence] \label{prop: etale_homotopy_exact_sequence}
                    \cite[\href{https://stacks.math.columbia.edu/tag/0C0J}{Tag 0C0J}]{stacks} Let $X$ be a connected scheme. If $f: Y \to X$ be a flat proper morphism of finite presentation whose geometric fibres $Y_{\bar{x}}$ are connected and reduced, then for any geometric point $\bar{x} \in X$, there exists a right-exact sequence of groups as follows:
                        $$\pi_1((Y_{\bar{x}})_{\fet}) \to \pi_1(Y_{\fet}) \to \pi_1(X_{\fet}) \to 1$$
                \end{proposition}
        
        \subsection{\texorpdfstring{$\ell$}{}-adic sheaves and Grothendieck's Galois Theory}
            The process of geometrising class field theory begins with the geometrisation of Galois representations, which thanks to a combination of proposition \ref{prop: curves_and_function_fields} and theorem \ref{theorem: geometric_galois_correspondence} are more or less the same as representations of \'etale fundamental groups of schemes. As such, we seek a geometrisation of representations of \'etale fundamental groups and this will be done via establishing a connection between said representations and a certain kind of abelian sheaves - called \textbf{lisse $\bar{\Q}_{\ell}$-adic sheaves} - on \say{the curve of global class field theory} (cf. convention \ref{conv: automorphic_side_conventions}); our efforts shall culminate in theorem \ref{theorem: galois_representations_are_lisse_sheaves}. In addition, we shall be collecting several facts about the algebra of lisse $\bar{\Q}_{\ell}$-adic sheaves for later use. In particular, we want to keep in mind that lisse $\bar{\Q}_{\ell}$-adic sheaves behave well around tensor products as well as pullbacks and pushforwards.  
        
            \subsubsection{Artin-Rees categories and adic sheaves}
                We begin by building our way up to the notion of \textbf{lisse $\bar{\Q}_{\ell}$-adic sheaves}, which are essential for the statement of the main theorem of this section, namely theorem \ref{theorem: galois_representations_are_lisse_sheaves}. For subtle technical reasons, this involves some abstraction in the form of so-called \textbf{Artin-Rees categories}, which are direct generalisations of categories of adically complete modules over Noetherian rings. Our main reference is \cite[Subsection 1.4]{conrad_etale_cohomology} and \cite[Expos\'e V]{sga5}, which was where the constructions below first appeared.
            
                \begin{definition}[Artin-Rees categories] \label{def: artin_rees_categories}
                    The \textbf{Artin-Rees category} associated to an abelian category $\calA$ is the full subcategory of $\calA_{\bullet} := \Pro(\calA)$ spanned by cofiltered diagrams $\{M_n\}_{n \in \Z}$; we denote it by $\calA_{\bullet}^{\AR}$. Of particular interest are the so-called \textbf{null systems}, which are objects $\{M_n\}_{n \in \Z} \in \calA_{\bullet}^{\AR}$ such that there exists $\nu \in \N$ so that for all $n \in \Z$ the morphism $M_n \to M_{n + \nu}$ is zero.
                \end{definition}
                
                \begin{proposition}[Artin-Rees categories are linear and abelian] \label{prop: artin_rees_categories_are_linear_and_abelian}
                    \cite[Expos\'e V, Propositions 2.2.2 et 2.4.1]{sga5} For any (locally finite) $\Lambda$-linear\footnote{I.e. if hom-sets of $\calA$ are (locally finite) $\Lambda$-modules (e.g. when $\calA \cong \Lambda\mod$).} abelian category $\calA$, the associated Artin-Rees category $\calA_{\bullet}^{\AR}$ is also a (locally finite) $\Lambda$-linear abelian category, with zero objects being the null systems. In fact, the Artin-Rees category $\calA_{\bullet}^{\AR}$ is the localisation\footnote{In the sense of \cite[\href{https://stacks.math.columbia.edu/tag/02MS}{Tag 02MS}]{stacks}.} of $\calA_{\bullet}$ at the thick\footnote{Cf. \cite[\href{https://stacks.math.columbia.edu/tag/02MO}{Tag 02MO}]{stacks}.} subcategory of null systems, meaning that an isomorphism in $\calA_{\bullet}^{\AR}$ (henceforth referred to as an \textbf{AR-isomorphism}) is a morphism in $\calA_{\bullet}$ whose kernel and cokernel are null. 
                \end{proposition}
                
                \begin{convention}[The setting for adic sheaves] \label{conv: l_adic_sheaves_conventions}
                    For our purposes, $\calX$ shall be a scheme that is locally of finite type\footnote{Although $\calX$ might actually be an algebraic stack of finite type over $S$; for details, see \cite{laszlo_olsson_adic_sheaves_on_artin_stacks_1} and \cite{laszlo_olsson_adic_sheaves_on_artin_stacks_2}. It should also be noted that in \cite[Subsection 1.4]{conrad_etale_cohomology}, it was only required that $\calX$ would be Noetherian, which is not sufficient for us, as $\Bun_{\GL_1}(X)$ is merely locally of finite type, and hence only locally Noetherian \textit{a priori}.} over a base scheme $S$ that is affine, regular, Noetherian\footnote{Note that this implies that $\calX$ is locally Noetherian (cf. \cite[\href{https://stacks.math.columbia.edu/tag/01T6}{Tag 01T6}]{stacks}).} and of dimension $\leq 1$, and of characteristic $p \geq 0$; moreoever, we would like to work under the assumption that every finite-type $S$-scheme $T$ is also of finite cohomological dimension. In addition, $\Lambda$ shall be a discrete valuation ring of mixed characteristic $(0, \ell)$ (for some auxiliary prime $\ell \not = p$) with maximal ideal $\m$, and fraction field $E$.
                \end{convention}
                
                \begin{definition}[Torsion objects in tensor categories] \label{def: torsion_objects_in_tensor_categories}
                    Let $A$ be a commutative ring, $I$ be an ideal of $A$, and $\calA$ be an $A$-linear category. Then, the subcategory of $\calA$ spanned by $I$-torsion objects is the one wherein the hom-sets are $\Hom_{\calA/I}(M, N) \cong \Hom_\calA(M, N) \tensor_A A/I$.
                \end{definition}
                \begin{definition}[Adic objects and lisse objects of Artin-Rees categories] \label{def: adic_objects_and_lisse_objects_of_artin_rees_categories}
                    Consider the Artin-Rees category $\calA_{\bullet}^{\AR}$ associated to a locally finite $\Lambda$-linear abelian category $\calA$. 
                        \begin{enumerate}
                            \item \textbf{(Adic objects):} $\calA_{\bullet}^{\AR}$ admits a full subcategory, denoted by $\calA_{\bullet}^{\ad}$, whose objects $\{M_n\}_{n \in \Z}$ are such that:
                                \begin{itemize}
                                    \item $M_n \cong 0$ for all $n < 0$,
                                    \item $M_n$ is $\m^{n + 1}$-torsion for all $n \geq 0$, and
                                    \item the canonical maps $\Lambda/\m^{n + 2} \tensor_{\Lambda} \Hom_{\calA_{\bullet}^{\AR}}(M_{\bullet}, N_{\bullet}) \to \Hom_{\calA_{\bullet}^{\AR}}(M_{\bullet}, N_{\bullet})/\m^n$ are isomorphisms of $\Lambda/\m^{n + 1}$-modules for all $n \geq 0$.
                                \end{itemize}
                            Objects of this full subcategory are said to be \textbf{$\m$-adic}.
                            \item \textbf{(Lisse objects):} Let $\calA_{\bullet}^{\fin}$ denote the category of Artin-Rees projective systems of objects of $\calA$ which are simultaneously Artinian and Noetherian. Objects of the category\footnote{Here, the intersection is understood to be at both the level of objects and that of morphisms.} $\calA_{\bullet}^{\lisse} := \calA_{\bullet}^{\ad} \cap \calA_{\bullet}^{\fin}$ are then said to be \textbf{lisse}. 
                        \end{enumerate}
                \end{definition}
                \begin{proposition}[Adic categories are linear and abelian] \label{prop: adic_categories_are_linear_and_abelian}
                    Let $\calA$ be a locally finite $\Lambda$-linear abelian category. Then, we shall have a tower $\calA_{\bullet}^{\lisse} \subset \calA_{\bullet}^{\ad} \subset \calA_{\bullet}^{\AR}$ of locally finite $\Lambda$-linear abelian categories, wherein the inclusions are fully faithful exact functors.
                \end{proposition}
                \begin{example}[Adic and lisse $\Lambda$-sheaves] \label{example: adic_sheaves}
                    Recall first of all that the category of constructible \'etale sheaves of $\Lambda$-modules on a Noetherian scheme - of which the category $\Lambda\mod^{\cons}(\calX_{\et})$ of constructible sheaves of $\Lambda$-modules on $\calX_{\et}$ is a special case - is a locally finite $\Lambda$-linear closed monoidal category (to see why, combine \cite[Propositions 3.20 and 3.22]{behrend_l_adic_sheaves_for_algebraic_stacks}). Then, the category of adic constructible $\m$-adic sheaves on $\calX$ (also called constructible $\Lambda$-sheaves), commonly denoted by $\Shv_{\Lambda}^{\ad}(\calX)$, is nothing but $\Lambda\mod^{\cons}(\calX_{\et})_{\bullet}^{\ad}$, and the category of lisse $\m$-adic sheaves on $\calX$, denoted by $\Shv_{\Lambda}^{\lisse}(\calX)$ is simply $\Lambda\mod^{\cons}(\calX_{\et})_{\bullet}^{\lisse}$.
                \end{example}
                
                \begin{definition}[$E$-objects] \label{def: E_objects}
                    Let $\calA$ be a locally finite $\Lambda$-linear abelian category. Then, we can define the associated category of \textbf{$E$-objects} to be the localisation $\calA \tensor_{\Lambda} E$ at $E$-linear morphisms, i.e. we define:
                        $$\Hom_{\calA \tensor_{\Lambda} E}(M, N) \cong \Hom_{\calA}(M, N) \tensor_{\Lambda} E$$
                    It can be easily verified that $\calA \tensor_{\Lambda} E$ is locally finite $E$-linear and abelian.
                \end{definition}
                \begin{definition}[$\bar{E}$-objects] \label{def: bar_E_objects}
                    Let $\calA$ be a locally finite $\Lambda$-linear abelian category. We define the associated category $\calA \tensor_{\Lambda} \bar{E}$ of \textbf{$\bar{E}$-objects} via:
                        $$\Hom_{\calA \tensor_{\Lambda} \bar{E}}(M, N) \cong \underset{\text{$E'/E$ finite extensions}}{\colim} \Hom_{\calA}(M, N) \tensor_{\Lambda} E'$$
                    Using the fact that associated categories of $E$-objects are abelian and $E$-linear, one can show that associated categories of $\bar{E}$-objects are abelian and $\bar{E}$-linear\footnote{The one technicality to keep in mind is that for finite-dimensional vector spaces, filtered colimits commute with kernels.}. 
                \end{definition}
                \begin{example}[Adic and lisse $E$-sheaves and $\bar{E}$-sheaves] \label{example: E_sheaves}
                    Because $\Shv_{\Lambda}^{\ad}(\calX)$ and $\Shv_{\Lambda}^{\lisse}(\calX)$ are locally finite $\Lambda$-linear and abelian (cf. proposition \ref{prop: adic_categories_are_linear_and_abelian}), one can define the categories of constructible adic $E$-sheaves and that of lisse $E$-sheaves on $\calX$ to be $\Shv_E^{\ad}(\calX) \cong \Shv_{\Lambda}^{\ad}(\calX) \tensor_{\Lambda} E$ and $\Shv_E^{\lisse}(\calX) \cong \Shv_{\Lambda}^{\lisse}(\calX) \tensor_{\Lambda} E$. Adic and lisse $\bar{E}$-sheaves can thus also be defined.
                \end{example}
                
            \subsubsection{Operations with adic sheaves}
                The following results shall be used throughout the rest of the paper without any explicit mention (that is, with the notable exception of theorem \ref{theorem: galois_representations_are_lisse_sheaves}). For details, we refer the reader to \cite[Sections 6-8]{laszlo_olsson_adic_sheaves_on_artin_stacks_2} (which is a sequel to \cite{laszlo_olsson_adic_sheaves_on_artin_stacks_1}) and \cite[Sections II.7-II.10]{kiehl_weissauer_weil_conjecture_perverse_sheaves_and_l_adic_fourier_transform}.
                
                \begin{proposition}[Tensor products of constructible adic objects] \label{prop: tensor_products_of_constructible_adic_objects}
                    \noindent
                    \begin{enumerate}
                        \item \cite[Proposition 6.1]{laszlo_olsson_adic_sheaves_on_artin_stacks_2} For any locally finite $\Lambda$-linear closed monoidal abelian category $(\calA, \tensor, \1)$, the associated categories $\calA_{\bullet}^{\ad}$ and $\calA_{\bullet}^{\lisse}$ of adic and lisse systems are monoidal with respect to term-wise tensor products $M_{\bullet} \tensor N_{\bullet} \cong \{M_n \tensor N_n\}_{n \in \Z}$. In fact, they both embed via fully faithful monoidal exact $\Lambda$-linear functors into $\calA_{\bullet}^{\AR}$, which is also monoidal with respect to term-wise tensor products. 
                        \item \cite[Theorem III.12.2 and Appendix A]{kiehl_weissauer_weil_conjecture_perverse_sheaves_and_l_adic_fourier_transform} Furthermore, $\calA_{\bullet}^{\ad}$ and $\calA_{\bullet}^{\lisse}$ (respectively, $\calA_{\bullet}^{\ad} \tensor_{\Lambda} \bar{E}$ and $\calA_{\bullet}^{\lisse} \tensor_{\Lambda} \bar{E}$ for any choice of algebraic closure $\bar{E}/E$) are closed monoidal categories with respect to these tensor products.
                    \end{enumerate}
                \end{proposition}
                \begin{example}[Tensor products of constructible adic sheaves] \label{def: tensor_products_of_constructible_adic_sheaves}
                    $\Lambda\mod^{\cons}(\calX_{\et})$ is a locally finite closed monoidal category with respect to tensor products over the constant sheaf $\underline{\Lambda}$ (cf. \cite[\href{https://stacks.math.columbia.edu/tag/093P}{Tag 093P}]{stacks}), so by proposition \ref{prop: tensor_products_of_constructible_adic_objects}, the category $\Shv_{\bar{\Q}_{\ell}}^{\lisse}(\calX)$ of lisse $\bar{\Q}_{\ell}$ sheaves on $\calX$ shall be closed monoidal with respect to the constant sheaf $\underline{\bar{\Q}_{\ell}}$.
                \end{example}
                
                \begin{proposition}[$*$-pullbacks and $*$-pushforwards] \label{prop: *_pullbacks_and_pushforwards_of_l_adic_sheaves}
                    For $f: \calX \to \calY$ a morphism of finite type between $S$-schemes that are locally of finite type (with $S$ as in convention \ref{conv: l_adic_sheaves_conventions}), there is an adjoint equivalence whose components $f^*$ and $f_*$ are computed term-wise as in \cite[\href{https://stacks.math.columbia.edu/tag/03PZ}{Tag 03PZ}]{stacks} and  \cite[\href{https://stacks.math.columbia.edu/tag/03PV}{Tag 03PV}]{stacks} respectively:
                        $$
                            \begin{tikzcd}
                            	{\Shv_{\bar{\Q}_{\ell}}^{\lisse}(\calX)} & {\Shv_{\bar{\Q}_{\ell}}^{\lisse}(\calY)}
                            	\arrow[""{name=0, anchor=center, inner sep=0}, "{f_*}"', bend right, from=1-1, to=1-2]
                            	\arrow[""{name=1, anchor=center, inner sep=0}, "{f^*}"', bend right, from=1-2, to=1-1]
                            	\arrow["\dashv"{anchor=center, rotate=-90}, draw=none, from=1, to=0]
                            \end{tikzcd}
                        $$
                    Furthermore, the functors $f^*$ and $f_*$ both preserve (external) tensor products of lisse $\bar{\Q}_{\ell}$-sheaves.
                \end{proposition}
                    \begin{proof}
                        Combine \cite[Proposition 8.3]{laszlo_olsson_adic_sheaves_on_artin_stacks_2} with \cite[Theorem II.7.1]{kiehl_weissauer_weil_conjecture_perverse_sheaves_and_l_adic_fourier_transform}
                    \end{proof}
                
                \begin{definition}[Stalks of constructible adic sheaves] \label{def: stalks_of_constructible_adic_sheaves}
                    \cite[Definition 1.4.4.3]{conrad_etale_cohomology} Let $\bar{x} \in \calX$ be a geometric point and $\calF_{\bullet} \in \Shv_{\bar{\Q}_{\ell}}^{\ad}(\calX)$ be a constructible $\bar{\Q}_{\ell}$-sheaf on $\calX$. Then, the \textbf{stalk} at $\bar{x}$ shall be given by $\underset{n \in \N}{\lim} (\calF_n)_{\bar{x}}$, with each term $(\calF_n)_{\bar{x}}$ being computed as stalks of \'etale sheaves (cf. \cite[\href{https://stacks.math.columbia.edu/tag/040R}{Tag 040R}]{stacks}).
                \end{definition}
                \begin{remark}
                    As tensor products and $*$-pullbacks/pushforwards of lisse $\bar{\Q}_{\ell}$ are computed term-wise, taking stalks of lisse $\bar{\Q}_{\ell}$-sheaves commutes with those operations.
                \end{remark}
                
                \begin{convention}[Continuous linear representations] \label{conv: continuous_linear_representations}
                    From now on, if $G$ is a topological group and $E$ is a topological field then we will be writing $\Rep_E(G)$ for the category of \textit{continuous} $E$-linear representations of $G$. In fact, all representations shall be implicitly assumed to be continuous. Also, for all $n \geq 1$, $\Rep_E^n(G)$ shall be used to denote the subcategory of $n$-dimensional $E$-linear representations of $G$.
                \end{convention}
                \begin{theorem}[Galois representations are lisse $\bar{\Q}_{\ell}$-sheaves] \label{theorem: galois_representations_are_lisse_sheaves}
                    \cite[Theorem 1.4.5.7]{conrad_etale_cohomology} Let $\calX$ be a connected Noetherian scheme and fix a geometric point $\bar{x} \in \calX$. Then, there exists a monoidal equivalence given by $\calF \mapsto \calF_{\bar{x}}$, from the symmetric monoidal category $\Shv_{\bar{\Q}_{\ell}}^{\lisse}(\calX)$ of lisse $\bar{\Q}_{\ell}$-sheaves to the symmetric monoidal category $\Rep_{\bar{\Q}_{\ell}}^{\fin}(\pi_1(\calX_{\fet}))$ of finite-dimensional continuous $\bar{\Q}_{\ell}$-linear representations of $\pi_1(\calX_{\fet})$.
                \end{theorem}

    \section{The Automorphic Side}
        \subsection{Divisors and the Abel-Jacobi map}
            \subsubsection{Effective divisors}
                \begin{definition}[Divisors] \label{def: divisors}
                    Let $Y$ be a scheme. An \textbf{effective (Cartier) divisor}\footnote{Historically referred to as a \say{modulus}.} on $Y$ is then a closed subscheme $D \subset Y$ whose ideal sheaf $\calI_D$ is a line bundle on $Y$. Its \textbf{degree} is the \href{https://stacks.math.columbia.edu/tag/0AYQ}{\underline{degree}} of the line bundle $\calI_D$.  
                \end{definition}
                
                \begin{convention}
                    The set of effective divisors (respectively, effective divisors of degree $d$) shall be denoted by $|\Div_Y^{\eff}|$ (respectively, $|\Div_Y^{\eff, (d)}|$), and as a straightforward consequence of the definition of effective divisors, it is precisely the set of invertible\footnote{A quasi-coherent ideal sheaf $\calJ \subset \calO_Y$ is invertible if and only if its local sections $\calJ(V)$ are principal ideals of $\calO_Y(V)$.} quasi-coherent $\calO_Y$-ideals (respectively, quasi-coherent $\calO_Y$-ideals of degree $d$).
                \end{convention}
                \begin{convention}
                    Given any effective divisor $D \subset Y$, any integer $n$, and any quasi-coherent $\calO_X$-module $\E$, let us write $\E(nD) \cong \E \tensor_{\calO_Y} \calI_D^{\tensor (-n)}$ (wherein $\calI_D^{\tensor (-n)} \cong (\calI_D^{\tensor (-1)})^{\tensor n}$). In particular, note that $\calO_Y(-D) \cong \calI_D$.
                \end{convention}
                \begin{remark}[Addition of effective divisors] \label{remark: adding_effective_divisors}
                    Let $Y$ be a scheme. Because effective divisors are line bundles, they are trivially flat\footnote{Given any line bundle $\calL$ on $Y$, the functor $- \tensor_{\calO_Y} \calL$ is an auto-equivalence of $\Coh_Y$, which is an abelian category, so $- \tensor_{\calO_Y} \calL$ is automatically (left-)exact.}, so given any pair of effective divisors $D, D' \subset Y$, one can define a new effective divisor $D + D'$ corresponding to the $\calO_Y$-ideal $\calI_D \calI_{D'}$, which is isomorphic to $\calI_D \tensor_{\calO_{X/k}} \calI_{D'}$ due to flatness, and hence $\deg(D + D') = \deg D + \deg D'$.
                \end{remark}
                
                \begin{convention}[The setting for geometric class field theory] \label{conv: automorphic_side_conventions}
                    \noindent
                    \begin{itemize}
                        \item First of all, let us fix once and for all a smooth, projective, and geometrically connected curve over an algebraically closed field $k$ (we will actually relax the algebraic closure constraint on $k$ later, as this is simply a matter of convenience).
                        \item For us, $\Bun_{\GL_1}(X)$ shall denote the moduli space of line bundles on $X$ (cf. \cite[\href{https://stacks.math.columbia.edu/tag/0372}{Tag 0372}]{stacks}). Traditionally, this is usually referred to as the \say{Picard stack} (in this case, it is actually a scheme, not just a stack; see remark \ref{remark: geometry_of_the_picard_stack}) and denoted by $\Pic_{X/k}$, but we opt for the notation $\Bun_{\GL_1}(X)$ because in the wider context of the Geometric Langlands Programme, one works with $\Bun_G(X)$ for $G$ a general connected reductive group (of which $\GL_1$ is a special case). 
                    \end{itemize}
                \end{convention}
                \begin{remark}[The geometry of $\Bun_{\GL_1}$] \label{remark: geometry_of_the_picard_stack}
                    Since we are working with a smooth projective curve $X$ over an algebraically closed field (cf. convention \ref{conv: automorphic_side_conventions}) and hence over a separably closed field, the functor $\Bun_{\GL_1}$ is \textit{a priori} represented by a scheme (cf. \cite[\href{https://stacks.math.columbia.edu/tag/0B9Z}{Tag 0B9Z}]{stacks}). Furthermore, if the genus of $X$ is $g \geq 0$, then one will have a decomposition $\Bun_{\GL_1}(X) \cong \coprod_{d \in \Z} \Bun_{\GL_1}^{(d)}(X)$, wherein each $\Bun_{\GL_1}^{(d)}(X)$ is the moduli scheme of line bundles of degree $d$  on $X$, which is a proper smooth variety of dimension $g$ over $\Spec k$ (cf. \cite[\href{https://stacks.math.columbia.edu/tag/0BA0}{Tag 0BA0}]{stacks}).
                \end{remark}
                
                \begin{remark}[Moduli space of effective divisors] \label{remark: moduli_space_of_effective_divisors}
                    For any base scheme $S$ and any $S$-scheme $Y$, the \textbf{Hilbert functor} of closed subschemes of degree $d \in \Z$ is the presheaf:
                        $$\Hilb_{Y/S}^{(d)}: \Sch_{/S}^{\op} \to \Sets$$
                        $$T \mapsto \{\text{Finite locally free closed subschemes $D \subset Y_T$ of degree $d$}\}$$
                    Should $Y$ be a geometrically irreducible smooth proper (respectively projective) curve over a field $C$ then interestingly, not only are finite locally free closed subschemes $D \subset Y_{C'}$ of degree $d$ precisely the effective divisors of degree $d$ on $Y_{C'}$ for any field extension $C'/C$ (cf. \cite[\href{https://stacks.math.columbia.edu/tag/0B9D}{Tag 0B9D}]{stacks}), but also, one has a bijection:
                        $$|\Div_{Y_{C'}/C'}^{\eff, (d)}| \cong \Hilb_{Y/C}^{(d)}(C')$$
                    between the set of $C'$-rational points of $\Hilb_{Y/C}^{(d)}$ and that of degree-$d$ effective divisors on $Y_{C'}$ (cf. \cite[\href{https://stacks.math.columbia.edu/tag/0B9I}{Tag 0B9I}]{stacks}). From this, one see that should $Y$ be proper (respectively Zariski-locally projective) and flat over some arbitrary base scheme $S$, and if its fibres $Y_s$ over points $s \in |S|$ are geometrically irreducible smooth proper (respectively projective) curves, then $\Hilb_{Y/S}^{(d)}$ would be the moduli space of degree-$d$ effective divisors on $Y$; thus, for proper (respectively Zariski-locally projective) and flat morphisms $Y \to S$, let us suggestively write $\Div_{Y/S}^{\eff, (d)}$ instead of $\Hilb_{Y/S}^{(d)}$. It is known moreover that $\Hilb_{Y/C}^{(d)}$ is represented by a smooth proper variety of dimension $d$ over $\Spec C$ (cf. \cite[\href{https://stacks.math.columbia.edu/tag/0B9I}{Tag 0B9I}]{stacks}). By putting everything together, one obtains a moduli space $\Div_{Y/C}^{\eff, (d)} \in (Y/C)_{\et}$ parametrising degree-$d$ effective divisors on $Y$, represented by a smooth proper variety of dimension $d$ over $\Spec C$ and naturally isomorphic to $\Hilb_{Y/C}^{(d)}$.
                \end{remark}
                
                It should also be noted that what we have just discussed is not the only way to show that $\Div_{X/k}^{\eff, (d)}$ is represented by a smooth proper variety of dimension $d$. Remark \ref{remark: moduli_space_of_effective_divisors} serves more as a demonstration that there \textit{should} be a moduli space of effective divisors (of a given degree $d$), rather than that there \textit{is} one. In fact, by combining remark \ref{remark: quotients_of_schemes_by_finite_group_schemes}, lemma \ref{lemma: smoothness_of_symmetric_powers}, and proposition \ref{prop: symmetric_powers_of_curves_parametrise_divisors}, we shall see that the functor $\Div_{X/k}^{\eff, (d)}$ is represented by a smooth proper variety of (pure) dimension $d$ by virtue of being naturally isomorphic to the functor of points of $X^{(k)}$ (which, of course, is smooth, proper, and of dimension $d$). It is, however, important to know that $\Div_{X/k}^{\eff, (d)}$ indeed satisfies \'etale descent to prove proposition \ref{prop: symmetric_powers_of_curves_parametrise_divisors}, and since this comes from the general fact that the Hilbert functor satisfies \'etale descent, remark \ref{remark: moduli_space_of_effective_divisors} remains necessary. 
                
                We now know that effective divisors of a given degree $d$ are parametrised by some smooth proper variety $\Div_{Y/S}^{\eff, (d)}$ of (relative) dimension $d$, but this is not entirely satisfactory: we would also like to know the identity of this smooth proper variety, and we shall after proposition \ref{prop: symmetric_powers_of_curves_parametrise_divisors}.
                \begin{remark}[Quotient of schemes by finite group schemes] \label{remark: quotients_of_schemes_by_finite_group_schemes}
                    For details on quotients of schemes by finite groups, we refer the reader to \cite[Expos\'e V]{SGA1}. For our purposes, we shall only need to keep in mind the following facts: 
                        \begin{enumerate}
                            \item Let $S$ be a base scheme and $Y$ be an $S$-scheme that is either \textit{(quasi-)projective or (quasi-)affine}, and if additionally. If $G$ be a \textit{finite, flat, and locally of finite presentation} group $S$-scheme acting \textit{freely}\footnote{This is to ensure that the $G$-action induces an \'etale equivalence relation on $Y$, since the $G$-action on $Y$ is free if and only if the corresponding homomorphism of sheaves of groups $G \to \Aut_{S_{\et}}(Y)$ is injective.} on $Y$, then the \href{https://stacks.math.columbia.edu/tag/025X}{\underline{algebraic space}} $Y/G$ is a scheme (cf. \cite[\href{https://stacks.math.columbia.edu/tag/07S7}{Tag 07S7}]{stacks}). 
                            \item If $Y$ is an affine scheme $\Spec A$ then the quotient $Y/G$ will also be affine and will be isomorphic to $\Spec A^G$, thanks to the group-cohomological fact that $H^0(A, G) \cong A^G$. 
                            
                            If $Y$ is (quasi-)projective then $Y/G$ will also be (quasi-)projective.
                        \end{enumerate}
                \end{remark}
                
            \subsubsection{Symmetric powers of curves; the Abel-Jacobi map}
                \begin{convention}[Symmetric group] \label{conv: symmetric_group}
                    If $d$ is any natural number then $\Sigma_d$ shall denote the symmetric group on $d$ elements.
                \end{convention}
                \begin{definition}[Symmetric powers of schemes] \label{def: symmetric_powers_of_schemes}
                    Let $S$ be a base scheme and let $Y$ be an $S$-scheme that is either (quasi-)projective or (quasi-)affine\footnote{In particular, the curve $X$ from convention \ref{conv: automorphic_side_conventions} is projective.}. Then for any $d \geq 1$, the \textbf{$d^{th}$ symmetric power} of $Y$ is the quotient scheme $\Sym^d_S(Y) := Y^d/\underline{\Sigma_d}_{/S}$ of $Y$ by the constant $S$-group scheme\footnote{Cf. \cite[\href{https://stacks.math.columbia.edu/tag/03YW}{Tag 03YW}]{stacks}.} $\underline{\Sigma_d}_{/S}$ with values in the symmetric group $\Sigma_d$ on $d$ elements (note that this group scheme is finite, flat, and locally of finite presentation over $S$, since it is represented by $\coprod_{\sigma \in \Sigma_d} S$); here, $\underline{\Sigma_d}_{/S}$ acts via permutations, which is well-known to be a free action.
                \end{definition}
                \begin{convention}
                    Because the base field $k$ of our curve from convention \ref{conv: automorphic_side_conventions} is fixed, let us write $X^{(d)}$ instead of $\Sym^d_k(X)$ for simplicity.
                \end{convention}
                \begin{lemma}[Smoothness of symmetric powers] \label{lemma: smoothness_of_symmetric_powers}
                    If $Y$ is a smooth variety of dimension $\leq 1$ over some field $C$ then $\Sym_C^d(Y)$ will also be a smooth variety of dimension $\leq 1$.
                \end{lemma}
                    \begin{proof}
                        Smoothness is preserved by base-changing, so we might as well assume that $C$ is algebraically closed. The case $\dim Y = 0$ is trivial, so let us assume that $\dim Y = 1$ (i.e. that $Y$ is a curve); in this case\footnote{We are intentionally confusing the point $y \in |Y|$ and the formal variable $y \in C[\![y]\!]$ because $(y)$ is the unique maximal ideal of $C[\![y]\!]$.}, the formal completion of the stalk $\calO_{Y, y}$ at any point $y \in |Y|$ is necessarily isomorphic to $C[\![y]\!]$, since $\calO_{Y, y}$ shall be a finite-type commutative algebra over an algebraically closed field; as a result, the formal completion of the stalk $\calO_{\Sym^d_C(Y), \vec{y}}$ at any point $\vec{y} \in \Sym^d_C(Y)$ is isomorphic to symmetric formal power series ring $k[\![y_1, ..., y_d]\!]^{\Sigma_d}$, which itself is isomorphic to $k[\![y_1, ..., y_d]\!]$ \textit{a priori} (cf. \cite[Theorem 3.89]{vinberg_2003_algebra}). By \cite[\href{https://stacks.math.columbia.edu/tag/07NX}{Tag 07NX}]{stacks}, a Noetherian local ring $(A, \m)$ is regular if and only if its $\m$-adic completion is regular, and if $A, B$ are finite-type regular commutative algebras over an algebraically closed field $C$ then $A \tensor_C B$ will also be regular as a $C$-algebra, so $\Sym_C^d(Y)$ will be smooth if $C[\![y]\!]$ is regular, which is definitely the case since $\dim C[\![y]\!] = \dim_C (y)/(y)^2 = 1$.
                    \end{proof}
                \begin{proposition}[Symmetric powers of curves parametrise divisors] \label{prop: symmetric_powers_of_curves_parametrise_divisors}
                    For each $d$, the moduli space $\Div_{X/k}^{\eff, (d)}$ is represented by the smooth variety $X^{(d)}$.
                \end{proposition}
                    \begin{proof}
                        Let us denote \textit{unordered} $d$-tuples by $\<x_1, ..., x_d\>$ and also, write $[x]$ for the divisor cut out by any closed point $x \in |X|$. Now, to begin, consider the function $X^{(d)} \to |\Div_{X/k}^{\eff, (d)}|$ given by $\<x_1, ..., x_d\> \mapsto [x_1] + ... + [x_d]$, which if shown to be bijective will demonstrate that $\Div_{X/k}^{\eff, (d)}$ is represented by the smooth variety $X^{(d)}$ via \'etale descent (cf. \cite[\href{https://stacks.math.columbia.edu/tag/024V}{Tag 024V}]{stacks}). $X$ is a geometrically connected smooth curve, so it is geometrically normal (cf. \cite[\href{https://stacks.math.columbia.edu/tag/056T}{Tag 056T}]{stacks}), which in turn implies that the stalk of its structure sheaf over its unique generic point is a normal local domain of Krull dimension $1$, hence a Dedekind domain (cf. \cite[\href{https://stacks.math.columbia.edu/tag/034X}{Tag 034X}]{stacks}). This implies that every divisor on $X$ splits into prime divisors (which correspond to closed points of $X$), and thus the function $\<x_1, ..., x_d\> \mapsto [x_1] + ... + [x_d]$ is surjective. Dedekind domains are special cases of UFDs, so we have also demonstrated that the function $\<x_1, ..., x_d\> \mapsto [x_1] + ... + [x_d]$ is injective.
                    \end{proof}
                \begin{convention}
                    From this point on, we shall write $\Div_{X/k}^{\eff, (d)}$ instead of $X^{(d)}$ whenever we would like to put emphasis on the fact that points of $X^{(d)}$ are effective divisors of degree $d$ on $X$ (such as in \ref{prop: the_unramified_abel_jacobi_map_is_a_projective_bundle}), and \textit{vice versa}, we shall write $X^{(d)}$ when symmetry is of importance, like in theorem \ref{theorem: unramified_abelian_geometric_class_field_theory}.
                \end{convention}
                
                \begin{definition}[The Abel-Jacobi map] \label{def: the_abel_jacobi_map}
                    Let $Y$ be a geometrically connected smooth projective curve over some field $C$. Then, the \textbf{$d^{th}$ Abel-Jacobi map} associated to $Y/C$, denoted by $\AJ_{Y/C}^{(d)}: \Div_{Y/C}^{\eff, (d)} \to \Bun_{\GL_1}^{(d)}(Y)$, is the morphism that maps degree-$d$ effective divisors $D$ to their associated line bundle $\calI_D$.
                \end{definition}
                \begin{remark}[Why the Abel-Jacobi map ?] 
                    Because the function field of the curve $X$ over $\Spec k$ from convention \ref{conv: automorphic_side_conventions} is some (global) field of the form $K \cong k'(t)$, where $k'/k$ is an algebraic extension (cf. proposition \ref{prop: curves_and_function_fields}), and since its ring of integers $\bbO_X \cong k'[t]$ is a Dedekind domain (this is due to Hilbert's Basis Theorem, which tells us that $\dim k'[t] = \dim k + 1 = 0 + 1 = 1$), each of the stalks $\calO_{X, x}$ at closed points $x \in |X|$ must be a discrete valuation ring. This tells us that there is a bijective correspondence between closed points $x \in |X|$ and places of $K$ that are trivial on $k'$. Now, also because $\bbO_X$ is a Dedekind domain, every ideal $\a$ therein factors into primes as, say $\a = \p_1^{e_1} ... \p_n^{e_n}$. Furthemore, $\bbO_X$ is actually a PID, as it is isomorphic to $k'[t]$, which is a UFD thanks to $k'$ itself being a UFD, by virtue of being a field. Thus, closed points $x \in |X|$ are not only in bijection with places of $K$ that are trivial on $k'$, but also effective divisors of degree $1$ on $X$. Now, since the set of rational points of $\Bun_{\GL_1}(X)$ have a close relationship with the class groups of $K$ (see convention \ref{conv: weil_uniformisation}), the Abel-Jacobi map can therefore be seen as a method of linking the data encoded in the primes of the global function field of $X$ and its class groups; this makes it more or less the essence of class field theory.
                \end{remark}
                \begin{proposition}[The Abel-Jacobi map is a projective bundle] \label{prop: the_unramified_abel_jacobi_map_is_a_projective_bundle}
                    \cite[Theorem 25.1(3)]{bhatt_abelian_varieties} Suppose that our curve $X$ from convention \ref{conv: automorphic_side_conventions} is of \href{https://stacks.math.columbia.edu/tag/0BY6}{\underline{genus}} $g \geq 0$. If $d \geq 2g - 1$ then every Abel-Jacobi map $\AJ_{X/k}^{(d)}: \Div_{X/k}^{\eff, (d)} \to \Bun_{\GL_1}^{(d)}(X)$ will be a surjective smooth projective morphism with fibres\footnote{Note that these are precisely the geometric fibres, since $k$ is algebraically closed.} over $k$-rational points isomorphic to $\P^{d - g}_k$.
                \end{proposition}
                    \begin{proof}
                        Combine \cite[Theorem 5.1 and Remark 5.6(c)]{milne_abelian_varieties}.
                    \end{proof}
                \begin{corollary}[Unramified Galois representations induced by the Abel-Jacobi map] \label{coro: unramified_galois_representations_induced_by_the_abel_jacobi_map}
                    Because $\AJ_{X/k}^{(d)}$ is projective and smooth, it is proper (see \cite[\href{https://stacks.math.columbia.edu/tag/01WC}{Tag 01WC}]{stacks}), flat, and of finite presentation (both properties implied by smoothness), and because $\P^{d - g}$ is connected and reduced for all $d \geq g$, there must exist an induced \'etale homotopy sequence as follows by proposition \ref{prop: etale_homotopy_exact_sequence}:
                        $$\pi_1((\P^{d - g}_k)_{\fet}) \to \pi_1((\Div_{X/k}^{\eff, (d)})_{\fet}) \to \pi_1((\Bun_{\GL_1}^{(d)}(X))_{\fet}) \to 1$$
                    Since $\P^{d - g}_k$ is \'etale-simply connected (this is a consequence of $k$ being algebraically closed; cf. example \ref{example: etale_fundamental_group_of_a_curve}), one thus obtains an equivalence between the categories of continuous $\ell$-adic characters of $\pi_1((\Div_{X/k}^{\eff, (d)})_{\fet})$ and of $\pi_1((\Bun_{\GL_1}^{(d)}(X))_{\fet})$ as below, wherein $(\AJ_{X/k}^{(d)})^*$ is the pullback of $\ell$-adic sheaves along the Abel-Jacobi map $\AJ_{X/k}^{(d)}: \Div_{X/k}^{\eff, (d)} \to \Bun_{\GL_1}^{(d)}(X)$:
                        $$(\AJ_{X/k}^{(d)})^*: \Rep^1_{\bar{\Q}_{\ell}}(\pi_1((\Div_{X/k}^{\eff, (d)})_{\fet})) \to \Rep^1_{\bar{\Q}_{\ell}}(\pi_1((\Bun_{\GL_1}^{(d)}(X))_{\fet}))$$
                        $$\chi \mapsto \chi \circ (\AJ_{X/k}^{(d)})^*$$
                \end{corollary}
        
         \subsection{Hecke eigensheaves and geometric class field theory}
            \begin{convention}[Symmetric powers of line bundles] \label{conv: symmetric_powers_of_line_bundles}
                \noindent
                \begin{itemize}
                    \item Write $\sigma^{(d)}: X^d \to X^{(d)}$ for the canonical quotient map and set $\Delta_X^{(d)} := \sigma^{(d)} \circ \Delta_X^d$. Next, for each lisse $\bar{\Q}_{\ell}$-sheaf $\calF \in \Shv_{\bar{\Q}_{\ell}}(X)$, we can construct an lisse $\bar{\Q}_{\ell}$-sheaf $\calF^{(d)} \in \Shv_{\bar{\Q}_{\ell}}(X^{(d)})$ given by $\calF^{(d)} \cong ((\Delta_X^{(d)})_*\calF)^{\Sigma_d}$. 
                    \item In addition, write $\tilde{h}_X^{(d)}: X \x X^{(d)} \to X^{(d + 1)}$ for the map given by $(-[x], D) \mapsto [x] + D$, and likewise, write $\cev{h}_X^{(d)}: X \x \Bun_{\GL_1}^{(d)}(X) \to \Bun_{\GL_1}^{(d + 1)}(X)$ for the map given by $\cev{h}_X^{(d)}(-[x], \E) \cong \E(-[x])$. 
                \end{itemize}
            \end{convention}
            \begin{definition}[Hecke eigensheaves] \label{def: hecke_eigensheaves}
                A \textit{non-zero} lisse $\bar{\Q}_{\ell}$-sheaf $\E \in \Shv_{\bar{\Q}_{\ell}}^{\lisse, 1}(\Bun_{\GL_1}(X))$ is called a \textbf{Hecke eigensheaf} (of rank $1$) if and only if there exists an $\ell$-adic sheaf $\calF \in \Shv_{\bar{\Q}_{\ell}}^{\lisse, 1}(X)$ (called the \textbf{eigenvalue} of $\E$) such that:
                    $$\cev{h}_X^*(\E) \cong \calF \boxtimes \E$$
            \end{definition}
            \begin{remark}
                It is easy to see that Hecke eigensheaves form a full symmetric monoidal subcategory of $\Shv_{\bar{\Q}_{\ell}}^{\lisse, 1}(\Bun_{\GL_1}(X))$, which we shall denote by $\Eig\Shv_{\bar{\Q}_{\ell}}^1(\Bun_{\GL_1}(X))$. In fact, the set of isomorphism classes of Hecke eigensheaves form a subgroup of $\Shv_{\bar{\Q}_{\ell}}^{\lisse, 1}(\Bun_{\GL_1}(X))$ with respect to tensor products. 
            \end{remark}
            
            Lemma \ref{lemma: hecke_eigensheaves_extend_to_lower_degrees}, which is regarding the \say{globality} of Hecke eigensheaves is merely a technicality in service of theorem \ref{theorem: unramified_abelian_geometric_class_field_theory}. The reader can safely skip ahead and refer back to it later.
            \begin{lemma}[Hecke eigensheaves extend to lower degrees] \label{lemma: hecke_eigensheaves_extend_to_lower_degrees}
                Let $\E$ be a Hecke eigensheaf defined on $\bigcup_{d \geq d_0 + 1} \Bun_{\GL_1}^{(d)}(X)$ (for any $d_0 \geq 0$) with eigenvalue $\calF \in \Shv_{\bar{\Q}_{\ell}}^{\lisse, 1}(X)$. Then, $\E$ can be extended uniquely to a Hecke eigensheaf on $\bigcup_{d \geq d_0} \Bun_{\GL_1}^{(d)}(X)$, also with eigenvalue $\calF$.
            \end{lemma}
                \begin{proof}
                    Consider the following commutative diagram, where $\cev{h}_x^{(d)}$ is given by $\calL \mapsto \calL(-[x])$:
                        $$
                            \begin{tikzcd}
                            	{\Spec k \x \Bun_{\GL_1}^{(d)}(X)} & {\Bun_{\GL_1}^{(d)}(X)} \\
                            	{X \x \Bun_{\GL_1}^{(d)}(X)} & {\Bun_{\GL_1}^{(d)}(X)}
                            	\arrow["{\cev{h}_X^{(d)}}", from=2-1, to=2-2]
                            	\arrow["{\cev{h}_x^{(d)}}", from=1-1, to=1-2]
                            	\arrow["{x \x \id}"', from=1-1, to=2-1]
                            	\arrow[Rightarrow, no head, from=1-2, to=2-2]
                            \end{tikzcd}
                        $$
                    By definition, we have an isomorphism $(\cev{h}_x^{(d)})^*\E \cong \calF \boxtimes \E$ for any Hecke eigensheaf $\E$ on $\Bun_{\GL_1}^{(d)}(X)$ with eigenvalue $\calF$, which induces an isomorphism $(\cev{h}_x^{(d)})^*\E \cong x^*\calF \boxtimes \E$ at each geometric point $x \in X$; but since $x \in X$ is a geometric point, $x^*\calF$ is - by definition - nothing but the stalk $\calF_x$, which is isomorphic to $\bar{\Q}_{\ell}$, since $\calF$ is an lisse $\bar{\Q}_{\ell}$-sheaf of rank $1$ on $X$. Next, fix an arbitrary finite set of geometric points $x_1, x_2, ..., x_n \in X$ and consider the following for any Hecke eigensheaf $\E_{\calF^{(d + n)}}$ on $\bigcup_{d \geq 2g - 1} \Bun_{\GL_1}^{(d)}(X)$ that corresponds to $\calF^{(d + n)} \in \Shv_{\bar{\Q}_{\ell}}^{\lisse, 1}(X^{(d + n)})$:
                        $$(\cev{h}_{x_1}^* \circ ... \circ \cev{h}_{x_n}^*)(\E_{\calF^{(d + n)}}) \cong \bigotimes_{i = 1}^n (\calF_{x_i} \boxtimes \E_{\calF^{(d)}})$$
                    which implies that:
                        $$\bigotimes_{i = 1}^n \calF_{x_i}^{\tensor (-1)} \boxtimes (\cev{h}_{x_1}^* \circ ... \circ \cev{h}_{x_n}^*)(\E_{\calF^{(d + n)}}) \cong \E_{\calF^{(d)}}$$
                    If we now take $d := 2g - 1 - n$, we will get the following equation on $\Spec k \x \Bun_{\GL_1}^{2g - 1 - n}(X)$, which yields us a \textit{unique} Hecke eigensheaf $\E_{\calF^{(2g - 1 - n)}}$ on $\bigcup_{d \geq 2g - 1 - n} \Bun_{\GL_1}^{(d)}(X)$ from $\E_{\calF^{(2g - 1)}}$ (which we have already):
                        $$\E_{\calF^{(2g - 1 - n)}} \cong \bigotimes_{i = 1}^n \calF_{x_i}^{\tensor (-1)} \boxtimes (\cev{h}_{x_1}^* \circ ... \circ \cev{h}_{x_n}^*)(\E_{\calF^{(2g - 1)}})$$
                \end{proof}
            
            \begin{theorem}[Unramified abelian geometric class field theory] \label{theorem: unramified_abelian_geometric_class_field_theory}
                There exists a canonical monoidal equivalence between the groupoid of rank-$1$ lisse $\bar{\Q}_{\ell}$-sheaves on $X$ and the groupoid of ($\ell$-adic) Hecke eigensheaves of rank $1$ on $\Bun_{\GL_1}(X)$:
                    $$\Autom_X: \Shv_{\bar{\Q}_{\ell}}^{\lisse, 1}(X) \to \Eig\Shv_{\bar{\Q}_{\ell}}^1(\Bun_{\GL_1}(X))$$
                which maps each lisse $\bar{\Q}_{\ell}$-sheaf $\calF \in \Shv_{\bar{\Q}_{\ell}}^{\lisse, 1}(X)$ to a Hecke eigensheaf:
                    $$\Autom_X(\calF) \in \Eig\Shv_{\bar{\Q}_{\ell}}^1(\Bun_{\GL_1}(X))$$
                with eigenvalue $\calF$.
            \end{theorem}
                \begin{proof}
                    Our strategy for this proof is to explicitly construct - for each $\calF \in \Shv_{\bar{\Q}_{\ell}}^{\lisse, 1}(X)$ - the corresponding Hecke eigensheaf $\Autom_X(\calF)$, and this will involve three steps:
                        \begin{enumerate}
                            \item \textbf{(A \say{Hecke eigensheaf} on $X^{(d)}$):} In this step we shall construct a lisse $\bar{\Q}_{\ell}$-sheaf on $X^{(d)}$ (hence on $\bigcup_{d \geq 2g - 1} X^{(d)}$; cf. proposition \ref{prop: the_unramified_abel_jacobi_map_is_a_projective_bundle}) satisfying an analogue of the Hecke eigensheaf property (cf. definition \ref{def: hecke_eigensheaves}) with the purpose in mind being that by pushing this lisse $\bar{\Q}_{\ell}$-adic sheaf forward using the Abel-Jacobi map, one shall obtain a legitimate Hecke eigensheaf on $\Bun_{\GL_1}^{(d)}(X)$ (hence on $\bigcup_{d \geq 2g - 1} \Bun_{\GL_1}^{(d)}(X)$).
                            
                            The first observation that one can make is that there is an isomorphism $(\sigma^{(d)})^*\calF^{(d)} \cong \calF^{\boxtimes d}$ of lisse $\bar{\Q}_{\ell}$-sheaves on $X^d$. Next, notice how there are commutative diagrams of the following form, wherein the maps $\tilde{h}_X^{(d)}$ are given by $(-[x], D) \mapsto [x] + D$ (cf. remark \ref{remark: adding_effective_divisors} and convention \ref{conv: symmetric_powers_of_line_bundles}):
                                $$
                                    \begin{tikzcd}
                                    	{X \x X^d} & {X^{(d + 1)}} \\
                                    	{X \x X^{(d)}}
                                    	\arrow["{\id_X \x \sigma^{(d)}}"', from=1-1, to=2-1]
                                    	\arrow["{\tilde{h}_X^{(d)}}"', dashed, from=2-1, to=1-2]
                                    	\arrow["{\sigma^{(d + 1)}}", from=1-1, to=1-2]
                                    \end{tikzcd}
                                $$
                            We thus have, as follows, a $\Sigma_d$-equivariant analogue on $X \x X^{(d)}$ of the Hecke eigensheaf property for all $\calF \in \Shv_{\bar{\Q}_{\ell}}^{\lisse, 1}(X)$:
                                $$(\tilde{h}_X^{(d)})^* \calF^{(d + 1)} \cong \calF \boxtimes \calF^{(d)}$$
                            \item \textbf{(A Hecke eigensheaf on $\Bun_{\GL_1}^{(d)}(X)$):} Recall from corollary \ref{coro: unramified_galois_representations_induced_by_the_abel_jacobi_map} that for every $d \geq 2g - 1$, there exists the following adjoint equivalence:
                                $$
                                    \begin{tikzcd}
                                    	{\Shv_{\bar{\Q}_{\ell}}^{\lisse, 1}(X^{(d)})} & {\Shv_{\bar{\Q}_{\ell}}^{\lisse, 1}(\Bun_{\GL_1}^{(d)}(X))}
                                    	\arrow[""{name=0, anchor=center, inner sep=0}, "{(\AJ_{X/k}^{(d)})_*}"', bend right, from=1-1, to=1-2]
                                    	\arrow[""{name=1, anchor=center, inner sep=0}, "{(\AJ_{X/k}^{(d)})^*}"', bend right, from=1-2, to=1-1]
                                    	\arrow["\dashv"{anchor=center, rotate=-90}, draw=none, from=1, to=0]
                                    \end{tikzcd}
                                $$ 
                            From this, one infers that every lisse $\bar{\Q}_{\ell}$-sheaf $\calF \in \Shv_{\bar{\Q}_{\ell}}^{\lisse, 1}(X^{(d)})$ has the form $(\AJ_{X/k}^{(d)})^* \E$ for some \textit{unique} lisse $\bar{\Q}_{\ell}$-sheaf $\E \in \Shv_{\bar{\Q}_{\ell}}^{\lisse, 1}(\Bun_{\GL_1}^{(d)}(X))$, meaning that there exists lisse $\bar{\Q}_{\ell}$-sheaves $\E_{\calF^{(d)}}, \E_{\calF^{(d + 1)}} \in \Shv_{\bar{\Q}_{\ell}}^{\lisse, 1}(\Bun_{\GL_1}^{(d)}(X))$ corresponding to $\calF^{(d)}, \calF^{(d + 1)} \in \Shv_{\bar{\Q}_{\ell}}^{\lisse, 1}(X^{(d)})$ respectively (and ultimately, to $\calF \in \Shv_{\bar{\Q}_{\ell}}^{\lisse, 1}(X)$) satisfying the following equation in $\Shv_{\bar{\Q}_{\ell}}^{\lisse, 1}(X \x X^{(d)})$:
                                $$(\tilde{h}_X^{(d)})^* (\AJ_{X/k}^{(d + 1)})^* \E_{\calF^{(d + 1)}} \cong \calF \boxtimes (\AJ_{X/k}^{(d)})^*\E_{\calF^{(d)}}$$
                            Next, consider the following commutative diagram:
                                $$
                                    \begin{tikzcd}
                                    	{X \x X^{(d)}} & {X^{(d + 1)}} \\
                                    	{X \x \Bun_{\GL_1}^{(d)}(X)} & {\Bun_{\GL_1}^{(d + 1)}(X)}
                                    	\arrow["{\cev{h}_X^{(d)}}", from=2-1, to=2-2]
                                    	\arrow["{\id_X \x \AJ_{X/k}^{(d)}}"', from=1-1, to=2-1]
                                    	\arrow["{\AJ^{(d + 1)}}", from=1-2, to=2-2]
                                    	\arrow["{\tilde{h}_X^{(d)}}", from=1-1, to=1-2]
                                    \end{tikzcd}
                                $$
                            which induces the following equations in $\Shv_{\bar{\Q}_{\ell}}^{\lisse, 1}(X \x X^{(d)})$ for all $\calF \in \Shv_{\bar{\Q}_{\ell}}^{\lisse, 1}(X)$:
                                $$
                                    \begin{aligned}
                                        (\id_X \x \AJ_{X/k}^{(d)})^* (\cev{h}_X^{(d)})^* \E_{\calF^{(d + 1)}} & \cong (\tilde{h}_X^{(d)})^* (\AJ^{(d + 1)})^* \E_{\calF^{(d + 1)}} \cong \calF \boxtimes (\AJ_{X/k}^{(d)})^*\E_{\calF^{(d)}}
                                        \\
                                        & \cong (\id_X \x \AJ_{X/k}^{(d)})^*(\calF \boxtimes \E_{\calF^{(d)}})
                                    \end{aligned}
                                $$
                            and since $(\id_X \x \AJ_{X/k}^{(d)})^*$ is an invertible functor (cf. corollary \ref{coro: unramified_galois_representations_induced_by_the_abel_jacobi_map}), we have, furthermore, the following equation in $\Shv_{\bar{\Q}_{\ell}}^{\lisse, 1}(X \x \Bun_{\GL_1}^{(d)}(X))$, which is precisely the Hecke eigensheaf property from definition \ref{def: hecke_eigensheaves}:
                                $$(\cev{h}_X^{(d)})^* \E_{\calF^{(d + 1)}} \cong \calF \boxtimes \E_{\calF^{(d)}}$$
                            We have thus obtained a \textit{unique} Hecke eigensheaf on $\bigcup_{d \geq 2g - 1} \Bun_{\GL_1}^{(d)}(X)$ from an arbitrary lisse $\bar{\Q}_{\ell}$-sheaf $\calF \in \Shv_{\bar{\Q}_{\ell}}^{\lisse, 1}(X)$.
                            \item \textbf{(A Hecke eigensheaf on $\Bun_{\GL_1}(X)$):} Finally, apply \ref{lemma: hecke_eigensheaves_extend_to_lower_degrees} to extend the Hecke eigensheaf that we have constructed on $\bigcup_{d \geq 2g - 1} \Bun_{\GL_1}^{(d)}(X)$ to $\Bun_{\GL_1}(X)$ (i.e. to degrees $d < 2g - 1$)
                        \end{enumerate}
                    By putting everything above anad lemma \ref{lemma: hecke_eigensheaves_extend_to_lower_degrees} together, one sees that the sought-for functor:
                        $$\Autom_X: \Shv_{\bar{\Q}_{\ell}}^{\lisse, 1}(X) \to \Eig\Shv_{\bar{\Q}_{\ell}}^1(\Bun_{\GL_1}(X))$$
                    can be constructed to be given by the following formula for all $\calF \in \Shv_{\bar{\Q}_{\ell}}^{\lisse, 1}(X)$ and all $d$ sufficiently large (e.g. $d \geq 2g - 1$):
                        $$\Autom_X(\calF) \cong \bigotimes_{i = 1}^{d - 1} \calF_{x_i}^{\tensor (-1)} \boxtimes (\cev{h}_{x_1}^* \circ ... \circ \cev{h}_{x_n}^*) (\cev{h}_X^{(1)})^* (\AJ_{X/k}^{(1)})_* \calF^{(d)}$$
                    where $x_1, ..., x_{d - 1} \in X$ is an arbitrary set of geometric points of $X$. One therefore also sees that $\Autom_X$ is monoidal by construction.
                \end{proof}
            
            \begin{corollary}[Geometric Langlands for $\GL_1$] \label{coro: geometric_langlands_for_GL1}
                By putting theorem \ref{theorem: galois_representations_are_lisse_sheaves} and theorem \ref{theorem: unramified_abelian_geometric_class_field_theory} together, one gets a monoidal equivalence $\Rep^1_{\bar{\Q}_{\ell}}(\pi_1(X_{\fet})) \cong \Eig\Shv_{\bar{\Q}_{\ell}}^1(\Bun_{\GL_1}(X))$.
            \end{corollary}
            
        \subsection{Artin Reciprocity via traces of Frobenii}
            As a final step, let us decategorify the left-hand side of the monoidal equivalence $\Rep^1_{\bar{\Q}_{\ell}}(\pi_1(X_{\fet})) \cong \Eig\Shv_{\bar{\Q}_{\ell}}^1(\Bun_{\GL_1}(X))$ from corollary \ref{coro: geometric_langlands_for_GL1} to obtain a correspondence between continuous $\ell$-adic characters of $\pi_1(X_{\fet})$ and so-called Hecke characters, using Grothendieck's Sheaf-Function Correspondence. Doing so will yield us the usual version of Artin Reciprocity for global function fields over perfect fields of positive characteristics, in terms of towers of finite abelian extensions and so-called Hecke characters (see proposition \ref{prop: hecke_characters_from_hecke_eigensheaves} and theorem \ref{theorem: artin_reciprocity_for_function_fields_over_finite_fields}), as well as Class Field Theory for global function fields over finite fields (cf. theorem \ref{theorem: class_field_theory_for_global_function_fields_over_finite_fields}).
            
            \subsubsection{Grothendieck's Sheaf-Function Correspondence via character sheaves}
                We begin by introducing Grothendieck's Sheaf-Function Corresondence\footnote{En Français: \say{\textit{La Correspondance Faisceaux-Fonctions de Grothendieck}}.}, which allows us to formally realise the notion of Hecke eigensheaves from definition \ref{def: hecke_eigensheaves} as a categorification of the classical notion of Hecke characters (cf. proposition \ref{prop: hecke_characters_from_hecke_eigensheaves}). For this, let us first discuss traces of Frobenius endomorphisms on $\ell$-adic sheaves, which unfortunately only makes sense over positive characteristics.
                \begin{convention} \label{conv: frobenii}
                    From now on, the base field $k$ from convention \ref{conv: automorphic_side_conventions} shall be some finite field $\F_q$ (which we note to be perfect). This gives us access to the (absolute) Frobenius on $X_S$, with $S$ being some perfect scheme over $\Spec \F_q$, which we denote by $\Frob_{X_S}$.
                \end{convention}
                
                \begin{definition}[Traces of Frobenii] \label{def: traces_of_frobenii}
                    The Frobenius on any Noetherian $\F_q$-scheme induces a functor $\Frob_Z^*: \Shv_{\bar{\Q}_{\ell}}^{\lisse, 1}(Z) \to \Shv_{\bar{\Q}_{\ell}}^{\lisse, 1}(Z)$, and at the level of stalks at geometric points $\bar{z} \in Z$, said functor gives rise to endomorphisms $\Frob_{\bar{z}}: \calF_{\bar{z}} \to \calF_{\bar{z}}$. The trace of any such endomorphism is called the \textbf{trace of Frobenius} on $\calF$ at $\bar{z}$.
                \end{definition}
                \begin{remark}[Basic properties of traces of Frobenii] \label{remark: basic_properties_of_traces_of_frobenii}
                    The following properties are trivial consequences of definition \ref{def: traces_of_frobenii}:
                        \begin{itemize}
                            \item The first thing that one should note is that in taking traces of Frobenii with respect to a fixed lisse $\bar{\Q}_{\ell}$-sheaf $\calF \in \Shv_{\bar{\Q}_{\ell}}^{\lisse, 1}(Z)$, one obtains a function $\frob^{\calF}: Z \to \bar{\Q}_{\ell}$.
                            \item For any pair of rank-$1$ lisse $\bar{\Q}_{\ell}$-sheaves $\calF, \calF' \in \Shv_{\bar{\Q}_{\ell}}^{\lisse, 1}(Z)$ and any fixed geometric point $\bar{z} \in Z$, one has $\frob^{\calF \tensor \calF'} = \frob^{\calF} \frob^{\calF'}$.
                            \item Let $\theta: Y \to Z$ be a morphism between $\F_q$-schemes that are locally of finite type and let $\calF$ be an lisse $\bar{\Q}_{\ell}$-sheaf of rank $1$ on $Z$. Then $f^{\theta^*\calF} = f^{\calF} \circ \theta_{\bar{\F}_q}$\footnote{To prove this, simply recall also the basic sheaf-theoretic fact that for any geometric point $\bar{y} \in Y$ over a fixed geometric point $\bar{z} \in Z(\bar{F}_q)$ (i.e. such that $\theta_{\bar{\F}_q}(\bar{y}) = \bar{z}$), one has an isomorphism $(\theta^*\calF)_{\theta_{\bar{\F}_q}(\bar{y})} \cong \calF_{\bar{z}}$ of stalks.}.
                        \end{itemize}
                \end{remark}
                
                In order to obtain continuous $\ell$-adic characters via taking traces of Frobenii on lisse $\bar{\Q}_{\ell}$-sheaves $\calF$, it is crucial that we demonstrate how for a suitable group scheme $G$, the function $\frob^{\calF}$ as in definition \ref{def: traces_of_frobenii} gives rise to a continuous group homomorphism $\xi^{\calF}: G(\F_q) \to \bar{\Q}_{\ell}^{\x}$. For this, it will be convenient to have the notion of \textbf{character sheaves} at our disposal, for which a good reference is \cite{cunningham_roe_function_sheaf_dictionary_quasi_characters_p_adic_tori} (whose presentation we shall also follow closely). 
                
                We begin this discussion by introducing the so-called \say{Weil sheaf condition}, necessary because eventually, we will want to show that the categories of character sheaves on $\Bun_{\GL_1}(X)$ and that of Hecke eigensheaves are equivalent (cf. lemma \ref{lemma: hecke_eigensheaves_are_character_sheaves}).
                \begin{definition}[Weil sheaves] \label{def: weil_sheaves}
                    Let $k$ be a perfect field of characteristic $p > 0$. The category of $\ell$-adic \textbf{Weil sheaves} of rank $1$ on a Noetherian $k$-schemes is the full subcategory of $\Shv_{\bar{\Q}_{\ell}}^{\lisse, 1}(Z)$ on which the pullback functor $\Frob_{Z}^*: \Shv_{\bar{\Q}_{\ell}}^{\lisse, 1}(Z) \to \Shv_{\bar{\Q}_{\ell}}^{\lisse, 1}(Z)$ is an equivalence\footnote{In other words, we can think of Weil sheaves as Frobenius-fixed lisse $\bar{\Q}_{\ell}$-sheaves.}. We suggestively denote this category by $\Shv_{\bar{\Q}_{\ell}}^{\lisse, 1}(Z)^{\Frob}$.
                \end{definition}
                \begin{remark}[Basic properties of Weil sheaves] \label{remark: properties_of_weil_sheaves}
                    Let $k$ be a perfect field of characteristic $p > 0$ and let $Z$ be a Noetherian $k$-scheme. 
                    \begin{itemize}
                        \item First of all, one sees - via the fact that $\Frob_Z^*$ commutes with tensor products of $\ell$-adic sheaves on $Z$ - that the category $\Shv_{\bar{\Q}_{\ell}}^{\lisse, 1}(Z)^{\Frob}$ of $\ell$-adic Weil sheaves over $Z$ is symmetric monoidal (cf. \cite[Definition 8.1.12]{EGNO}) with respect to the usual tensor product of $\ell$-adic sheaves.
                        \item \cite[Proposition 5.20]{tendler_2015_geometric_class_field_theory} Let $\theta: Y \to Z$ be a morphism between $\F_q$-schemes that are locally of finite type and let $\calF$ be a Weil sheaf on $Z$. Then $\theta^*\calF$ shall, in turn, be a Weil sheaf on $Y$. Since $\Shv_{\bar{\Q}_{\ell}}^{\lisse, 1}(Z)^{\Frob}$ is a full subcategory of $\Shv_{\bar{\Q}_{\ell}}^{\lisse, 1}(Z)$ - which is compatible with $*$-pullbacks - one then sees that $\Shv_{\bar{\Q}_{\ell}}^{\lisse, 1}(Z)^{\Frob}$ is also compatible with $*$-pullbacks.
                    \end{itemize}
                \end{remark}
                
                \begin{definition}[Character sheaves] \label{def: character_sheaves}
                    Let $k$ be a perfect field of some prime characteristic $p$ and $G$ be a Noetherian\footnote{In \cite{cunningham_roe_function_sheaf_dictionary_quasi_characters_p_adic_tori}, the more specialised case of smooth group schemes over $\Spec \F_q$ is considered.} commutative group scheme over $\Spec k$, whose group structure is given by $\mu_G: G \x G \to G$. In such a situation, an $\ell$-adic \textbf{character sheaf} of $G$ shall be an Weil sheaf $\E$ of rank $1$ over $G$, such that $\mu_G^*\E \cong \E \boxtimes \E$.
                \end{definition}
                \begin{remark}[Rigid symmetric monoidal categories of character sheaves] \label{remark: rigid_monoidal_categories_of_character_sheaves}
                    \cite[Subsection 1.2]{cunningham_roe_function_sheaf_dictionary_quasi_characters_p_adic_tori} It can be easily checked, through verifying the relevant axioms (see \cite[Definition 2.10.11]{EGNO}), that $\ell$-adic character sheaves on a given Noetherian commutative group scheme $G$ over $\Spec k$ (for some field $k$) form a rigid symmetric monoidal subcategory of $\Shv_{\bar{\Q}_{\ell}}^{\lisse, 1}(G)$, which we denote by $\Char\Shv_{\bar{\Q}_{\ell}}(G)$. Interestingly, if $G$ is a \textit{smooth} commutative group scheme over $\Spec \F_q$ then $\Char\Shv_{\bar{\Q}_{\ell}}(G)$ will actually be a groupoid, a phenomenon that is in perfect analogy with Schur's Lemma (cf. \cite[Lemma 3.6]{lam_first_course_in_noncommutative_rings}) and coincidentally, can be shown via a straightforward application of Schur's Lemma to stalks of character sheaves (cf. \cite[Lemma 1.3]{cunningham_roe_function_sheaf_dictionary_quasi_characters_p_adic_tori}); as a consequence, the isomorphism classes of character sheaves of a smooth commutative group scheme over $\Spec \F_q$ form an abelian group. 
                \end{remark}
                
                It turns out that Hecke eigensheaves on $\Bun_{\GL_1}(X)$ are the same as character sheaves, which is rather fortunate for us, as this directly implies that Hecke eigensheaves correspond to certain continuous $\ell$-adic characters of $\Bun_{\GL_1}(X)(\F_q)$ via lemma \ref{lemma: sheaf_function_correspondence_for_connected_algebraic_groups}. Nevertheless, there remains a technical difficulty that we need to overcome, that being the fact that $\Bun_{\GL_1}(X)$ is not connected (cf. remark \ref{remark: geometry_of_the_picard_stack}).
                \begin{remark}[The connected-smooth-\'etale short exact sequence] \label{remark: the_connected_smooth_etale_short_exact_sequence}
                    For theorem \ref{theorem: sheaf_function_correspondence_for_smooth_groups}, recall that for any scheme $Y$ that is locally of finite type over a field $k$, there exists a maximal $k$-subalgebra $O(Y)$ of $\Gamma(Y, \calO_Y)$ which is \'etale over $k$ (cf. \cite[Proposition 5.44]{milne_algebraic_groups}). We can then define the scheme $\pi_0(Y)$ of \say{connected components} of $Y$ to be $\Spm O(Y)$ and in doing so, we will have obtained a morphism $Y \to \pi_0(Y)$ that is universal among all \'etale $\F_q$-schemes. 
                    
                    Let $G$ be a smooth commutative group scheme over $\Spec k$, let $G^0$ be the connected component of the identity thereof (known to be a connected commutative algebraic group over $\Spec k$; cf. \cite[Lemma 3.2]{cunningham_roe_function_sheaf_dictionary_quasi_characters_p_adic_tori}). It is known that the universal morphism $G \to \pi_0(G)$ induces the following short exact sequence of commutative algebraic groups (cf. \cite[Proposition 5.48]{milne_algebraic_groups}), known as the \textbf{connected-smooth-\'etale short exact sequence}:
                        $$
                            \begin{tikzcd}
                            	0 & {G^0} & G & {\pi_0(G)} & 0
                            	\arrow[from=1-1, to=1-2]
                            	\arrow["{\iota_{G^0}}", from=1-2, to=1-3]
                            	\arrow["{\pi_{G^0}}", from=1-3, to=1-4]
                            	\arrow[from=1-4, to=1-5]
                            \end{tikzcd}
                        $$
                    This short exact sequence can be shown (see \cite[Proposition 3.3, Lemma 3.4, and Proposition 3.5]{cunningham_roe_function_sheaf_dictionary_quasi_characters_p_adic_tori}) to induce another short exact sequence of abelian groups as follows:
                        $$
                            \begin{tikzcd}
                            	0 & {\Char\Shv_{\bar{\Q}_{\ell}}(\pi_0(G))} & {\Char\Shv_{\bar{\Q}_{\ell}}(G)} & {\Char\Shv_{\bar{\Q}_{\ell}}(G^0)} & 0
                            	\arrow[from=1-1, to=1-2]
                            	\arrow["{\pi_{G^0}^*}", from=1-2, to=1-3]
                            	\arrow["{\iota_{G^0}^*}", from=1-3, to=1-4]
                            	\arrow[from=1-4, to=1-5]
                            \end{tikzcd}
                        $$
                    One thus infers that any investigation of character sheaves on smooth commutative group schemes can be broken down into analyses of character sheaves on connected commutative group schemes of finite type and of those on \'etale commutative group schemes (see lemmas \ref{lemma: sheaf_function_correspondence_for_connected_algebraic_groups} and \ref{lemma: sheaf_function_correspondence_for_etale_commutative_group_schemes} respectively).
                \end{remark}
                
                \begin{lemma}[Grothendieck's Sheaf-Function Correspondence for connected algebraic groups] \label{lemma: sheaf_function_correspondence_for_connected_algebraic_groups}
                    \cite[Proposition 1.14]{cunningham_roe_function_sheaf_dictionary_quasi_characters_p_adic_tori} If $H$ is a connected commutative algebraic group over $\Spec \F_q$ then taking traces of Frobenii:
                        $$\trace(\Frob_H^* \mid -): \Char\Shv_{\bar{\Q}_{\ell}}(H) \to \Rep_{\bar{\Q}_{\ell}}^1(H(\F_q))$$
                    will be a group isomorphism.
                \end{lemma}
                \begin{lemma}[Grothendieck's Sheaf-Function Correspondence for \'etale commutative group schemes] \label{lemma: sheaf_function_correspondence_for_etale_commutative_group_schemes}
                    Suppose that $Q$ is an \'etale commutative algebraic group over $\Spec \F_q$.
                    \begin{itemize}
                        \item \cite[Proposition 2.7]{cunningham_roe_function_sheaf_dictionary_quasi_characters_p_adic_tori} There exists a group isomorphism:
                            $$\ker \trace(\Frob_Q^* \mid -) \cong H^0(\rmW_{\F_q}, H^2(Q_{\bar{\F}_q}, \bar{\Q}_{\ell}^{\x}))$$
                        where $\rmW_{\F_q}$ denotes the cyclic subgroup of $\Gal(\bar{\F}_q/\F_q)$ generated by $\Frob_{\F_q}$. 
                        \item \cite[Remark 2.9]{cunningham_roe_function_sheaf_dictionary_quasi_characters_p_adic_tori} $H^2(Q_{\bar{\F}_q}, \bar{\Q}_{\ell}^{\x})$ is trivial if and only if $Q_{\bar{\F}_q}$ is cyclic.
                    \end{itemize}
                \end{lemma}
                \begin{corollary}[Kernels of traces of Frobenii on \'etale group schemes] \label{coro: kernels_of_traces_of_frobenii_on_etale_group_schemes}
                    By a general fact from the theory of group cohomology, we know that $H^0(\rmW_{\F_q}, H^2(Q_{\bar{\F}_q}, \bar{\Q}_{\ell}^{\x})) \cong H^2(Q_{\bar{\F}_q}, \bar{\Q}_{\ell}^{\x})^{\rmW_{\F_q}}$. We also know that $H^2(Q_{\bar{\F}_q}, \bar{\Q}_{\ell}^{\x})$ is trivial if and only if $Q_{\bar{\F}_q}$ is cyclic. By putting the two facts together, we deduce that $H^2(Q_{\bar{\F}_q}, \bar{\Q}_{\ell}^{\x})^{\rmW_{\F_q}}$ will be trivial if $Q_{\bar{\F_q}}$ is cyclic. 
                \end{corollary}
                
                \begin{remark}[The category of locally compact Hausdorff Hausdorff abelian groups] \label{remark: the_category_of_locally_compact_hausdorff_abelian_groups}
                    A fine technicality that one shall need to keep in mind for theorem \ref{theorem: sheaf_function_correspondence_for_smooth_groups} is that the category $\Loc\Comp\Ab$ of (small) locally compact Hausdorff Hausdorff abelian groups is quasi-abelian (cf. \cite[Proposition 1.2]{hoffmann_2007_homological_algebra_of_topological_groups}). Among other things, this tells us that if $D$ a divisible locally compact Hausdorff abelian group (cf. \cite[Definition 3.1]{hoffmann_2007_homological_algebra_of_topological_groups}) then the functor $\Loc\Comp\Ab(-, D): \Loc\Comp\Ab^{\op} \to \Ab$ will map open embeddings in $\Loc\Comp\Ab$ to surjective homomorphisms in $\Ab$; in particular, this means that $\Loc\Comp\Ab(-, D)$ preserves all monomorphisms between discrete abelian groups (which are trivially compact and Hausdorff) and is thus exact.
                \end{remark}
                \begin{theorem}[Grothendieck's Sheaf-Function Correspondence for smooth groups] \label{theorem: sheaf_function_correspondence_for_smooth_groups}
                    \cite[Theorem 3.6]{cunningham_roe_function_sheaf_dictionary_quasi_characters_p_adic_tori} If $G$ is a smooth commutative group scheme over $\Spec \F_q$ then $\trace(\Frob_G^* \mid -)$ as in lemma \ref{lemma: sheaf_function_correspondence_for_connected_algebraic_groups} will be a surjective group homomorphism with kernel $H^2(\pi_0(G_{\bar{\F}_q}), \bar{\Q}_{\ell}^{\x})^{\rmW_{\F_q}}$, with $\rmW_{\F_q}$ denoting the subgroup of $\Gal(\bar{\F}_q/\F_q)$ that is generated by $\Frob_{\F_q}$.
                \end{theorem}
                    \begin{proof}
                        Because the locally compact Hausdorff abelian group $\bar{\Q}_{\ell}^{\x}$ is divisible by virtue of being the group of units of a field, the functor\footnote{Which is tautologically equal to $\Rep^1_{\bar{\Q}_{\ell}}(-)$.} $\Loc\Comp\Ab(-, \bar{\Q}_{\ell}^{\x}): \Loc\Comp\Ab^{\op} \to \Ab$ is exact \textit{a priori} (cf. remark \ref{remark: the_category_of_locally_compact_hausdorff_abelian_groups}). Through taking traces of Frobenii, we then obtain the following commutative diagram in $\Ab$, wherein the rows are short exact sequences:
                            $$
                                \begin{tikzcd}
                                	0 & {\Char\Shv_{\bar{\Q}_{\ell}}(\pi_0(G))} & {\Char\Shv_{\bar{\Q}_{\ell}}(G)} & {\Char\Shv_{\bar{\Q}_{\ell}}(G^0)} & 0 \\
                                	0 & {\Rep^1_{\bar{\Q}_{\ell}}(\pi_0(G)(\F_q))} & {\Rep^1_{\bar{\Q}_{\ell}}(G(\F_q))} & {\Rep^1_{\bar{\Q}_{\ell}}(G^0(\F_q))} & 0
                                	\arrow[from=1-1, to=1-2]
                                	\arrow["{\pi_{G^0}^*}", from=1-2, to=1-3]
                                	\arrow["{\iota_{G^0}^*}", from=1-3, to=1-4]
                                	\arrow[from=1-4, to=1-5]
                                	\arrow[from=2-1, to=2-2]
                                	\arrow[from=2-2, to=2-3]
                                	\arrow[from=2-3, to=2-4]
                                	\arrow[from=2-4, to=2-5]
                                	\arrow["{\trace(\Frob_{\pi_0(G)}^* \mid -)}", from=1-2, to=2-2]
                                	\arrow["{\trace(\Frob_G^* \mid -)}", from=1-3, to=2-3]
                                	\arrow["{\trace(\Frob_{G^0}^* \mid -)}", from=1-4, to=2-4]
                                \end{tikzcd}
                            $$
                        Now, because $G^0$ is a connected commutative algebraic group, we can apply lemma \ref{lemma: sheaf_function_correspondence_for_connected_algebraic_groups} to see that $\trace(\Frob_{G^0}^* \mid -)$ must be an isomorphism; at the same time, $\trace(\Frob_{\pi_0(G)^* \mid -})$ is surjective due to $\pi_0(G)$ being \'etale as a commutative algebraic group over $\Spec \F_q$ (cf. lemma \ref{lemma: sheaf_function_correspondence_for_etale_commutative_group_schemes}). An application of the Snake Lemma (cf. \cite[\href{https://stacks.math.columbia.edu/tag/07JV}{Tag 07JV}]{stacks}) then yields us the following long exact sequence in $\Ab$ (the sequence with the map $\delta$), from which one clearly sees that the cokernel of $\trace(\Frob_G^* \mid -)$ is trivial and the map is therefore surjective as claimed:
                            $$
                                \begin{tikzcd}
                                	0 & {\ker \trace(\Frob^*_{\pi_0(G)} \mid -)} & {\ker \trace(\Frob^*_G \mid -)} & 0 \\
                                	& {\Char\Shv_{\bar{\Q}_{\ell}}(\pi_0(G))} & {\Char\Shv_{\bar{\Q}_{\ell}}(G)} & {\Char\Shv_{\bar{\Q}_{\ell}}(G^0)} \\
                                	& {\Rep^1_{\bar{\Q}_{\ell}}(\pi_0(G))} & {\Rep^1_{\bar{\Q}_{\ell}}(G)} & {\Rep^1_{\bar{\Q}_{\ell}}(G^0)} \\
                                	& 0 & {\coker \trace(\Frob^*_G \mid -)} & 0 & 0
                                	\arrow["{\trace(\Frob^*_{\pi_0(G)} \mid -)}", two heads, from=2-2, to=3-2]
                                	\arrow["{\trace(\Frob^*_G \mid -)}", from=2-3, to=3-3]
                                	\arrow["{\trace(\Frob^*_{G_0} \mid -)}", tail, two heads, from=2-4, to=3-4]
                                	\arrow["{\pi_{G_0}^*}", from=2-2, to=2-3]
                                	\arrow["{\iota_{G_0}^*}", from=2-3, to=2-4]
                                	\arrow[from=3-2, to=3-3]
                                	\arrow[from=3-3, to=3-4]
                                	\arrow[from=1-1, to=1-2]
                                	\arrow[from=1-2, to=1-3]
                                	\arrow[from=1-3, to=1-4]
                                	\arrow["\delta"', dashed, from=1-4, to=4-2]
                                	\arrow[from=4-2, to=4-3]
                                	\arrow[from=4-3, to=4-4]
                                	\arrow[from=4-4, to=4-5]
                                	\arrow[from=1-4, to=2-4]
                                	\arrow[from=3-4, to=4-4]
                                	\arrow[from=3-2, to=4-2]
                                	\arrow[from=1-2, to=2-2]
                                	\arrow[from=1-3, to=2-3]
                                	\arrow[from=3-3, to=4-3]
                                \end{tikzcd}
                            $$
                        Lastly, in order to compute the kernel, simply apply corollary \ref{coro: kernels_of_traces_of_frobenii_on_etale_group_schemes}.
                    \end{proof}
            
            \subsubsection{Artin Reciprocity and class field theory for global function fields over finite fields}
                \begin{convention}[Global function field of $X$] \label{conv: global_function_field}
                    Let us write $K_X$ for the global function field of $X$ (i.e the stalk of the structure sheaf of $X$ at the unique generic point), $\bbO_X$ for its ring of integers, and $\A_X$ for the corresponding ring of ad\`eles. For more details on these constructions, we refer the reader to \cite[Section VI.1]{neukirch_2010_algebraic_number_theory}. 
                \end{convention}
                
                \begin{lemma}[Hecke eigensheaves are Weil sheaves] \label{lemma: hecke_eigensheaves_are_weil_sheaves}
                    Any Hecke eigensheaf on $\Bun_{\GL_1}(X)$ is a Weil sheaf.
                \end{lemma}
                    \begin{proof}
                        Fix an arbitrary Hecke eigensheaf $\E_{\calF}$ with eigenvalue $\calF$ and consider the following commutative diagram:
                            $$
                                \begin{tikzcd}
                                	{X \x \Bun_{\GL_1}(X)} & {\Bun_{\GL_1}(X)} \\
                                	{X \x \Bun_{\GL_1}(X)} & {\Bun_{\GL_1}(X)}
                                	\arrow["{\id_X \x \Frob_{\Bun_{\GL_1}(X)}}"', from=1-1, to=2-1]
                                	\arrow["{\cev{h}_X}", from=2-1, to=2-2]
                                	\arrow["{\cev{h}_X}", from=1-1, to=1-2]
                                	\arrow["{\Frob_{\Bun_{\GL_1}(X)}}", from=1-2, to=2-2]
                                \end{tikzcd}
                            $$
                        from which one infers that $\cev{h}_X^* \Frob_{\Bun_{\GL_1}(X)}^* \E_{\calF} \cong (\id_X \x \Frob_{\Bun_{\GL_1}(X)})^* \cev{h}_X^* \E_{\calF} \cong \calF \boxtimes \Frob_{\Bun_{\GL_1}(X)}^* \E_{\calF}$, which in turn tells us that $\Frob^*_{\Bun_{\GL_1}(X)} \E_{\calF}$ is a Hecke eigensheaf with eigenvalue $\calF$. This can be interpreted as $\cev{h}_X^*$ acting on $\E_{\calF}$ and $\Frob^*_{\Bun_{\GL_1}(X)} \E_{\calF}$ as $\calF \boxtimes -$, so we shall now attempt to show that $\calF \boxtimes -$ is a faithful functor. For this, recall that faithful flatness is a local property and so we can simply check whether $\calF_{\bar{x}} \tensor -$ is a faithful functor for all geometric points $\bar{x} \in X$. However, $\calF_{\bar{x}}$ is a finite-dimensional vector space, so it is faithfully flat \textit{a priori}, so the functor $\calF_{\bar{x}} \tensor -$ is necessarily faithful. As stated, this implies that $\E_{\calF} \cong \Frob_{\Bun_{\GL_1}(X)}^* \E_{\calF}$, i.e. that $\E_{\calF}$ is a Weil sheaf on $\Bun_{\GL_1}(X)$.
                    \end{proof}
                \begin{remark}[The character sheaf condition for $\Bun_{\GL_1}(X)$] \label{remark: character_sheaf_condition_for_BunGL1}
                    Let $\mu_X: \Bun_{\GL_1}(X) \x \Bun_{\GL_1}(X) \to \Bun_{\GL_1}(X)$ denote the group structure on $\Bun_{\GL_1}(X)$. Note that for the smooth commutative group $\F_q$-scheme $\Bun_{\GL_1}(X)$, the character sheaf condition (with respect to $\mu_X$) is equivalent to requiring that for all $d \in \Z$, one has $(\mu_X^{(d)})^*\E_{\calF^{(d + 1)}} \cong \E_{\calF^{(1)}} \boxtimes \E_{\calF^{(d)}}$, where $\mu_X^{(d)}: \Bun_{\GL_1}^{(1)}(X) \x \Bun_{\GL_1}^{(d)}(X) \to \Bun_{\GL_1}^{(d + 1)}(X)$ is given by $\mu_X^{(d)}(\calL', \calL) \cong \calL' \tensor_{\calO_X} \calL$. 
                \end{remark}
                \begin{lemma}[Hecke eigensheaves are character sheaves] \label{lemma: hecke_eigensheaves_are_character_sheaves}
                    There exists a group isomorphism:
                        $$\Eig\Shv_{\bar{\Q}_{\ell}}^1(\Bun_{\GL_1}(X)) \cong \Char\Shv_{\bar{\Q}_{\ell}}(\Bun_{\GL_1}(X))$$
                \end{lemma}
                    \begin{proof}
                        The group structures on $\Eig\Shv_{\bar{\Q}_{\ell}}^1(\Bun_{\GL_1}(X))$ and on $\Char\Shv_{\bar{\Q}_{\ell}}(\Bun_{\GL_1}(X))$ are both inherited from $\Shv_{\bar{\Q}_{\ell}}^{\lisse, 1}(\Bun_{\GL_1}(X))$, so it shall suffice to show that the two underlying sets are in bijetion.
                        \begin{itemize}
                            \item \textbf{(Hecke eigensheaves are character sheaves):} First of all, it is easy to check that the following diagram commutes:
                                $$
                                    \begin{tikzcd}
                                    	{X \x \Bun_{\GL_1}^{(d)}(X)} && {\Bun_{\GL_1}^{(1)}(X) \x \Bun_{\GL_1}^{(d)}(X)} \\
                                    	& {\Bun_{\GL_1}^{(d + 1)}(X)}
                                    	\arrow["{\cev{h}_X^{(d)}}"', from=1-1, to=2-2]
                                    	\arrow["{\AJ_{X/\F_q}^{(1)} \x \id_{\Bun_{\GL_1}^{(d)}(X)}}", from=1-1, to=1-3]
                                    	\arrow["{\mu_X^{(d)}}", from=1-3, to=2-2]
                                    \end{tikzcd}
                                $$
                            Next, recall that the $d^{th}$ Abel-Jacobi maps $\AJ_{X/\F_q}^{(d)}$ (for all $d \in \Z$) is invertible at the level of sheaf pullback, i.e. the functor:
                                $$(\AJ_{X/\F_q}^{(d)})^*: \Shv_{\bar{\Q}_{\ell}}^{\lisse, 1}(\Bun_{\GL_1}^{(d)}(X)) \to \Shv_{\bar{\Q}_{\ell}}^{\lisse, 1}(X^{(d)})$$
                            is an equivalence (cf. corollary \ref{coro: unramified_galois_representations_induced_by_the_abel_jacobi_map}). As a result, one has the following for all Hecke eigensheaves $\E_{\calF^{(d + 1)}} \in \Eig\Shv_{\bar{\Q}_{\ell}}^1(\Bun_{\GL_1}^{(d + 1)}(X))$ (corresponding to $\calF^{(d + 1)} \in \Shv_{\bar{\Q}_{\ell}}^{\lisse, 1}(X^{(d + 1)})$ in the sense of corollary \ref{coro: unramified_galois_representations_induced_by_the_abel_jacobi_map}) with eigenvalue $\calF \in \Shv_{\bar{\Q}_{\ell}}^{\lisse, 1}(X)$:
                                $$(\mu_X^{(d)})^*\E_{\calF^{(d + 1)}} \cong (\AJ_{X/\F_q}^{(1)} \x \id_{\Bun_{\GL_1}^{(d)}(X)})_* (\cev{h}_X^{(d)})^* \E_{\calF^{(d + 1)}} \cong (\AJ_{X/\F_q}^{(1)})_*\calF \boxtimes \E_{\calF^{(d)}}$$
                            wherein the lisse $\bar{\Q}_{\ell}$-sheaf $\E_{\calF^{(d)}}$ corresponds to $\calF^{(d)} \in \Shv_{\bar{\Q}_{\ell}}^{\lisse, 1}(X^{(d)})$ (again, in the sense of corollary \ref{coro: unramified_galois_representations_induced_by_the_abel_jacobi_map}). Now, observe that - again due to corollary \ref{coro: unramified_galois_representations_induced_by_the_abel_jacobi_map} - corresponding to:
                                $$(\AJ_{X/\F_q}^{(1)})_*\calF \in \Shv_{\bar{\Q}_{\ell}}^{\lisse, 1}(X^{(1)})$$
                            is a lisse $\bar{\Q}_{\ell}$-sheaf:
                                $$\E_{\calF^{(1)}} \in \Shv_{\bar{\Q}_{\ell}}^{\lisse, 1}(\Bun_{\GL_1}^{(1)}(X))$$
                            Putting everything together then yields $\mu_X^{(d)})^*\E_{\calF^{(d + 1)}} \cong \E_{\calF^{(1)}} \boxtimes \E_{\calF^{(d)}}$, which is precisely the character sheaf property for $\Bun_{\GL_1}(X)$.
                            \item \textbf{(Character sheaves are Hecke eigensheaves):} Let $\E$ be a character sheaf on $\Bun_{\GL_1}(X)$. Then, we have:
                                $$\cev{h}_X^*\E_{\calF^{(d + 1)}} \cong (\AJ_{X/\F_q}^{(1)} \x \id_{\Bun_{\GL_1}^{(d)}(X)})^* (\mu_X^{(d)})^* \E_{\calF^{(d + 1)}} \cong (\AJ_{X/\F_q}^{(1)})^*\E_{\calF^{(1)}} \boxtimes \E_{\calF^{(d)}}$$
                            Another application of corollary \ref{coro: unramified_galois_representations_induced_by_the_abel_jacobi_map} tells us that:
                                $$(\AJ_{X/\F_q}^{(1)})^*\E_{\calF^{(1)}} \in \Shv_{\bar{\Q}_{\ell}}^{\lisse, 1}(\Bun_{\GL_1}^{(1)}(X))$$
                            corresponds to:
                                $$\calF \in \Shv_{\bar{\Q}_{\ell}}^{\lisse, 1}(X)$$
                            This implies that:
                                $$\cev{h}_X^*\E_{\calF^{(d + 1)}} \cong \calF \boxtimes \E^{\calF^{(d)}}$$
                            which tells us that $\E$ is a Hecke eigensheaf with eigenvalue $\calF$.
                        \end{itemize}
                    \end{proof}
                \begin{convention}[Weil Uniformisation] \label{conv: weil_uniformisation}
                    For proposition \ref{prop: hecke_characters_from_hecke_eigensheaves}, suppose that we are willing to take for granted the special case of the Weil Uniformisation Theorem for $\Bun_{\GL_1}(X)$, which asserts that there is an isomorphism:
                        $$\Bun_{\GL_1}(X)(\F_q) \cong \GL_1(K_X)\backslash\GL_1(\A_X)/\GL_1(\bbO_{X})$$
                    For a proof, see \cite[Proposition 1.1.2]{toth_geometric_abelian_class_field_theory}.
                \end{convention}
                \begin{definition}[Hecke characters] \label{def: hecke_characters}
                    An $\ell$-adic \textbf{Hecke character} of $X$ (or rather, of $K_X$) is a continuous $\ell$-adic character of $\GL_1(K_X)\backslash\GL_1(\A_X)/\GL_1(\bbO_{X})$. They form a group, denoted by $\scrA_{\GL_1}(X, \bar{\Q}_{\ell})$.
                \end{definition}
                \begin{proposition}[Hecke characters from Hecke eigensheaves] \label{prop: hecke_characters_from_hecke_eigensheaves}
                    There is a group isomorphism via traces of Frobenii:
                        $$\trace(\Frob_{\Bun_{\GL_1}(X)}^* \mid -): \Eig\Shv_{\bar{\Q}_{\ell}}^1(\Bun_{\GL_1}(X)) \to \scrA_{\GL_1}(X, \bar{\Q}_{\ell})$$
                    through which Hecke characters are obtained from Hecke eigensheaves of rank $1$ on $\Bun_{\GL_1}(X)$.
                \end{proposition}
                    \begin{proof}
                        Thanks to lemma \ref{lemma: hecke_eigensheaves_are_character_sheaves}, we will only need to show that $\Char\Shv_{\bar{\Q}_{\ell}}^1(\Bun_{\GL_1}(X)) \cong \scrA_{\GL_1}(X, \bar{\Q}_{\ell})$ via taking traces of Frobenii. However, because $\pi_0(\Bun_{\GL_1}(X)) \cong \Z$ (cf. remark \ref{remark: geometry_of_the_picard_stack}), meaning that it is cyclic, we get the proposition immediately via applying theorem \ref{theorem: sheaf_function_correspondence_for_smooth_groups} followed by corollary \ref{coro: kernels_of_traces_of_frobenii_on_etale_group_schemes}.
                    \end{proof}
                
                \begin{theorem}[Artin Reciprocity for global function fields over finite fields] \label{theorem: artin_reciprocity_for_function_fields_over_finite_fields}
                    \cite[Theorem VI.5.5]{neukirch_2010_algebraic_number_theory} There is a group isomorphism:
                        $$\Artin_X: \Rep^1_{\bar{\Q}_{\ell}}(\pi_1(X_{\fet})) \xrightarrow[]{\cong} \scrA_{\GL_1}(X, \bar{\Q}_{\ell})$$
                    called the \textbf{Artin map}.
                \end{theorem}
                    \begin{proof}
                        Combine theorem \ref{prop: hecke_characters_from_hecke_eigensheaves} and corollary \ref{coro: geometric_langlands_for_GL1}. 
                    \end{proof}
                    
                \begin{lemma}[Abelianising continuous characters] \label{lemma: abelianising_continuous_characters}
                    Let $G$ be a topological group and $E$ a topological field. Then, there is a group isomorphism $\Rep^1_E(G^{\ab}) \cong \Rep^1_E(G)$.
                \end{lemma}
                    \begin{proof}
                        First of all, observe that the action of any $E$-character $\chi: G \to \GL_1(E)$ on the commutator subgroup $[G, G] \leq G$ is actually trivial, a consequence of the fact that $\GL_1(E)$ is abelian: for any $\gamma, \gamma' \in [G, G]$, one has $\chi([\gamma, \gamma']) = [\chi(\gamma), \chi(\gamma')] = 1_E$. As a result, $\Rep^1_E([G, G])$ is nothing but the trivial group. Now, because surjective continuous group homomorphisms are epimorphisms in the category of topological groups\footnote{One could also prove this by first showing that the forgetful functor $\Top\Grp \to \Grp$ admits a left-adjoint which sends any group $G \in \Grp$ to the associated discrete group $(G, \discrete)$, and then showing that because left/right adjoints preserve finite (co)limits, mono/epimorphisms of topological groups must be nothing but injective/surjective continuous homomorphisms.} (cf. \cite{monomorphisms_and_epimorphisms_of_topological_groups}), and because the functor $\Top\Grp(-, \GL_1(E)): \Top\Grp \to \Ab$ preserves all monomorphisms in $\Top\Grp^{\op}$ (i.e. all epimorphisms in $\Top\Grp$), the short exact sequence $1 \to [G, G] \to G \to G^{\ab} \to 1$ of topological groups induces a left-exact sequence $0 \to \Rep^1_E(G^{\ab}) \hookrightarrow \Rep^1_E(G) \to \Rep^1_E([G, G])$ of abelian groups. However, since $\Rep^1_E([G, G])$ is trivial (as shown above), this implies that $\Rep^1_E(G^{\ab}) \cong \Rep^1_E(G)$ as claimed.
                    \end{proof}
                \begin{lemma}[Galois groups of composite fields] \label{lemma: galois_groups_of_composite_fields}
                    \cite[Theorem 5.13]{keith_conrad_galois_correspondence} Let $K$ be a field, let $F/K$ be a Galois extension, and $L/K, L'/K$ be finite subextensions corresponding to subgroups $N, N' \leq \Gal(\bar{K}/K)$. Then, $\Aut(LL'/K) \cong N \cap N'$, and also, $NN' \cong \Aut(\bar{K}/L \cap L')$.
                \end{lemma}
                \begin{theorem}[Class field theory for global function fields over finite fields] \label{theorem: class_field_theory_for_global_function_fields_over_finite_fields}
                    \cite[Theorem VI.6.1]{neukirch_2010_algebraic_number_theory} There exists an equivalence of categories as below, such that the degree of the finite unramified abelian extensions and the indices of the subgroups coincide:
                        $$
                            \begin{tikzcd}
                            	{\{\text{Finite unramified abelian extensions $L/K_X$ and $K_X$-algebra homomorphisms}\}^{\op}} \\
                            	{\{\text{Finite-index subgroups of $\GL_1(K_X)\backslash\GL_1(\A_X)/\GL_1(\bbO_{X})$}\}}
                            	\arrow[from=1-1, to=2-1]
                            \end{tikzcd}
                        $$ 
                     \footnote{Note that everything is well defined because the composite extension $L L'/K_X$ is finite, abelian, and unramified (the first two properties are trivial, and the third is due to \cite[Corollary II.7.3]{neukirch_2010_algebraic_number_theory}), and because the group $N_{L L'}$ is indeed a subgroup of finite-index of $\GL_1(K_X)\backslash\GL_1(\A_X)/\GL_1(\bbO_{X})$, by virtue of being generated by two such subgroups, namely $N_L$ and $N_{L'}$.}Furthermore, for any pair of finite unramified abelian extensions $L/K_X, L'/K_X$ corresponding to subgroups $N_L N_{L'} \leq \GL_1(K_X)\backslash\GL_1(\A_X)/\GL_1(\bbO_{X})$, one has $N_{L L'} = N_L \cap N_{L'}$ and $N_{L \cap L'} = N_L N_{L'}$.
                \end{theorem}
                    \begin{proof}
                        First of all, we must apply lemma \ref{lemma: abelianising_continuous_characters}, which tells us that $\Rep^1_{\bar{\Q}_{\ell}}(\pi_1(X_{\fet})) \cong \Rep^1_{\bar{\Q}_{\ell}}(\pi_1^{\ab}(X_{\fet}))$. Then, observe that because the locally compact Hausdorff abelian group $\bar{\Q}_{\ell}^{\x}$ is divisble by virtue of being the group of units of a field, the functor $\Loc\Comp\Ab(-, \bar{\Q}_{\ell}^{\x}): \Comp\Ab^{\op} \to \Ab$ is exact \textit{a priori} (cf. remark \ref{remark: the_category_of_locally_compact_hausdorff_abelian_groups}). Consequently, we have the following commutative diagram wherein all the arrows are equivalences:
                            $$
                                \begin{tikzcd}[scale cd=0.75]
                                	{\{\text{Finite-index subgroups of $\pi_1^{\ab}(X_{\fet})$}\}^{\op}} & {\{\text{Finite-index subgroups of $\GL_1(K_X)\backslash\GL_1(\A_X)/\GL_1(\bbO_{X})$}\}^{\op}} \\
                                	{\{\text{Finite quotients of $\Rep^1_{\bar{\Q}_{\ell}}(\pi_1(X_{\fet}))$}\}} & {\{\text{Finite quotients of $\scrA_{\GL_1}(X, \bar{\Q}_{\ell})$}\}}
                                	\arrow["{\Comp\Ab(-, \bar{\Q}_{\ell}^{\x})}"', from=1-1, to=2-1]
                                	\arrow[dashed, from=1-1, to=1-2]
                                	\arrow["{\Comp\Ab(-, \bar{\Q}_{\ell}^{\x})^{-1}}"', from=2-2, to=1-2]
                                	\arrow["\Artin", from=2-1, to=2-2]
                                \end{tikzcd}
                            $$
                        From this, we get the following commutative diagram of categories, wherein the functors are also equivalences, and the bottom arrow is as in the previous diagram:
                            $$
                                \begin{tikzcd}[scale cd=0.75]
                                	{\{\text{Abelian finite \'etale covers of $X$}\}} & {\{\text{Finite unramified abelian extensions $L/K_X$ and $K_X$-algebra homomorphisms}\}^{\op}} \\
                                	{\{\text{Finite-index subgroups of $\pi_1^{\ab}(X_{\fet})$}\}} & {\{\text{Finite-index subgroups of $\GL_1(K_X)\backslash\GL_1(\A_X)/\GL_1(\bbO_{X})$}\}}
                                	\arrow[from=2-1, to=2-2]
                                	\arrow["{\text{Theorem \ref{theorem: geometric_galois_correspondence}}}"', from=1-1, to=2-1]
                                	\arrow["\text{Prop. \ref{prop: curves_and_function_fields}}"', from=1-2, to=1-1]
                                	\arrow[dashed, from=1-2, to=2-2]
                                \end{tikzcd}
                            $$
                        It is then obvious that the sought-for equivalence of categories exists:
                            $$
                                \begin{tikzcd}
                                	{\{\text{Finite unramified abelian extensions $L/K_X$ and $K_X$-algebra homomorphisms}\}^{\op}} \\
                                	{\{\text{Finite-index subgroups of $\GL_1(K_X)\backslash\GL_1(\A_X)/\GL_1(\bbO_{X})$}\}}
                                	\arrow[from=1-1, to=2-1]
                                \end{tikzcd}
                            $$ 
                            
                        Now, let $N_L$ be the finite-index subgroup of $\GL_1(K_X)\backslash\GL_1(\A_X)/\GL_1(\bbO_{X})$ that corresponds - via the equivalence of categories above - to a given finite unramified abelian extension $L/K_X$, and suppose that it is of index $e$. Such a subgroup corresponds to a subgroup of index $e$ of $\pi_1^{\ab}(X_{\fet})$. Since $\pi_1^{\ab}(X_{\fet}) \cong \Gal(K_X^{\ab}/K_X)$, one thus sees - via the Fundamental Theorem of Galois Theory (cf. \cite[\href{https://stacks.math.columbia.edu/tag/0BML}{Tag 0BML}]{stacks}) - that $L/K_X$ must be of degree $e$, as claimed.
                        
                        To prove the last assertion, note that since we have an equivalence between the categories of finite-index subgroups of $\pi_1^{\ab}(X_{\fet})$ and those of $\GL_1(K_X)\backslash\GL_1(\A_X)/\GL_1(\bbO_{X})$, and since $\pi_1^{\ab}(X_{\fet}) \cong \Gal(K_X^{\ab}/K_X)$, we can simply apply lemma \ref{lemma: galois_groups_of_composite_fields} directly.
                    \end{proof}
    
    \addcontentsline{toc}{section}{References}
    \printbibliography

\end{document}