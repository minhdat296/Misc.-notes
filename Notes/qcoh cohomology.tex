\input{article preambles}

\setcounter{section}{-1}

\input{commands}

\begin{document}

    \title{Cohomology of quasi-coherent sheaves on schemes}
    
    \author{Dat Minh Ha}
    \maketitle
    
    \begin{abstract}
        
    \end{abstract}
    
    {
      \hypersetup{} 
      %\dominitoc
      \tableofcontents %sort sections alphabetically
    }

    \section{Introduction}

    \section{Generalities on quasi-coherent sheaves}
        \subsection{The category of quasi-coherent sheaves}
            Suppose that $X$ is a scheme - say, with a fixed Zariski cover:
                $$\{ f_i: \Spec A_i \to X \}_{i \in I}$$
            and that we are interested in somehow \say{patching} together the categories $A_i\mod$\footnote{... viewed as sheaves of modules on the affine schemes $\Spec A_i$ - we will explain this point of view in more details shortly.} into a global category of sheaves of modules on $X$. A \textit{na\"ive} guess might be that one should just consider the category $\scrO_X\mod$ and be done with it: after all, this is a rather good category for homological algebraic purposes, being abelian and having enough injectives.
            
            However, one can do better, and one should want more out of a category of $\scrO_X$-modules. $\scrO_X\mod$ is without enough projectives, so whatever \say{the} derived category of $\scrO_X$-modules might be, its objects must be of cohomological degrees $\geq 0$ and we are thus unable to speak of left-derived functors. The category $\scrO_X\mod$ is also somehow \say{too large}: it is almost never compactly generated. The category $\QCoh(X)$ consisting of the so-called \say{quasi-coherent} $\scrO_X$-modules is a certain full subcategory of $\scrO_X\mod$ that helps us fix these issues.

            \begin{convention}
                Unless stated to be otherwise, all sites will be assumed to be small. 

                If $X$ is a scheme and $\tau$ is a topology on the category of schemes, then we will write $X_{\tau} := (\Sch_{/X})_{\tau}$ to mean the category $\Sch_{/X}$ equipped with the topology $\tau$. 

                
            \end{convention}

            We begin by seeing how, for any commutative ring $A$, the category $A\mod$ is the same as that of $\scrO_X$-modules for $X := \Spec A$. 
            \begin{lemma}[Sheafifying modules] \label{lemma: sheafifying_modules}
                Let $A$ be a commutative ring and set $X := \Spec A$. There is a fully faithful exact functor:
                    $$A\mod \hookrightarrow \scrO_X\mod$$
                sending each $A$-module $M$ to the $\scrO_X$-module ${}^{\sh}M$ on the (small) site $X_{\Zar}$ given by:
                    $${}^{\sh}M(\Spec A[1/f]) := M[1/f]$$
                for every $f \in A$. 
            \end{lemma}
            \begin{corollary}
                Let $A$ be a commutative, $X := \Spec A$, and consider a prime ideal $\p$ of $A$, defining a point $x \in |X|$. Then:
                    $${}^{\sh}M_x \cong M_{\p}$$
            \end{corollary}
            
            Now, let us see the definition of the category of quasi-coherent modules. This can be stated more generally for any site, not simply the Zariski site of a scheme, but we will not need to work in this full generality.
            \begin{definition}[Quasi-coherent modules] \label{def: quasi_coherent_modules}
                If $A$ is a commutative ring then we define the category $\QCoh(\Spec A)$ of \textbf{quasi-coherent $\scrO_{\Spec A}$-modules} to be the essential image of the fully faithful functor $A\mod \hookrightarrow \scrO_X\mod$ from lemma \ref{lemma: sheafifying_modules}.

                If $X$ is a scheme with a Zariski cover:
                    $$\{ f_i: \Spec A_i \to X \}_{i \in I}$$
                then the category $\QCoh(X)$ of quasi-coherent $\scrO_X$-modules will be the subcategory of $\scrO_X\mod$ consisting of those objects $\scrM$ such that for every $i \in I$, one has that:
                    $$f_i^*\scrM \cong {}^{\sh}M_i$$
                for some $A_i$-module $M_i$.
            \end{definition}
            \begin{remark}
                $\QCoh(X)$ is a full subcategory of $\scrO_X\mod$.
            \end{remark}
            \begin{remark}
                For any commutative ring $A$, the inverse functor:
                    $$\QCoh(\Spec A) \to A\mod$$
                is given simply by taking global sections, i.e.:
                    $$\scrM \mapsto \Gamma(\Spec A, \scrM)$$
            \end{remark}

            Now, it might be tempting to conclude immediately that for any scheme $X$ with a Zariski cover:
                $$\{f_i: \Spec A_i \to X\}_{i \in I}$$
            the category $\QCoh(X)$ is an AB5 category - and hence has enough injectives and all limits - as this is true for each of the categories $A_i\mod$. However, there are certain subtle issues of more-or-less a set-theoretic nature that prevent this from being an obvious result.

            We shall see below that for any scheme $X$, there shall exist a cardinal $\aleph_X$ such that every $\scrM \in \Ob( \QCoh(X) )$ is the filtered colimit of some diagram of $\aleph_X$-small quasi-coherent $\scrO_X$-submodules of $\scrM$. 
            \begin{definition}[Small quasi-coherent modules] \label{def: small_quasi_coherent_modules}
                Let $X$ be a scheme with a Zariski cover:
                    $$\{f_i: \Spec A_i \to X\}_{i \in I}$$
                and fix some cardinal $\aleph$. Also, let $\scrM$ be a quasi-coherent $\scrO_X$-module. If each $A_i$-module $\Gamma(\Spec A_i, f_i^*\scrM)$ is generated by a set of cardinality $\leq \aleph$, then we shall say that $\scrM$ is \textbf{$\aleph$-small}. 

                Denote the full subcategory of $\QCoh(X)$ consisting of $\aleph$-small objects by $\QCoh(X)^{\leq \aleph}$.
            \end{definition}
            \begin{lemma}[Closure properties of small modules] \label{lemma: closure_properties_of_small_quasi_coherent_modules}
                Let $X$ be a scheme and fix an infinite cardinal $\aleph$. Then, $\QCoh(X)^{\leq \aleph}$ will be closed under direct sums and tensor products over $\scrO_X$.
            \end{lemma}
                \begin{proof}
                    
                \end{proof}
            \begin{theorem}[Gabber's Theorem: Homological properties of $\QCoh$] \label{theorem: qcoh_homological_properties}
                Let $X$ be a scheme. Then $\QCoh(X)$ will be an AB5 category and hence will have enough injectives and all limits. 
            \end{theorem}
                \begin{proof}
                    
                \end{proof}
            \begin{corollary}
                Let $X$ be a scheme. Then, the canonical fully faithful embedding:
                    $$\QCoh(X) \hookrightarrow \scrO_X\mod$$
                admits a quasi-inverse right-adjoint $Q_X: \scrO_X\mod \to \QCoh(X)$, usually called the \textbf{coherator}.
            \end{corollary}
                
            \begin{proposition}[Monoidal properties of $\QCoh$] \label{prop: qcoh_monoidal_properties}
                Let $X$ be a scheme. $\QCoh(X)$ is symmetric monoidal via $\tensor_{\scrO_X}$, but is is \textit{not} monoidally closed: in general, if $\scrF, \scrG \in \Ob(\QCoh(X))$ then $\Hom_{\scrO_X}(\scrF, \scrG)$ may fail to be quasi-coherent. That said, if $\scrF$ is finitely presented\footnote{Let $X$ be a scheme with a fixed Zariski covering $\{ f_i: \Spec A_i \to X \}_{i \in I}$. An $\scrO_X$-module $\scrM$ is said to be \textbf{finitely presented} if and only if for each $i \in I$, there exist $m_i, n_i \in \N$ such that $f_i^*\scrF \cong \coker\left( \scrO_{\Spec A_i}^{\oplus m_i} \to \scrO_{\Spec A_i}^{\oplus n_i} \right)$.} as an $\scrO_X$-module then $\Hom_{\scrO_X}(\scrF, \scrG)$ will be quasi-coherent. 
            \end{proposition}
            As all good constructions in algebraic geometry are, the creation of quasi-coherent modules is also functorial, albeit in a slightly weak sense. 
            \begin{proposition}[Functoriality of $\QCoh$] \label{prop: qcoh_functoriality}
                The assignment:
                    $$(\pi: Y \to X) \mapsto ( \pi^*: \QCoh(X) \to \QCoh(Y) )$$
                defines a stack (in categories!) over $X_{\Zar}$. 
            \end{proposition}

        \subsection{Finiteness and vanishing theorems}
            In order to be able to discuss the question of finiteness of cohomology of quasi-coherent modules, we firstly need to impose some finiteness conditions on these modules. Doing so yields the notion of \say{coherent modules}.
        
            Proposition \ref{prop: qcoh_monoidal_properties} prompts the following definition:
            \begin{definition}[Coherent modules] \label{def: coherent_modules}
                Let $X$ be a scheme with a fixed Zariski covering:
                    $$\{ f_i: \Spec A_i \to X \}_{i \in I}$$
                The category $\Coh(X)$ of \textbf{coherent $\scrO_X$-modules} will then be the full subcategory of $\QCoh(X)$ consisting of finite-type objects, which is to say that $\scrM \in \Ob( \QCoh(X) )$ is coherent if and only if $\Gamma(\Spec A_i, f_i^*\scrM)$ is a finite $A_i$-module for every $i \in I$.
            \end{definition}
            Even though we just gave a definition of coherent sheaves on a general scheme, we will usually be considering coherent $\scrO_X$-modules over \textit{locally Noetherian} schemes $X$, for the same reason that finitely generated modules are usually considered over Noetherian rings. 
            \begin{proposition}[Coherent modules over (locally) Noetherian schemes] \label{prop: coherent_modules_over_noetherian_schemes}
                Let $X$ be a locally Noetherian scheme. Then, an $\scrO_X$-module $\scrM$ will be coherent if and only if it is finitely presented.
            \end{proposition}
            \begin{example}
                Let $X$ be a locally Noetherian scheme. Then the structure sheaf $\scrO_X$ itself is coherent. 
            \end{example}
            \begin{example}[Incoherent rings]
                If $X$ is a scheme such that $\scrO_X$ is not coherent as a module over itself, then naturally, we shall refer to $\scrO_X$ as an \textbf{incoherent sheaf of rings}. There is the following example of such a ring, due to Brian Conrad: \href{http://math.stanford.edu/~vakil/216blog/incoherent.pdf}{http://math.stanford.edu/~vakil/216blog/incoherent.pdf}.
            \end{example}

        \subsection{Base-changing cohomology: the pull-push yoga}

    \section{Duality theorems}
        \subsection{Serre Duality}

        \subsection{Verdier Duality}

    \section{Zariski's Main Theorem}
        \subsection{The Theorem of Formal Functions}

        \subsection{Zariski's Main Theorem}
    
    \addcontentsline{toc}{section}{References}
    \printbibliography

\end{document}