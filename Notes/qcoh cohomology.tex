\input{article preambles}

\setcounter{section}{-1}

\input{commands}

\begin{document}

    \title{Cohomology of quasi-coherent sheaves on schemes}
    
    \author{Dat Minh Ha}
    \maketitle
    
    \begin{abstract}
        
    \end{abstract}
    
    {
      \hypersetup{} 
      %\dominitoc
      \tableofcontents %sort sections alphabetically
    }

    \section{Introduction}
        \begin{convention}
            If we have functors:
                $$L: \C \to \D, R: \D \to \C$$
            such that:
                $$L \ladjoint R$$
            then we will usually write out the adjunction as:
                $$L: \C \leftrightarrows \D: R$$
            with $L$ being on the left to signify that it is the left-adjoint component. 
        \end{convention}

    \section{A six-functor formalism for quasi-coherent modules on a scheme}
        \subsection{The category of quasi-coherent modules on a scheme; compact generation}
            Suppose that $X$ is a scheme - say, with a fixed Zariski cover:
                $$\{ f_i: \Spec A_i \to X \}_{i \in I}$$
            and that we are interested in somehow \say{patching} together the categories $A_i\mod$\footnote{... viewed as sheaves of modules on the affine schemes $\Spec A_i$ - we will explain this point of view in more details shortly.} into a global category of sheaves of modules on $X$. A \textit{na\"ive} guess might be that one should just consider the category $\scrO_X\mod$ and be done with it: after all, this is a rather good category for homological algebraic purposes, being abelian and having enough injectives.
            
            However, we can do better. Morally speaking, the category of quasi-coherent $\scrO_X$-modules - typically denoted by $\QCoh(X)$ - is obtained by taking the \say{$2$-limit} (whatever that might mean) along the pullbacks:
                $$f^*_i: \scrO_X\mod \to A_i\mod$$
            thus making it what one obtains by \say{gluing} together the module categories $A_i\mod$. As such, one of the first properties of the category $\QCoh(X)$ that more-or-less is by construction, is that locally, it has enough projectives, since each $A_i\mod$ has enough projectives; it should be noted that in general, $\QCoh(X)$ fails to have enough \say{global} projectives. Furthermore, unlike $\scrO_X\mod$, $\QCoh(X)$ is compactly generated. This is a somewhat non-trivial phenomenon, but it does mean that (co)limits of objects of $\QCoh(X)$ will be easier to handle than those of $\scrO_X\mod$. Of course, $\QCoh(X)$ retains all the nice properties of $\scrO_X\mod$, such as being abelian (in fact, $\QCoh(X)$ is AB5) and having enough injectives. We will also see that it is both complete and cocomplete, closed under extensions, and filtered colimits in $\QCoh(X)$ are exact.

            \begin{convention}
                Unless stated to be otherwise, all sites will be assumed to be small. 

                If $X$ is a scheme and $\tau$ is a topology on the category of schemes, then we will write $X_{\tau} := (\Sch_{/X})_{\tau}$ to mean the category $\Sch_{/X}$ equipped with the topology $\tau$. 
            \end{convention}

            We begin by seeing how, for any commutative ring $A$, the category $A\mod$ is the same as that of $\scrO_X$-modules for $X := \Spec A$. 
            \begin{lemma}[Sheafifying modules] \label{lemma: sheafifying_modules}
                Let $A$ be a commutative ring and set $X := \Spec A$. There is a fully faithful exact functor:
                    $$A\mod \hookrightarrow \scrO_X\mod$$
                sending each $A$-module $M$ to the $\scrO_X$-module ${}^{\sh}M$ on the (small) site $X_{\Zar}$ given by:
                    $${}^{\sh}M(\Spec A[1/f]) := M[1/f]$$
                for every $f \in A$. 
            \end{lemma}
            \begin{corollary}
                Let $A$ be a commutative, $X := \Spec A$, and consider a prime ideal $\p$ of $A$, defining a point $x \in |X|$. Then:
                    $${}^{\sh}M_x \cong M_{\p}$$
            \end{corollary}
            
            Now, let us see the definition of the category of quasi-coherent modules. This can be stated more generally for any site, not simply the Zariski site of a scheme, but we will not need to work in this full generality.
            \begin{definition}[Quasi-coherent modules] \label{def: quasi_coherent_modules}
                If $A$ is a commutative ring then we define the category $\QCoh(\Spec A)$ of \textbf{quasi-coherent $\scrO_{\Spec A}$-modules} to be the essential image of the fully faithful functor $A\mod \hookrightarrow \scrO_X\mod$ from lemma \ref{lemma: sheafifying_modules}.

                If $X$ is a scheme with a Zariski cover:
                    $$\{ f_i: \Spec A_i \to X \}_{i \in I}$$
                then the category $\QCoh(X)$ of quasi-coherent $\scrO_X$-modules will be the subcategory of $\scrO_X\mod$ consisting of those objects $\scrM$ such that for every $i \in I$, one has that:
                    $$f_i^*\scrM \cong {}^{\sh}M_i$$
                for some $A_i$-module $M_i$.
            \end{definition}
            \begin{remark}
                $\QCoh(X)$ is a full subcategory of $\scrO_X\mod$.
            \end{remark}
            \begin{theorem}[$\QCoh$ global section functor admits right-adjoint] \label{theorem: qcoh_global_section_functor_admits_right_adjoint}
                Suppose that $X$ is a scheme with a Zariski covering:
                    $$\{f_i: \Spec A_i \to X\}_{i \in I}$$
                Then, for each $i \in I$, there is an adjunction\footnote{In fact, since the functor ${}^{\sh}(-): A\mod \to \scrO_{\Spec A}\mod$ is fully faithful (cf. lemma \ref{lemma: sheafifying_modules}), this adjunction is a reflective localisation. This means that in a certain technical sense, for any scheme $X$ with a Zariski cover $\{f_i: \Spec A_i \to X\}_{i \in I}$, the process of taking local sections via the functors $\Gamma(\Spec A_i, f_i^*(-)): f_i^*\scrO_X\mod \to A_i\mod$ really is localising the category $\scrO_X\mod$ to yield $A_i\mod$.}:
                    $${}^{\sh}(-): A_i\mod \leftrightarrows \scrO_X\mod: \Gamma(\Spec A_i, f_i^*(-))$$
                    
                When $X := \Spec A$ is affine, these adjunctions yield us adjoint equivalences of the form:
                    $${}^{\sh}(-): A\mod \leftrightarrows \QCoh(X): \Gamma(X, -)$$
            \end{theorem}
                \begin{proof}
                    
                \end{proof}

            Now that we have a working definition of what a quasi-coherent module over the structure sheaf of a scheme is, let us investigate the homological properties of the category of such modules. 

            Our first agenda item is Gabber's Theorem, which tells us that for $X$ a scheme, the category $\QCoh(X)$ is a Grothendieck AB5 category - i.e. it is an abelian category that is complete and cocomplete, closed under extensions, wherein filtered colimits are exact, and it has a small generating \textit{set} - and has enough injectives. This is a non-trivial result, but very necessary as we would like to be assured that homological algebra within $\QCoh(X)$ is possible. Along the way, we will also be discussing the failure of $\QCoh(X)$ to globally possess enough projective objects; this will entail looking at locally projective modules. 
            \begin{remark}[A historical aside]
                Before complex geometry \textit{\`a la} Serre and especially before scheme theory, the kind of linear-algebraic objects that were considered over spaces were not general sheaves of modules over the structure sheaf of a space, but merely vector bundles, which we now view as finite locally free modules over the structure sheaf. 

                Let us briefly remind ourselves of some shortcomings of vector bundles. Consider a general commutative ring $A$ and the category $\Vect(A)$ of finite locally free $A$-modules., i.e. $A$-modules $\scrE$ such that for every $\p \in |\Spec A|$, the localisation $\scrE_{\p}$ is finite and free over $A_{\p}$. In general, this category fails to be abelian: e.g. \todo{Find a counter-example}
            \end{remark}

            \begin{definition}[Locally projective modules] \label{def: locally_projective_modules}
                Let $X$ be a scheme with a Zariski cover:
                    $$\{f_i: \Spec A_i \to X\}_{i \in I}$$
                We say that a quasi-coherent $\scrO_X$-module $\scrM$ is \textbf{locally projective} if and only if the $A_i$-module $M_i := \Gamma(\Spec A_i, f_i^*\scrM)$ is projective for every $i \in I$. Note that this definition only makes sense for $\scrM$ being quasi-coherent, since otherwise $M_i$ might fail to be an object of $A_i\mod$.
            \end{definition}
            \begin{remark}[$\QCoh$ has enough local projectives]
                Let $X$ be a scheme with a Zariski cover:
                    $$\{f_i: \Spec A_i \to X\}_{i \in I}$$
                Since each of the categories $A_i\mod$ has enough projectives, it is clear that $\QCoh(X)$ locally has enough projectives, in the sense that for every object $\scrM \in \Ob( \QCoh(X) )$, there is an epimorphism:
                    $$\scrP \to \scrM$$
                from some \textit{locally} projective object $\scrP \in \Ob( \QCoh(X) )$. 
            \end{remark}
            
            \begin{definition}[Small quasi-coherent modules] \label{def: small_quasi_coherent_modules}
                Let $X$ be a scheme with a Zariski cover:
                    $$\{f_i: \Spec A_i \to X\}_{i \in I}$$
                and fix some cardinal $\aleph$. Also, let $\scrM$ be a quasi-coherent $\scrO_X$-module. If each $A_i$-module $\Gamma(\Spec A_i, f_i^*\scrM)$ is generated by a set of cardinality $\leq \aleph$, then we shall say that $\scrM$ is \textbf{$\aleph$-small}. 

                Denote the full subcategory of $\QCoh(X)$ consisting of $\aleph$-small objects by $\QCoh(X)_{\leq \aleph}$.
            \end{definition}
            \begin{lemma}[Closure properties of small modules] \label{lemma: closure_properties_of_small_quasi_coherent_modules}
                Let $X$ be a scheme and fix an infinite cardinal $\aleph$. Then, $\QCoh(X)_{\leq \aleph}$ will be closed under direct sums and tensor products over $\scrO_X$.
            \end{lemma}
                \begin{proof}
                    
                \end{proof}
            \begin{theorem}[Gabber's Theorem: Homological properties of $\QCoh$] \label{theorem: qcoh_homological_properties}
                Let $X$ be a scheme. Then $\QCoh(X)$ will be an AB5 category and hence will have enough injectives and all limits. 
            \end{theorem}
                \begin{proof}
                    
                \end{proof}
            \begin{corollary}
                Let $X$ be a scheme. Then, the canonical fully faithful embedding:
                    $$\QCoh(X) \hookrightarrow \scrO_X\mod$$
                admits a quasi-inverse right-adjoint $Q_X: \scrO_X\mod \to \QCoh(X)$, usually called the \textbf{coherator}.
            \end{corollary}
            
            Before we move on, let us consider the following meta-definition, which in itself is not quite a well-defined mathematical notion, but nevertheless is a good schematic for what a \say{good} cohomology theory should look like. Over the course of the rest of this section, we shall see how quasi-coherent modules enjoy one such cohomology theory.
            \begin{definition}[Six-functor formalisms] \label{def: six_functor_formalisms}
                A six-functor formalism is a datum satisfying the following hypotheses:
                \begin{itemize}
                    \item A pair $(\C, E)$ of a finitely complete $(\infty, 1)$-category $\C$ (e.g. $\C := \Sch$) and a specified class $E \subseteq \Mor(\C)$ of morphisms which:
                    \begin{itemize}
                        \item contains all isomorphisms,
                        \item is closed under pullbacks, and
                        \item is closed under compositions.
                    \end{itemize}
                    \item \textbf{(Functoriality):} Two $(\infty, 2)$-functors:
                        $$\Shv^*, \Shv^!: \C^{\op} \to (\infty, 1)\-\Stab\Cat_2$$
                    which assign to each object $X \in \Ob(\C)$ a stable $(\infty, 1)$-category $\Shv(X)$ and to each morphism $(f: X \to Y) \in \Mor(\C)$, two respective adjunctions:
                        $$f_*: \Shv(X) \leftrightarrows \Shv(Y): f_*$$
                        $$f^!: \Shv(Y) \leftrightarrows \Shv(X): f_!$$
                    At this point, we have a \textbf{four-functor formalism}; the functors $f_*$ and $f_!$ respectively, are called the \textbf{$*$- and $!$-pushforwards} respectively, whereas the functors $f^*$ and $f^!$ are the \textbf{$*$- and $!$-pullbacks}. 
                    \item \textbf{(Base-change):} For each $(f: X \to Y) \in E$, the $!$-pushforward:
                        $$f_!: \Shv(X) \to \Shv(Y)$$
                    is to satisfy the following property: for each $(\infty, 1)$-pullback square:
                        $$
                            \begin{tikzcd}
                        	{X'} & X \\
                        	{Y'} & Y
                        	\arrow["{f'}"', from=1-1, to=2-1]
                        	\arrow["q", from=2-1, to=2-2]
                        	\arrow["p", from=1-1, to=1-2]
                        	\arrow["f", from=1-2, to=2-2]
                        	\arrow["\lrcorner"{anchor=center, pos=0.125}, draw=none, from=1-1, to=2-2]
                            \end{tikzcd}
                        $$
                    there is a natural isomorphism of $(\infty, 1)$-functors $\Shv(X) \to \Shv(Y')$:
                        $$q^* f_! \xrightarrow[]{\cong} f'_! p^*$$
                    and these natural isomorphisms should be compatible with pasting $(\infty, 1)$-pullbacks. 
                    \item \textbf{(Duality):} As a stable $(\infty, 1)$-category, each $\Shv(X)$ ought to be self-dual in the sense of Lurie. Moreover, this duality ought to exchange the adjunctions $(f^* \ladjoint f_*)$ and $(f_! \ladjoint f^!)$ with one another. 
                    \item \textbf{(Tensoriality):} For each $X, Y \in \Ob(\C)$, there ought to be an $(\infty, 1)$-functor:
                        $$\boxtimes_{X, Y}: \Shv(X) \tensor \Shv(Y) \to \Shv(X \x Y)$$
                    (with $\Shv(X) \tensor \Shv(Y)$ being in the sense of Lurie) natural in $X, Y$, which is such that for every $X \in \Ob(\C)$ (and let us write $\Delta_X: X \to X^2$ for the diagonal thereof), the following composition defines a closed symmetric monoidal structure on $\Shv(X)$:
                        $$\Shv(X)^{\tensor 2} \xrightarrow[]{\boxtimes_{X, X}} \Shv(X^2) \xrightarrow[]{\Delta_X^*} \Shv(X)$$
                    These tensor products and their corresponding internal hom-functors are the remaining two functors in the six-functor formalism. 
                        
                    Furthermore, we require that the $*$-pullback:
                        $$f^*: \Shv(Y) \to \Shv(X)$$
                    is $(\infty, 1)$-monoidal for each $(f: X \to Y) \in \Mor(\C)$ with respect to the aforementioned closed symmetric monoidal structures on $\Shv(Y), \Shv(X)$.

                    Finally, we require that with respect to any $(f: X \to Y) \in \Mor(\C)$, $\Shv(X)$ becomes a module over the closed symmetric monoidal category $\Shv(Y)$ in the following way.  
                \end{itemize}
            \end{definition}
            \begin{example}
                Take $\C$ to be the $1$-category $\Sch_{/S}$ of schemes (viewed as a trivial $(\infty, 1)$-category) - or even the $(\infty, 1)$-category of derived schemes - over a fixed base scheme $S$. Next, consider $E$ to either be the class of open immersions or of proper morphisms; we shall write $E_{\open}$ and $E_{\proper}$ to denote these classes, respectively.

                Consider firstly:
                    $$\QCoh^*: \Sch_{/S}^{\op} \to (\infty, 1)\-\Cat_2$$
                to be the $(\infty, 2)$-functor as in proposition \ref{prop: qcoh_*_pullbacks}, sending $S$-schemes $X$ to the categories of quasi-coherent $\scrO_X$-modules and sending morphisms of $S$-schemes:
                    $$f: X \to Y$$
                to (t-exact) left-derived $*$-pullbacks:
                    $$Lf^*: \rmD(\QCoh(Y)) \to \rmD(\QCoh(X))$$
                between the derived categories of $\scrO_Y, \scrO_X$-modules with quasi-coherent cohomologies, viewed as stable $(\infty, 1)$-categories; the heart of the standartd t-structures on these categories are nothing but the abelian categories $\QCoh(Y)$ and $\QCoh(X)$. 
            \end{example}

        \subsection{Functorial characterisations of properties of morphisms of schemes; the pull-push yoga for quasi-coherent modules}
            \begin{convention}
                We write $(\infty, 1)-\Stab\Cat_r$ to mean the $(\infty, r)$-category ($r = 1, 2$) of stable $(\infty, 1)$-categories, $(\infty, 1)$-functors between them, and natural transformations between said functors when $r = 2$.
            \end{convention}
        
            As all good constructions in algebraic geometry are, the creation of quasi-coherent modules is also functorial, albeit in a slightly weak sense. 
            \begin{proposition}[$*$-pullbacks and pushforwards of quasi-coherent modules] \label{prop: qcoh_*_pullbacks}
                Fix a base scheme $S$. There is a weak $(\infty, 2)$-functor:
                    $$\QCoh^*: \Sch_{/S}^{\op} \to (\infty, 1)\-\Cat_2$$
                assigning to each $X \in \Ob(\Sch_{/S})$ the category $\QCoh(X)$ as in definition \ref{def: quasi_coherent_modules} and to each $(f: X \to Y) \in \Mor(\Sch_{/S})$ the colimit-preserving pullback functor:
                    $$f^*: \QCoh(Y) \to \QCoh(X)$$
                This functor admits a right-adjoint $f_*$ per the Adjoint Functor Theorem. 
            \end{proposition}

            \begin{definition}[Dual stable $(\infty, 1)$-categories] \label{def: dual_stable_(infinity, 1)_categories}
                
            \end{definition}
            
            Pushing forward along proper morphisms is rather special. For convenience, let us recall the definition of properness in the context of morphisms of schemes:
            \begin{definition}[Proper morphisms] \label{def: proper_morphisms}
                
            \end{definition}
            One should keep in mind that:
            \begin{example}
                Closed immersions are proper.
            \end{example}
            \begin{lemma}[Dualising complexes and properly supported cohomology] \label{lemma: dualising_complexes_and_properly_supported_cohomology}
                
            \end{lemma}
                \begin{proof}
                    
                \end{proof}
            \begin{theorem}[Proper pushforwards] \label{theorem: proper_pushforwards}
                If $f: X \to Y$ is morphism of schemes then there shall exist a natural transformation of functors $\rmD(\QCoh(X)) \to \rmD(\QCoh(Y))$:
                    $$Lf_! \to Rf_*$$
                which will be a natural isomorphism when $f: X \to Y$ is proper. Equivalently, there is a natural transformation:
                    $$\id \to Rf^! \circ Rf_*$$
                that becomes the unit of adjunction when $f: X \to Y$ is proper. 
            \end{theorem}
                \begin{proof}
                    
                \end{proof}

            The first application of functoriality is the so-called Serre's Affineness Criterion, which gives a cohomological characterisation of affine morphisms amongst all qcqs morphisms $f: X \to Y$ of schemes via a verification of whether or not the pushforward $f_*\scrO_X$ is a compact projective generator of $\QCoh(Y)$. The absolute version of this theorem - which of course does not rely on functoriality of $\QCoh$ - says that a qcqs scheme $X$ is affine if and only if $\scrO_X$ is a compact projective generator of $\QCoh(X)$.
            
            For convenience, let us firstly recall what it means for a morphism of schemes to be quasi-compact and quasi-separated.
            \begin{definition}[Quasi-compactness and quasi-separatedness] \label{def: qcqs}
                A morphism of schemes $f: X \to Y$ is \textbf{quasi-compact (qc)} if for every quasi-compact open subset $|V| \subset |Y|$, the inverse image $|f|^{-1}(|V|)$ is a quasi-compact open subset of $|X|$. Recall that a topological space $S$ is \textbf{quasi-compact} if and only if every open covering thereof is refined by a finite subcover. 

                A morphism of schemes $f: X \to Y$ is \textbf{quasi-separated (qs)} if and only if its diagonal $\Delta_{X/Y}: X \to X \x_Y X$ is qc.
            \end{definition}
            \begin{remark}
                Being qcqs for schemes is to be thought of as being analogous to being (quasi-)compact and Hausdorff in point-set topology. In particular, the notion of being qcqs is meant to mirror the fact that, only inside a Hausdorff topological space is one guaranteed that the intersection of two compact subsets is once again compact.  
            \end{remark}
            \begin{lemma} \label{lemma: affine_morphisms_are_qcqs}
                Affine morphisms of schemes are qcqs.
            \end{lemma}
            \begin{lemma}
                Let $\calA$ be a cocomplete abelian category with a compact projective generator $A$. Then $\scrA$ will be equivalent to the category of left-modules over the endomorphism algebra $\calA(A, A)$:
                    $$\calA \cong {}^l\calA(A, A)\mod$$
            \end{lemma}
                \begin{proof}
                    
                \end{proof}
            \begin{theorem}[Compact generation of $\QCoh$ on affine schemes] \label{theorem: compact_generation_of_qcoh_on_affine_schemes}
                Let $X$ be a qcqs scheme. Then $X$ is affine if and only if $\scrO_X$ is a compact projective generator of the cocomplete abelian category $\QCoh(X)$.
            \end{theorem}
                \begin{proof}
                    Clear from the fact that $\Gamma(X, \scrO_X) = \End_{\scrO_X}(\scrO_X)$ by definition.
                \end{proof}
            \begin{theorem}[Serre's Affineness Criteria] \label{theorem: serre_affineness_criteria}
                For any qcqs morphism of schemes $f: X \to Y$, the following are equivalent:
                \begin{enumerate}
                    \item For any $\scrM \in \Ob(\QCoh(X))$, one has that:
                        $$i > 0 \implies R^i f_*\scrM \cong 0$$
                    \item $f_*\scrO_X$ is a compact projective generator of $\QCoh(Y)$. 
                    \item $f: X \to Y$ is affine\footnote{Note that $X$ is necessarily qcqs in this case (cf. lemma \ref{lemma: affine_morphisms_are_qcqs}).}.
                    \item For any ideal sheaf $\scrI \subset \scrO_X$ defined by a closed subscheme $Z \subset X$, one has that:
                        $$R^1 f_*\scrI \cong 0$$
                \end{enumerate}
            \end{theorem}
                \begin{proof}
                    \begin{enumerate}
                        \item Assume that 1. holds and consider $\scrM \cong \scrO_X$. By hypothesis, we have that:
                            $$i > 0 \implies R^i f_*\scrO_X \cong 0$$
                        \item 
                        \item 
                        \item 
                    \end{enumerate}
                \end{proof}
            \begin{remark}
                Amongst the criteria from theorem \ref{theorem: serre_affineness_criteria}, the condition that if any ideal sheaf $\scrI \subset \scrO_X$ defined by a closed subscheme $Z \subset X$, one has that:
                    $$R^1 f_*\scrI \cong 0$$
                then the morphism $f: X \to Y$ is affine deserves some attention, as it is perhaps the most closely related to how affineness is perceived in classical algebraic geometry and complex geometry. 

                A closed subscheme $Z \subset X$ consists of the vanishing loci of systems of polynomial equations generating the ideals $\Gamma(U, j^*\scrI) \subset \Gamma(U, j^*\scrO_X)$\footnote{\textit{A priori}, given a morphism of schemes $f: X \to Y$ and an ideal sheaf $\scrJ \subset \scrO_Y$, the pullback $f^*\scrJ$ is an $f^*\scrO_X$-ideal if and only if $f$ is flat, but not in general, since $f^*$ might fail to be left-exact. In this case, we are fine since open immersions are flat.}, for any open immersion $j: U \hookrightarrow X$. 
            \end{remark}

            \begin{theorem}[Extending $\QCoh$ from qc open subschemes] \label{theorem: extending_qcoh_from_qc_open_subschemes}
                Let:
                    $$j: U \hookrightarrow X$$
                be a qcqs\footnote{Technically, we only need to assume quasi-compactness, since open immersions are separated and hence quasi-separated \textit{a priori}.} open immersion. Then one has a natural isomorphism of functors $\rmD(\QCoh(X)) \to \rmD(\QCoh(U))$:
                    $$Rj^! \xrightarrow[]{\cong} Lj^*$$
                Equivalently, there is an adjunction counit:
                    $$Rf^! \circ Rf_* \xrightarrow[]{\cong} \id$$
            \end{theorem}
                \begin{proof}
                    
                \end{proof}

        \subsection{Cohomological base-change}

        \subsection{Duality}

        \subsection{Tensor structures}

    \section{Some applications}
        \subsection{Coherent sheaves and perfect complexes; cohomology of projective spaces}
            \begin{definition}[Coherent modules] \label{def: coherent_modules}
                Let $X$ be a scheme with a fixed Zariski covering:
                    $$\{ f_i: \Spec A_i \to X \}_{i \in I}$$
                The category $\Coh(X)$ of \textbf{coherent $\scrO_X$-modules} will then be the full subcategory of $\QCoh(X)$ consisting of finite-type objects, which is to say that $\scrM \in \Ob( \QCoh(X) )$ is coherent if and only if $\Gamma(\Spec A_i, f_i^*\scrM)$ is a finite $A_i$-module for every $i \in I$.
            \end{definition}
            Even though we just gave a definition of coherent sheaves on a general scheme, we will usually be considering coherent $\scrO_X$-modules over \textit{locally Noetherian} schemes $X$, for the same reason that finitely generated modules are usually considered over Noetherian rings. 
            \begin{proposition}[Coherent modules over (locally) Noetherian schemes] \label{prop: coherent_modules_over_noetherian_schemes}
                Let $X$ be a locally Noetherian scheme. Then, an $\scrO_X$-module $\scrM$ will be coherent if and only if it is finitely presented.
            \end{proposition}
            \begin{example}
                Let $X$ be a locally Noetherian scheme. Then the structure sheaf $\scrO_X$ itself is coherent. 
            \end{example}
            \begin{example}[Incoherent rings]
                If $X$ is a scheme such that $\scrO_X$ is not coherent as a module over itself, then naturally, we shall refer to $\scrO_X$ as an \textbf{incoherent sheaf of rings}. There is the following example of such a ring, due to Brian Conrad: \href{http://math.stanford.edu/~vakil/216blog/incoherent.pdf}{http://math.stanford.edu/~vakil/216blog/incoherent.pdf}.
            \end{example}

        \subsection{Zariski's Main Theorem}
            \begin{theorem}[The Theorem of Formal Functions] \label{theorem: formal_function_theorem}

            \end{theorem}
                \begin{proof}
                    
                \end{proof}

            \begin{theorem}[Zariski's Main Theorem] \label{theorem: zariski_main_theorem}
                
            \end{theorem}
                \begin{proof}
                    
                \end{proof}

            \begin{theorem}[Stein's Factorisation Theorem] \label{theorem: stein_factorisation}
                
            \end{theorem}

        \subsection{The archimedean GAGA principle of Serre}
            \begin{convention}
                In this subsection, we work exclusively over $\bbC$. 
            \end{convention}

            \begin{lemma}[Complex-analytic completions of finite-type $\bbC$-algebras] \label{lemma: complex_analytic_completions_of_finite_type_C_algebras}
                There is an \textbf{analytification functor}:
                    $$(-)^{\an}: \bbC\-\Comm\Alg^{\ft} \to \bbC\-\Ban\Comm\Alg^{\locconvex}$$
                from the category of finite-type $\bbC$-algebras to that of locally convex Banach $\bbC$-algebras, sending objects $A$ of the former to objects $A^{\an}$ of the former, given as the complex-analytic completion of $A$. Furthermore, this functor is compatible with base-change in the sense that, if:
                    $$\phi: A \to B$$
                is a homomorphism of finite-type $\bbC$-algebras then:
                    $$B^{\an} \cong (A \tensor_{A, \phi} B)^{\an} \cong A^{\an} \hattensor_{A, \phi} B$$
            \end{lemma}
                \begin{proof}
                    
                \end{proof}
            \begin{remark}[Associated complex-analytic topological spaces] \label{remark: associated_complex_analytic_topological_spaces}
                Let $X$ be a finite-type $\bbC$-scheme. Since $\bbC$ is algebraically closed, closed points of $X$ are in bijection with $X(\bbC)$ by Hilbert's \textit{Nullstellensatz}. At the same time, $X(\bbC)$ - by definition - consists of all solutions to the system of polynomials in $\bbC^n$ or $\P^{n, \an}_{\bbC}$ cut out by $X$ (say, $\dim X \leq n$), so $X(\bbC)$ naturally inherits the subspace topology from the complex-analytic space $\bbC^n$ or $\P^{n, \an}_{\bbC}$. In this sense, we have that:
                    $$X^{\an} := X(\bbC)$$
                is the natural complex-analytic topological space associated to the underlying topological space of $X$.

                The identification of closed points of $X$ gives rise to a natural continuous embedding:
                    $$i_X: X^{\an} \to X$$
            \end{remark}
            In order to obtain complex-analytic locally ringed spaces associated to locally finite-type $\bbC$-schemes, we need also to somehow complex-analytically complete the structure sheaves of these schemes. 
            \begin{proposition}[Complex-analytic completions of locally finite-type $\bbC$-schemes] \label{prop: complex_analytic_completions_of_locally_finite_type_C_schemes}
                There is an \textbf{analytification functor}:
                    $$(-)^{\an}: \Sch_{/\Spec \bbC}^{\lft} \to \An\Spc$$
                    $$(X, \scrO_X) \mapsto (X^{\an}, \scrO_{X^{\an}})$$
                from the category of locally finite-type $\bbC$-schemes to that of complex-analytic spaces, determined by:
                    $$\scrO_X^{\an} := (i_X^*\scrO_X)^{\an}$$
                with notations as in remark \ref{remark: associated_complex_analytic_topological_spaces}. Furthermore, if:
                    $$f: X \to Y$$
                is a morphism of locally finite-type $\bbC$-schemes then we will obtain a pullback square:
                    $$
                        \begin{tikzcd}
                    	{X^{\an}} & {Y^{\an}} \\
                    	X & Y
                    	\arrow["{f^{\an}}", from=1-1, to=1-2]
                    	\arrow["f", from=2-1, to=2-2]
                    	\arrow["{i_X}"', from=1-1, to=2-1]
                    	\arrow["{i_Y}", from=1-2, to=2-2]
                    	\arrow["\lrcorner"{anchor=center, pos=0.125}, draw=none, from=1-1, to=2-2]
                        \end{tikzcd}
                    $$
            \end{proposition}
                \begin{proof}
                    
                \end{proof}
            \begin{lemma}[(Faithful) flatness of complex-analytifications] \label{lemma: flatness_of_complex_analytifications}
                Let $A$ be an arbitrary finite-type $\bbC$-algebra. Then $A^{\an}$ will be flat over $A$. 
            \end{lemma}
                \begin{proof}
                    
                \end{proof}
            \begin{corollary} \label{coro: flatness_of_complex_analytifications}
                For any locally finite-type $\bbC$-scheme $X$, the pullback functor:
                    $$i_X^*: \scrO_X\mod \to \scrO_{X^{\an}}\mod$$
                is exact. 
            \end{corollary}
            
            \begin{remark}
                Consider a morphism:
                    $$f: X \to Y$$
                between schemes locally of finite type over $\Spec \bbC$. This gives rise to a pullback square of locally ringed spaces:
                    $$
                        \begin{tikzcd}
                    	{X^{\an}} & {Y^{\an}} \\
                    	X & Y
                    	\arrow["{f^{\an}}", from=1-1, to=1-2]
                    	\arrow["f", from=2-1, to=2-2]
                    	\arrow["{i_X}"', from=1-1, to=2-1]
                    	\arrow["{i_Y}", from=1-2, to=2-2]
                    	\arrow["\lrcorner"{anchor=center, pos=0.125}, draw=none, from=1-1, to=2-2]
                        \end{tikzcd}
                    $$
                as stated in proposition \ref{prop: complex_analytic_completions_of_locally_finite_type_C_schemes}, which in turn gives rise to the following cohomological base-change map/spectral sequence:
                    $$Li_Y^* \circ R f_* \Rightarrow R f^{\an}_* \circ Li_X^*$$
                but since the functors $i_X^*, i_Y^*$ are exact \textit{a priori} (cf. corollary \ref{coro: flatness_of_complex_analytifications}), this reduces down to a cohomological comparison map as follows:
                    $$i_Y^* \circ R f_* \Rightarrow R f^{\an}_* \circ i_X^*$$
                The existence of such a natural transformation allows us to formulate a version of proper base-change in this context, which yields us cohomological comparison isomorphisms that interpolate between the Zariski and complex-analytic topologies on locally finite-type $\bbC$-schemes and their associated analytic spaces respectively (see corollary \ref{coro: GAGA_cohomological_comparison}). 
            \end{remark}
            \begin{theorem}[Relative analytification of sheaves of modules] \label{theorem: relative_analytification_of_sheaves_of_modules}
                Suppose that:
                    $$f: X \to Y$$
                is a morphism between schemes locally of finite type over $\Spec \bbC$. If $f$ is proper, then the canonical natural transformation:
                    $$i_Y^* \circ R f_* \Rightarrow R f^{\an}_* \circ i_X^*$$
                will be a natural isomorphism of t-exact functors $\rmD^+(\scrO_X\mod) \to \rmD^+(\scrO_{Y^{\an}}\mod)$.
            \end{theorem}
                \begin{proof}
                    This is a result of proper base-change for general ringed spaces (see \cite[\href{https://stacks.math.columbia.edu/tag/09V4}{Tag 09V4}]{stacks} for now). 
                \end{proof}
            \begin{corollary}[GAGA cohomological comparison] \label{coro: GAGA_cohomological_comparison}
                When $Y \cong \Spec \bbC$ (and hence $i_Y^*$ is just the identity functor), there are isomorphisms of $\bbC$-vector spaces:
                    $$H^{\bullet}(X, \scrM) \cong H^{\bullet}(X^{\an}, i_X^*\scrM)$$
                In turn, this implies that the pullback functor:
                    $$i_X^*: \scrO_X\mod \to \scrO_{X^{\an}}\mod$$
                is fully faithful on top of being exact, and since both of its domain and codomain are abelian categories, this implies in particular that short exact sequences are reflected. 
            \end{corollary}

            We shall now see that when we restrict our attention to coherent modules only, the module analytification functor $i_X^*$ (for any locally finite-type $\bbC$-scheme $X$) will furthermore be essentially surjective, thus giving rise to an adjoint equivalence:
                $$i_X^*: \Coh(X) \leftrightarrows \Coh(X^{\an}): i_{X *}$$
            between the categories of coherent modules on $X$ and on $X^{\an}$, with the latter being given as the category of coherent modules on the ringed space $(X^{\an}, \scrO_{X^{\an}})$ (cf. \cite[\href{https://stacks.math.columbia.edu/tag/01BU}{Tag 01BU}]{stacks}). The fact that:
                $$i_X^*: \scrO_X\mod \to \scrO_{X^{\an}}\mod$$
            is fully faithful means that, should its restriction down to coherent modules:
               $$i_X^*: \Coh(X) \to \Coh(X^{\an})$$
           be well-defined (cf. lemma \ref{lemma: absolute_analytification_of_coherent_modules}) then we will be able to exploit the compact generation of $\Coh(X)$ to see that the set of (compact) generators via finite colimits of $\Coh(X^{\an})$ contains that of $\Coh(X)$ as a subset. The proof of essential surjectivity then reduces down to a proof of essential surjectivity on these generators. 
            \begin{remark}[A few reminders on coherent modules on ringed spaces]
                Again, we refer the reader to \cite[\href{https://stacks.math.columbia.edu/tag/01BU}{Tag 01BU}]{stacks} for a more detailed discussion, but the following list of properties is important enough for our purposes to warrant at least a mention. Suppose for a moment that $(X, \scrO_X)$ is a general ringed space and write $\Coh(X)$ to denote the category of coherent $\scrO_X$-modules.
                \begin{itemize}
                    \item $\Coh(X)$ is an abelian subcategory of $\scrO_X\mod$, and this is somewhat interesting, seeing how $\QCoh(X)$ is not generally even abelian for ringed spaces, unlike for schemes where quasi-coherent modules are extremely well-behaved (cf. theorem \ref{theorem: qcoh_homological_properties})
                    \item In fact, $\Coh(X)$ is closed under all extensions/short exacct sequences, and thus is a Serre subcategory of $\scrO_X\mod$ by definition.
                    \item $\Coh(X)$ has enough injectives, and said injective objects are flasque. 
                    \item An $\scrO_X$-module is coherent if and only if it is finitely presented.
                    \item If:
                        $$f: X \to Y$$
                    is a morphism between general ringed spaces and if $\scrN$ is a coherent $\scrO_Y$-module, then we will not usually be guaranteed that the pullback $f^*\scrN$ is coherent over $\scrO_X$. When $f$ is proper, though, and if $\scrM$ is some coherent $\scrO_X$-module, then the pushforward $f_*\scrM$ will in fact be coherent (cf. lemma \ref{lemma: pushforwards_of_analytic_coherent_modules}), and this is one of the reasons why having the cohomological base-change formula as in theorem \ref{theorem: relative_analytification_of_sheaves_of_modules} is important for our purposes!

                    That said, there is a well-defined pullback functor:
                        $$f^*: \scrO_Y\mod^{\ft} \to \scrO_X\mod^{\ft}$$
                    between the categories of finitely generated/finite-type $\scrO_Y$- and $\scrO_X$-modules. The issue mentioned above stems from the fact that the pullback of a finitely presented module is only finitely generated in general.
                \end{itemize}
                
                These properties will from now on be used without explicit mention.
            \end{remark}
            \begin{lemma}[Pushforwards of analytic coherent modules] \label{lemma: pushforwards_of_analytic_coherent_modules}
                Suppose that:
                    $$f: \calX \to \calY$$
                is a morphism of complex-analytic spaces. If $f$ is proper then there will be a well-defined t-exact functor:
                    $$Rf_*: \rmD^+(\Coh(\calX)) \to \rmD^+(\Coh(\calY))$$
            \end{lemma}
                \begin{proof}
                    
                \end{proof}
            \begin{lemma}[Absolute analytification of coherent modules] \label{lemma: absolute_analytification_of_coherent_modules}
                Let $X$ be a locally finite-type $\bbC$-scheme. Then, there is a well-defined functor:
                    $$i_X^*: \Coh(X) \to \Coh(X^{\an})$$
                (i.e. the pullback functor $i_X^*: \scrO_X\mod \to \scrO_{X^{\an}}\mod$ in particular does send coherent $\scrO_X$-modules to coherent $\scrO_{X^{\an}}$-modules).
            \end{lemma}
                \begin{proof}[Sketch]
                    $i_X^*$ is exact (cf. corollary \ref{coro: GAGA_cohomological_comparison}), so it preserves compactness of objects. 
                \end{proof}
            \begin{theorem}[Relative analytification of coherent modules] \label{theorem: relative_analytification_of_coherent_modules}
                Suppose that:
                    $$f: X \to Y$$
                is a morphism between schemes locally of finite type over $\Spec \bbC$. If $f$ is proper, then the canonical natural transformation:
                    $$i_Y^* \circ R f_* \Rightarrow R f^{\an}_* \circ i_X^*$$
                will be a natural isomorphism of t-exact functors $\rmD^+(\Coh(X)) \to \rmD^+(\Coh(Y^{\an}))$.

                When $Y \cong \Spec \bbC$, the above implies that:
                    $$H^{\bullet}(X, \scrM) \cong H^{\bullet}(X^{\an}, i_X^*\scrM)$$
                are finite-dimensional $\bbC$-vector spaces for any coherent $\scrO_X$-module $\scrM$.
            \end{theorem}
            \begin{remark}[\textit{D\'evissage}]
                Let us recall Chow's Lemma, which says that should $S$ be a Noetherian base scheme and $\pi: X \to S$ be a proper $S$-scheme, then there will exist a projective $S$-scheme $\pi': X' \to S$ along with a morphism of $S$-schemes:
                    $$f: X' \to X$$
                for which there is a \textit{dense} open subscheme $U \subset X$ such that:
                    $$X' \x_{f, X} U \cong U$$
                If, in addition, both $X$ and $X'$ are irreducible then $f$ will be birational. Furthermore, if $X$ is reduced, irreducible, or integral, then the same can be assumed for $X'$; in particular, this means that if $X$ is a variety (i.e. when all those adjectives are satisfied and $S$ is the spectrum of a field) then we can assume without loss of generality that $X'$ too is a variety over the same field.

                Using Chow's Lemma, and letting $S := \Spec \bbC$, we see that any proper algebraic $\bbC$-variety $Y$ is birationally equivalent to a projective $\bbC$-variety $X$, for which there is some closed immersion:
                    $$j_X: X \hookrightarrow \P^n_{\bbC}$$
                We know that the abelian category:
                    $$\Coh(\P^n_{\bbC})$$
                is generated via finite colimits by finitely many (compact) objects (namely Serre's twisting line bundles)\footnote{This statement remains true when we replace $\bbC$ with an arbitrary commutative ring.}, 
            \end{remark}

            \begin{theorem}[Compact generation of coherent modules over analytifications] \label{theorem: compact_generation_of_coherent_modules_over_analytifications}
                For any locally finite-type proper $\bbC$-scheme $X$, any coherent $\scrO_{X^{\an}}$-module $\scrM$ admits a 
            \end{theorem}
                \begin{proof}
                    
                \end{proof}
            The following is a corollary to a combination of corollary \ref{coro: GAGA_cohomological_comparison} and theorem \ref{theorem: compact_generation_of_coherent_modules_over_analytifications}.
            \begin{corollary}[Serre's GAGA]
                For any locally finite-type proper $\bbC$-scheme $X$, there is an adjoint equivalence of categories:
                    $$i_X^*: \Coh(X) \leftrightarrows \Coh(X^{\an}): i_{X *}$$
            \end{corollary}
        
    \addcontentsline{toc}{section}{References}
    \printbibliography

\end{document}