\input{article preambles}

\setcounter{section}{-1}

\input{commands}

\begin{document}

    \title{Perverse sheaves and intersection (co)homology}
    
    \author{Dat Minh Ha}
    \maketitle
    
    \begin{abstract}
        These are notes on perverse $\ell$-adic \'etale sheaves as cohomological devices for dealing with singular varieties. 
    \end{abstract}
    
    {
      \hypersetup{} 
      %\dominitoc
      \tableofcontents %sort sections alphabetically
    }

    \section{Introduction}
        Suppose that:
            $$\pi: Y \to X$$
        is a surjection of smooth varieties. Using Poincar\'e Duality, what one can show is that such a map induces injections on \'etale cohomologies, for all $i \geq 0$:
            $$H^i\pi^*: H^i(X_{\et}, \bar{\Q}_{\ell}) \hookrightarrow H^i(Y_{\et}, \bar{\Q}_{\ell})$$
        This phenomenon, however fails generally should $X$ is not smooth; e.g. if $\pi: Y \to X$ is a resolution of singularities. The idea with perverse (constructible) $\ell$-adic sheaves is that, a similar result can be recovered by switching from the standard t-structures on $\rmD^b_c(X_{\et}, \bar{\Q}_{\ell})$ and on $\rmD^b_c(Y_{\et}, \bar{\Q}_{\ell})$ to the so-called \say{perverse} t-structures; this amounts to performing very particular cohomological shifts. 
    
    \addcontentsline{toc}{section}{References}
    \printbibliography

\end{document}