\documentclass[a4paper, 11pt]{article}

%\usepackage[center]{titlesec}

\usepackage{amsfonts, amssymb, amsmath, amsthm, amsxtra}

\usepackage{foekfont}

\usepackage{MnSymbol}

\usepackage{pdfrender, xcolor}
%\pdfrender{StrokeColor=black,LineWidth=.4pt,TextRenderingMode=2}

%\usepackage{minitoc}
%\setcounter{section}{-1}
%\setcounter{tocdepth}{}
%\setcounter{minitocdepth}{}
%\setcounter{secnumdepth}{}

\usepackage{graphicx}

\usepackage[english]{babel}
\usepackage[utf8]{inputenc}
%\usepackage{mathpazo}
%\usepackage{euler}
\usepackage{eucal}
\usepackage{bbm}
\usepackage{bm}
\usepackage{csquotes}
\usepackage[nottoc]{tocbibind}
\usepackage{appendix}
\usepackage{float}
\usepackage[T1]{fontenc}
\usepackage[
    left = \flqq{},% 
    right = \frqq{},% 
    leftsub = \flq{},% 
    rightsub = \frq{} %
]{dirtytalk}

\usepackage{imakeidx}
\makeindex

%\usepackage[dvipsnames]{xcolor}
\usepackage{hyperref}
    \hypersetup{
        colorlinks=true,
        linkcolor=teal,
        filecolor=pink,      
        urlcolor=teal,
        citecolor=magenta
    }
\usepackage{comment}

% You would set the PDF title, author, etc. with package options or
% \hypersetup.

\usepackage[backend=biber, style=alphabetic, sorting=nty]{biblatex}
    \addbibresource{bibliography.bib}

\raggedbottom

\usepackage{mathrsfs}
\usepackage{mathtools} 
\mathtoolsset{showonlyrefs} 
%\usepackage{amsthm}
\renewcommand\qedsymbol{$\blacksquare$}

\usepackage{tikz-cd}
\tikzcdset{scale cd/.style={every label/.append style={scale=#1},
    cells={nodes={scale=#1}}}}
\usepackage{tikz}
\usepackage{setspace}
\usepackage[version=3]{mhchem}
\parskip=0.1in
\usepackage[margin=25mm]{geometry}

\usepackage{listings, lstautogobble}
\lstset{
	language=matlab,
	basicstyle=\scriptsize\ttfamily,
	commentstyle=\ttfamily\itshape\color{gray},
	stringstyle=\ttfamily,
	showstringspaces=false,
	breaklines=true,
	frameround=ffff,
	frame=single,
	rulecolor=\color{black},
	autogobble=true
}

\usepackage{todonotes,tocloft,xpatch,hyperref}

% This is based on classicthesis chapter definition
\let\oldsec=\section
\renewcommand*{\section}{\secdef{\Sec}{\SecS}}
\newcommand\SecS[1]{\oldsec*{#1}}%
\newcommand\Sec[2][]{\oldsec[\texorpdfstring{#1}{#1}]{#2}}%

\newcounter{istodo}[section]

% http://tex.stackexchange.com/a/61267/11984
\makeatletter
%\xapptocmd{\Sec}{\addtocontents{tdo}{\protect\todoline{\thesection}{#1}{}}}{}{}
\newcommand{\todoline}[1]{\@ifnextchar\Endoftdo{}{\@todoline{#1}}}
\newcommand{\@todoline}[3]{%
	\@ifnextchar\todoline{}
	{\contentsline{section}{\numberline{#1}#2}{#3}{}{}}%
}
\let\l@todo\l@subsection
\newcommand{\Endoftdo}{}

\AtEndDocument{\addtocontents{tdo}{\string\Endoftdo}}
\makeatother

\usepackage{lipsum}

%   Reduce the margin of the summary:
\def\changemargin#1#2{\list{}{\rightmargin#2\leftmargin#1}\item[]}
\let\endchangemargin=\endlist 

%   Generate the environment for the abstract:
%\newcommand\summaryname{Abstract}
%\newenvironment{abstract}%
    %{\small\begin{center}%
    %\bfseries{\summaryname} \end{center}}

\newtheorem{theorem}{Theorem}[section]
    \numberwithin{theorem}{subsection}
\newtheorem{proposition}{Proposition}[section]
    \numberwithin{proposition}{subsection}
\newtheorem{lemma}{Lemma}[section]
    \numberwithin{lemma}{subsection}
\newtheorem{claim}{Claim}[section]
    \numberwithin{claim}{subsection}
\newtheorem{question}{Question}[section]
    \numberwithin{question}{section}

\theoremstyle{definition}
    \newtheorem{definition}{Definition}[section]
        \numberwithin{definition}{subsection}

\theoremstyle{remark}
    \newtheorem{remark}{Remark}[section]
        \numberwithin{remark}{subsection}
    \newtheorem{example}{Example}[section]
        \numberwithin{example}{subsection}    
    \newtheorem{convention}{Convention}[section]
        \numberwithin{convention}{subsection}
    \newtheorem{corollary}{Corollary}[section]
        \numberwithin{corollary}{subsection}

\setcounter{section}{-1}

\renewcommand{\implies}{\Rightarrow}
\renewcommand{\cong}{\simeq}
\newcommand{\ladjoint}{\dashv}
\newcommand{\radjoint}{\vdash}
\newcommand{\<}{\langle}
\renewcommand{\>}{\rangle}
\newcommand{\ndiv}{\hspace{-2pt}\not|\hspace{5pt}}
\newcommand{\cond}{\blacktriangle}
\newcommand{\decond}{\triangle}
\newcommand{\solid}{\blacksquare}
\newcommand{\ot}{\leftarrow}
\renewcommand{\-}{\text{-}}
\renewcommand{\mapsto}{\leadsto}
\renewcommand{\leq}{\leqslant}
\renewcommand{\geq}{\geqslant}
\renewcommand{\setminus}{\smallsetminus}
\newcommand{\punc}{\overset{\circ}}
\renewcommand{\div}{\operatorname{div}}
\newcommand{\grad}{\operatorname{grad}}
\newcommand{\curl}{\operatorname{curl}}
\makeatletter
\DeclareRobustCommand{\cev}[1]{%
  {\mathpalette\do@cev{#1}}%
}
\newcommand{\do@cev}[2]{%
  \vbox{\offinterlineskip
    \sbox\z@{$\m@th#1 x$}%
    \ialign{##\cr
      \hidewidth\reflectbox{$\m@th#1\vec{}\mkern4mu$}\hidewidth\cr
      \noalign{\kern-\ht\z@}
      $\m@th#1#2$\cr
    }%
  }%
}
\makeatother

\newcommand{\N}{\mathbb{N}}
\newcommand{\Z}{\mathbb{Z}}
\newcommand{\Q}{\mathbb{Q}}
\newcommand{\R}{\mathbb{R}}
\newcommand{\bbC}{\mathbb{C}}
\NewDocumentCommand{\x}{e{_^}}{%
  \mathbin{\mathop{\times}\displaylimits
    \IfValueT{#1}{_{#1}}
    \IfValueT{#2}{^{#2}}
  }%
}
\NewDocumentCommand{\pushout}{e{_^}}{%
  \mathbin{\mathop{\sqcup}\displaylimits
    \IfValueT{#1}{_{#1}}
    \IfValueT{#2}{^{#2}}
  }%
}
\newcommand{\supp}{\operatorname{supp}}
\newcommand{\im}{\operatorname{im}}
\newcommand{\coim}{\operatorname{coim}}
\newcommand{\coker}{\operatorname{coker}}
\newcommand{\id}{\mathrm{id}}
\newcommand{\chara}{\operatorname{char}}
\newcommand{\trdeg}{\operatorname{trdeg}}
\newcommand{\rank}{\operatorname{rank}}
\newcommand{\trace}{\operatorname{tr}}
\newcommand{\length}{\operatorname{length}}
\newcommand{\height}{\operatorname{ht}}
\renewcommand{\span}{\operatorname{span}}
\newcommand{\e}{\epsilon}
\newcommand{\p}{\mathfrak{p}}
\newcommand{\q}{\mathfrak{q}}
\newcommand{\m}{\mathfrak{m}}
\newcommand{\n}{\mathfrak{n}}
\newcommand{\calF}{\mathcal{F}}
\newcommand{\calG}{\mathcal{G}}
\newcommand{\calO}{\mathcal{O}}
\newcommand{\F}{\mathbb{F}}
\DeclareMathOperator{\lcm}{lcm}
\newcommand{\gr}{\operatorname{gr}}
\newcommand{\vol}{\mathrm{vol}}
\newcommand{\ord}{\operatorname{ord}}
\newcommand{\projdim}{\operatorname{proj.dim}}
\newcommand{\injdim}{\operatorname{inj.dim}}
\newcommand{\flatdim}{\operatorname{flat.dim}}
\newcommand{\globdim}{\operatorname{glob.dim}}
\renewcommand{\Re}{\operatorname{Re}}
\renewcommand{\Im}{\operatorname{Im}}
\newcommand{\sgn}{\operatorname{sgn}}
\newcommand{\coad}{\operatorname{coad}}
\newcommand{\ch}{\operatorname{ch}} %characters of representations

\newcommand{\Ad}{\mathrm{Ad}}
\newcommand{\GL}{\mathrm{GL}}
\newcommand{\SL}{\mathrm{SL}}
\newcommand{\PGL}{\mathrm{PGL}}
\newcommand{\PSL}{\mathrm{PSL}}
\newcommand{\Sp}{\mathrm{Sp}}
\newcommand{\GSp}{\mathrm{GSp}}
\newcommand{\GSpin}{\mathrm{GSpin}}
\newcommand{\rmO}{\mathrm{O}}
\newcommand{\SO}{\mathrm{SO}}
\newcommand{\SU}{\mathrm{SU}}
\newcommand{\rmU}{\mathrm{U}}
\newcommand{\rmY}{\mathrm{Y}}
\newcommand{\rmu}{\mathrm{u}}
\newcommand{\rmV}{\mathrm{V}}
\newcommand{\gl}{\mathfrak{gl}}
\renewcommand{\sl}{\mathfrak{sl}}
\newcommand{\diag}{\mathfrak{diag}}
\newcommand{\pgl}{\mathfrak{pgl}}
\newcommand{\psl}{\mathfrak{psl}}
\newcommand{\fraksp}{\mathfrak{sp}}
\newcommand{\gsp}{\mathfrak{gsp}}
\newcommand{\gspin}{\mathfrak{gspin}}
\newcommand{\frako}{\mathfrak{o}}
\newcommand{\so}{\mathfrak{so}}
\newcommand{\su}{\mathfrak{su}}
%\newcommand{\fraku}{\mathfrak{u}}
\newcommand{\Spec}{\operatorname{Spec}}
\newcommand{\Spf}{\operatorname{Spf}}
\newcommand{\Spm}{\operatorname{Spm}}
\newcommand{\Spv}{\operatorname{Spv}}
\newcommand{\Spa}{\operatorname{Spa}}
\newcommand{\Spd}{\operatorname{Spd}}
\newcommand{\Proj}{\operatorname{Proj}}
\newcommand{\Gr}{\mathrm{Gr}}
\newcommand{\Hecke}{\mathrm{Hecke}}
\newcommand{\Sht}{\mathrm{Sht}}
\newcommand{\Quot}{\mathrm{Quot}}
\newcommand{\Hilb}{\mathrm{Hilb}}
\newcommand{\Pic}{\mathrm{Pic}}
\newcommand{\Div}{\mathrm{Div}}
\newcommand{\Jac}{\mathrm{Jac}}
\newcommand{\Alb}{\mathrm{Alb}} %albanese variety
\newcommand{\Bun}{\mathrm{Bun}}
\newcommand{\loopspace}{\mathbf{\Omega}}
\newcommand{\suspension}{\mathbf{\Sigma}}
\newcommand{\tangent}{\mathrm{T}} %tangent space
\newcommand{\Eig}{\mathrm{Eig}}
\newcommand{\Cox}{\mathrm{Cox}} %coxeter functors
\newcommand{\rmK}{\mathrm{K}} %Killing form
\newcommand{\km}{\mathfrak{km}} %kac-moody algebras
\newcommand{\Dyn}{\mathrm{Dyn}} %associated Dynkin quivers
\newcommand{\Car}{\mathrm{Car}} %cartan matrices of finite quivers
\newcommand{\uce}{\mathfrak{uce}} %universal central extension of lie algebras

\newcommand{\Ring}{\mathrm{Ring}}
\newcommand{\Cring}{\mathrm{CRing}}
\newcommand{\Alg}{\mathrm{Alg}}
\newcommand{\Leib}{\mathrm{Leib}} %leibniz algebras
\newcommand{\Fld}{\mathrm{Fld}}
\newcommand{\Sets}{\mathrm{Sets}}
\newcommand{\Equiv}{\mathrm{Equiv}} %equivalence relations
\newcommand{\Cat}{\mathrm{Cat}}
\newcommand{\Grp}{\mathrm{Grp}}
\newcommand{\Ab}{\mathrm{Ab}}
\newcommand{\Sch}{\mathrm{Sch}}
\newcommand{\Coh}{\mathrm{Coh}}
\newcommand{\QCoh}{\mathrm{QCoh}}
\newcommand{\Perf}{\mathrm{Perf}} %perfect complexes
\newcommand{\Sing}{\mathrm{Sing}} %singularity categories
\newcommand{\Desc}{\mathrm{Desc}}
\newcommand{\Sh}{\mathrm{Sh}}
\newcommand{\Psh}{\mathrm{PSh}}
\newcommand{\Fib}{\mathrm{Fib}}
\renewcommand{\mod}{\-\mathrm{mod}}
\newcommand{\comod}{\-\mathrm{comod}}
\newcommand{\bimod}{\-\mathrm{bimod}}
\newcommand{\Vect}{\mathrm{Vect}}
\newcommand{\Rep}{\mathrm{Rep}}
\newcommand{\Grpd}{\mathrm{Grpd}}
\newcommand{\Arr}{\mathrm{Arr}}
\newcommand{\Esp}{\mathrm{Esp}}
\newcommand{\Ob}{\mathrm{Ob}}
\newcommand{\Mor}{\mathrm{Mor}}
\newcommand{\Mfd}{\mathrm{Mfd}}
\newcommand{\Riem}{\mathrm{Riem}}
\newcommand{\RS}{\mathrm{RS}}
\newcommand{\LRS}{\mathrm{LRS}}
\newcommand{\TRS}{\mathrm{TRS}}
\newcommand{\TLRS}{\mathrm{TLRS}}
\newcommand{\LVRS}{\mathrm{LVRS}}
\newcommand{\LBRS}{\mathrm{LBRS}}
\newcommand{\Spc}{\mathrm{Spc}}
\newcommand{\Top}{\mathrm{Top}}
\newcommand{\Topos}{\mathrm{Topos}}
\newcommand{\Nil}{\mathfrak{nil}}
\newcommand{\J}{\mathfrak{J}}
\newcommand{\Stk}{\mathrm{Stk}}
\newcommand{\Pre}{\mathrm{Pre}}
\newcommand{\simp}{\mathbf{\Delta}}
\newcommand{\Res}{\mathrm{Res}}
\newcommand{\Ind}{\mathrm{Ind}}
\newcommand{\Pro}{\mathrm{Pro}}
\newcommand{\Mon}{\mathrm{Mon}}
\newcommand{\Comm}{\mathrm{Comm}}
\newcommand{\Fin}{\mathrm{Fin}}
\newcommand{\Assoc}{\mathrm{Assoc}}
\newcommand{\Semi}{\mathrm{Semi}}
\newcommand{\Co}{\mathrm{Co}}
\newcommand{\Loc}{\mathrm{Loc}}
\newcommand{\Ringed}{\mathrm{Ringed}}
\newcommand{\Haus}{\mathrm{Haus}} %hausdorff spaces
\newcommand{\Comp}{\mathrm{Comp}} %compact hausdorff spaces
\newcommand{\Stone}{\mathrm{Stone}} %stone spaces
\newcommand{\Extr}{\mathrm{Extr}} %extremely disconnected spaces
\newcommand{\Ouv}{\mathrm{Ouv}}
\newcommand{\Str}{\mathrm{Str}}
\newcommand{\Func}{\mathrm{Func}}
\newcommand{\Crys}{\mathrm{Crys}}
\newcommand{\LocSys}{\mathrm{LocSys}}
\newcommand{\Sieves}{\mathrm{Sieves}}
\newcommand{\pt}{\mathrm{pt}}
\newcommand{\Graphs}{\mathrm{Graphs}}
\newcommand{\Lie}{\mathrm{Lie}}
\newcommand{\Env}{\mathrm{Env}}
\newcommand{\Ho}{\mathrm{Ho}}
\newcommand{\rmD}{\mathrm{D}}
\newcommand{\Cov}{\mathrm{Cov}}
\newcommand{\Frames}{\mathrm{Frames}}
\newcommand{\Locales}{\mathrm{Locales}}
\newcommand{\Span}{\mathrm{Span}}
\newcommand{\Corr}{\mathrm{Corr}}
\newcommand{\Monad}{\mathrm{Monad}}
\newcommand{\Var}{\mathrm{Var}}
\newcommand{\sfN}{\mathrm{N}} %nerve
\newcommand{\Diam}{\mathrm{Diam}} %diamonds
\newcommand{\co}{\mathrm{co}}
\newcommand{\ev}{\mathrm{ev}}
\newcommand{\bi}{\mathrm{bi}}
\newcommand{\Nat}{\mathrm{Nat}}
\newcommand{\Hopf}{\mathrm{Hopf}}
\newcommand{\Dmod}{\mathrm{D}\mod}
\newcommand{\Perv}{\mathrm{Perv}}
\newcommand{\Sph}{\mathrm{Sph}}
\newcommand{\Moduli}{\mathrm{Moduli}}
\newcommand{\Pseudo}{\mathrm{Pseudo}}
\newcommand{\Lax}{\mathrm{Lax}}
\newcommand{\Strict}{\mathrm{Strict}}
\newcommand{\Opd}{\mathrm{Opd}} %operads
\newcommand{\Shv}{\mathrm{Shv}}
\newcommand{\Char}{\mathrm{Char}} %CharShv = character sheaves
\newcommand{\Huber}{\mathrm{Huber}}
\newcommand{\Tate}{\mathrm{Tate}}
\newcommand{\Affd}{\mathrm{Affd}} %affinoid algebras
\newcommand{\Adic}{\mathrm{Adic}} %adic spaces
\newcommand{\Rig}{\mathrm{Rig}}
\newcommand{\An}{\mathrm{An}}
\newcommand{\Perfd}{\mathrm{Perfd}} %perfectoid spaces
\newcommand{\Sub}{\mathrm{Sub}} %subobjects
\newcommand{\Ideals}{\mathrm{Ideals}}
\newcommand{\Isoc}{\mathrm{Isoc}} %isocrystals
\newcommand{\Ban}{\-\mathrm{Ban}} %Banach spaces
\newcommand{\Fre}{\-\mathrm{Fr\acute{e}}} %Frechet spaces
\newcommand{\Ch}{\mathrm{Ch}} %chain complexes
\newcommand{\Pure}{\mathrm{Pure}}
\newcommand{\Mixed}{\mathrm{Mixed}}
\newcommand{\Hodge}{\mathrm{Hodge}} %Hodge structures
\newcommand{\Mot}{\mathrm{Mot}} %motives
\newcommand{\KL}{\mathrm{KL}} %category of Kazhdan-Lusztig modules
\newcommand{\Pres}{\mathrm{Pres}} %presentable categories
\newcommand{\Noohi}{\mathrm{Noohi}} %category of Noohi groups
\newcommand{\Inf}{\mathrm{Inf}}
\newcommand{\LPar}{\mathrm{LPar}} %Langlands parameters
\newcommand{\ORig}{\mathrm{ORig}} %overconvergent sites
\newcommand{\Quiv}{\mathrm{Quiv}} %quivers
\newcommand{\Def}{\mathrm{Def}} %deformation functors
\newcommand{\Root}{\mathrm{Root}}
\newcommand{\gRep}{\mathrm{gRep}}
\newcommand{\Higgs}{\mathrm{Higgs}}
\newcommand{\BGG}{\mathrm{BGG}}
\newcommand{\Poiss}{\mathrm{Poiss}}

\newcommand{\Aut}{\mathrm{Aut}}
\newcommand{\Inn}{\mathrm{Inn}}
\newcommand{\Out}{\mathrm{Out}}
\newcommand{\der}{\mathfrak{der}} %derivations on Lie algebras
\newcommand{\frakend}{\mathfrak{end}}
\newcommand{\aut}{\mathfrak{aut}}
\newcommand{\inn}{\mathfrak{inn}} %inner derivations
\newcommand{\out}{\mathfrak{out}} %outer derivations
\newcommand{\Stab}{\mathrm{Stab}}
\newcommand{\Cent}{\mathrm{Cent}}
\newcommand{\Norm}{\mathrm{Norm}}
\newcommand{\stab}{\mathfrak{stab}}
\newcommand{\cent}{\mathfrak{cent}}
\newcommand{\norm}{\mathfrak{norm}}
\newcommand{\Rad}{\operatorname{Rad}}
\newcommand{\Transporter}{\mathrm{Transp}} %transporter between two subsets of a group
\newcommand{\Conj}{\mathrm{Conj}}
\newcommand{\Diag}{\mathrm{Diag}}
\newcommand{\Gal}{\mathrm{Gal}}
\newcommand{\bfG}{\mathbf{G}} %absolute Galois group
\newcommand{\Frac}{\mathrm{Frac}}
\newcommand{\Ann}{\mathrm{Ann}}
\newcommand{\Val}{\mathrm{Val}}
\newcommand{\Chow}{\mathrm{Chow}}
\newcommand{\Sym}{\mathrm{Sym}}
\newcommand{\End}{\mathrm{End}}
\newcommand{\Mat}{\mathrm{Mat}}
\newcommand{\Diff}{\mathrm{Diff}}
\newcommand{\Autom}{\mathrm{Autom}}
\newcommand{\Artin}{\mathrm{Artin}} %artin maps
\newcommand{\sk}{\mathrm{sk}} %skeleton of a category
\newcommand{\eqv}{\mathrm{eqv}} %functor that maps groups $G$ to $G$-sets
\newcommand{\Inert}{\mathrm{Inert}}
\newcommand{\Fil}{\mathrm{Fil}}
\newcommand{\Prim}{\mathfrak{Prim}}
\newcommand{\Nerve}{\mathrm{N}}
\newcommand{\Hol}{\mathrm{Hol}} %holomorphic functions %holonomy groups
\newcommand{\Bi}{\mathrm{Bi}} %Bi for biholomorphic functions
\newcommand{\chev}{\mathfrak{chev}} %chevalley relations
\newcommand{\bfLie}{\mathbf{Lie}} %non-reduced lie algebra associated to generalised cartan matrices
\newcommand{\frakLie}{\mathfrak{Lie}} %reduced lie algebra associated to generalised cartan matrices
\newcommand{\frakChev}{\mathfrak{Chev}} 
\newcommand{\Rees}{\operatorname{Rees}}
\newcommand{\Dr}{\mathrm{Dr}} %Drinfeld's quantum double 
\newcommand{\frakDr}{\mathfrak{Dr}} %classical double of lie bialgebras

\renewcommand{\projlim}{\varprojlim}
\newcommand{\indlim}{\varinjlim}
\newcommand{\colim}{\operatorname{colim}}
\renewcommand{\lim}{\operatorname{lim}}
\newcommand{\toto}{\rightrightarrows}
%\newcommand{\tensor}{\otimes}
\NewDocumentCommand{\tensor}{e{_^}}{%
  \mathbin{\mathop{\otimes}\displaylimits
    \IfValueT{#1}{_{#1}}
    \IfValueT{#2}{^{#2}}
  }%
}
\NewDocumentCommand{\singtensor}{e{_^}}{%
  \mathbin{\mathop{\odot}\displaylimits
    \IfValueT{#1}{_{#1}}
    \IfValueT{#2}{^{#2}}
  }%
}
\NewDocumentCommand{\hattensor}{e{_^}}{%
  \mathbin{\mathop{\hat{\otimes}}\displaylimits
    \IfValueT{#1}{_{#1}}
    \IfValueT{#2}{^{#2}}
  }%
}
\NewDocumentCommand{\semidirect}{e{_^}}{%
  \mathbin{\mathop{\rtimes}\displaylimits
    \IfValueT{#1}{_{#1}}
    \IfValueT{#2}{^{#2}}
  }%
}
\newcommand{\eq}{\operatorname{eq}}
\newcommand{\coeq}{\operatorname{coeq}}
\newcommand{\Hom}{\mathrm{Hom}}
\newcommand{\Maps}{\mathrm{Maps}}
\newcommand{\Tor}{\mathrm{Tor}}
\newcommand{\Ext}{\mathrm{Ext}}
\newcommand{\Isom}{\mathrm{Isom}}
\newcommand{\stalk}{\mathrm{stalk}}
\newcommand{\RKE}{\operatorname{RKE}}
\newcommand{\LKE}{\operatorname{LKE}}
\newcommand{\oblv}{\mathrm{oblv}}
\newcommand{\const}{\mathrm{const}}
\newcommand{\free}{\mathrm{free}}
\newcommand{\adrep}{\mathrm{ad}} %adjoint representation
\newcommand{\NL}{\mathbb{NL}} %naive cotangent complex
\newcommand{\pr}{\operatorname{pr}}
\newcommand{\Der}{\mathrm{Der}}
\newcommand{\Frob}{\mathrm{Fr}} %Frobenius
\newcommand{\frob}{\mathrm{f}} %trace of Frobenius
\newcommand{\bfpt}{\mathbf{pt}}
\newcommand{\bfloc}{\mathbf{loc}}
\DeclareMathAlphabet{\mymathbb}{U}{BOONDOX-ds}{m}{n}
\newcommand{\0}{\mymathbb{0}}
\newcommand{\1}{\mathbbm{1}}
\newcommand{\2}{\mathbbm{2}}
\newcommand{\Jet}{\mathrm{Jet}}
\newcommand{\Split}{\mathrm{Split}}
\newcommand{\Sq}{\mathrm{Sq}}
\newcommand{\Zero}{\mathrm{Z}}
\newcommand{\SqZ}{\Sq\Zero}
\newcommand{\lie}{\mathfrak{lie}}
\newcommand{\y}{\mathrm{y}} %yoneda
\newcommand{\Sm}{\mathrm{Sm}}
\newcommand{\AJ}{\phi} %abel-jacobi map
\newcommand{\act}{\mathrm{act}}
\newcommand{\ram}{\mathrm{ram}} %ramification index
\newcommand{\inv}{\mathrm{inv}}
\newcommand{\Spr}{\mathrm{Spr}} %the Springer map/sheaf
\newcommand{\Refl}{\mathrm{Refl}} %reflection functor]
\newcommand{\HH}{\mathrm{HH}} %Hochschild (co)homology
\newcommand{\Poinc}{\mathrm{Poinc}}
\newcommand{\Simpson}{\mathrm{Simpson}}

\newcommand{\bbU}{\mathbb{U}}
\newcommand{\V}{\mathbb{V}}
\newcommand{\W}{\mathbb{W}}
\newcommand{\calU}{\mathcal{U}}
\newcommand{\calW}{\mathcal{W}}
\newcommand{\rmI}{\mathrm{I}} %augmentation ideal
\newcommand{\bfV}{\mathbf{V}}
\newcommand{\C}{\mathcal{C}}
\newcommand{\D}{\mathcal{D}}
\newcommand{\T}{\mathscr{T}} %Tate modules
\newcommand{\calM}{\mathcal{M}}
\newcommand{\calN}{\mathcal{N}}
\newcommand{\calP}{\mathcal{P}}
\newcommand{\calQ}{\mathcal{Q}}
\newcommand{\A}{\mathbb{A}}
\renewcommand{\P}{\mathbb{P}}
\newcommand{\calL}{\mathcal{L}}
\newcommand{\scrL}{\mathscr{L}}
\newcommand{\E}{\mathcal{E}}
\renewcommand{\H}{\mathbf{H}}
\newcommand{\scrS}{\mathscr{S}}
\newcommand{\calX}{\mathcal{X}}
\newcommand{\calY}{\mathcal{Y}}
\newcommand{\calZ}{\mathcal{Z}}
\newcommand{\calS}{\mathcal{S}}
\newcommand{\calR}{\mathcal{R}}
\newcommand{\scrX}{\mathscr{X}}
\newcommand{\scrY}{\mathscr{Y}}
\newcommand{\scrZ}{\mathscr{Z}}
\newcommand{\calA}{\mathcal{A}}
\newcommand{\calB}{\mathcal{B}}
\renewcommand{\S}{\mathcal{S}}
\newcommand{\B}{\mathbb{B}}
\newcommand{\bbD}{\mathbb{D}}
\newcommand{\G}{\mathbb{G}}
\newcommand{\horn}{\mathbf{\Lambda}}
\renewcommand{\L}{\mathbb{L}}
\renewcommand{\a}{\mathfrak{a}}
\renewcommand{\b}{\mathfrak{b}}
\renewcommand{\c}{\mathfrak{c}}
\renewcommand{\d}{\mathfrak{d}}
\renewcommand{\t}{\mathfrak{t}}
\renewcommand{\r}{\mathfrak{r}}
\newcommand{\fraku}{\mathfrak{u}}
\newcommand{\frakv}{\mathfrak{v}}
\newcommand{\frake}{\mathfrak{e}}
\newcommand{\bbX}{\mathbb{X}}
\newcommand{\frakw}{\mathfrak{w}}
\newcommand{\frakG}{\mathfrak{G}}
\newcommand{\frakH}{\mathfrak{H}}
\newcommand{\frakE}{\mathfrak{E}}
\newcommand{\frakF}{\mathfrak{F}}
\newcommand{\g}{\mathfrak{g}}
\newcommand{\h}{\mathfrak{h}}
\renewcommand{\k}{\mathfrak{k}}
\newcommand{\z}{\mathfrak{z}}
\newcommand{\fraki}{\mathfrak{i}}
\newcommand{\frakj}{\mathfrak{j}}
\newcommand{\del}{\partial}
\newcommand{\bbE}{\mathbb{E}}
\newcommand{\scrO}{\mathscr{O}}
\newcommand{\bbO}{\mathbb{O}}
\newcommand{\scrA}{\mathscr{A}}
\newcommand{\scrB}{\mathscr{B}}
\newcommand{\scrE}{\mathscr{E}}
\newcommand{\scrF}{\mathscr{F}}
\newcommand{\scrG}{\mathscr{G}}
\newcommand{\scrM}{\mathscr{M}}
\newcommand{\scrN}{\mathscr{N}}
\newcommand{\scrP}{\mathscr{P}}
\newcommand{\frakS}{\mathfrak{S}}
\newcommand{\frakT}{\mathfrak{T}}
\newcommand{\calI}{\mathcal{I}}
\newcommand{\calJ}{\mathcal{J}}
\newcommand{\scrI}{\mathscr{I}}
\newcommand{\scrJ}{\mathscr{J}}
\newcommand{\scrK}{\mathscr{K}}
\newcommand{\calK}{\mathcal{K}}
\newcommand{\scrV}{\mathscr{V}}
\newcommand{\scrW}{\mathscr{W}}
\newcommand{\bbS}{\mathbb{S}}
\newcommand{\scrH}{\mathscr{H}}
\newcommand{\bfA}{\mathbf{A}}
\newcommand{\bfB}{\mathbf{B}}
\newcommand{\bfC}{\mathbf{C}}
\renewcommand{\O}{\mathbb{O}}
\newcommand{\calV}{\mathcal{V}}
\newcommand{\scrR}{\mathscr{R}} %radical
\newcommand{\sfR}{\mathsf{R}} %quantum R-matrices
\newcommand{\sfr}{\mathsf{r}} %classical R-matrices
\newcommand{\rmZ}{\mathrm{Z}} %centre of algebra
\newcommand{\rmC}{\mathrm{C}} %centralisers in algebras
\newcommand{\bfGamma}{\mathbf{\Gamma}}
\newcommand{\scrU}{\mathscr{U}}
\newcommand{\rmW}{\mathrm{W}} %Weil group
\newcommand{\frakM}{\mathfrak{M}}
\newcommand{\frakN}{\mathfrak{N}}
\newcommand{\frakB}{\mathfrak{B}}
\newcommand{\frakX}{\mathfrak{X}}
\newcommand{\frakY}{\mathfrak{Y}}
\newcommand{\frakZ}{\mathfrak{Z}}
\newcommand{\frakU}{\mathfrak{U}}
\newcommand{\frakR}{\mathfrak{R}}
\newcommand{\frakP}{\mathfrak{P}}
\newcommand{\frakQ}{\mathfrak{Q}}
\newcommand{\sfX}{\mathsf{X}}
\newcommand{\sfY}{\mathsf{Y}}
\newcommand{\sfZ}{\mathsf{Z}}
\newcommand{\sfS}{\mathsf{S}}
\newcommand{\sfT}{\mathsf{T}}
\newcommand{\sfOmega}{\mathsf{\Omega}} %drinfeld p-adic upper-half plane
\newcommand{\rmA}{\mathrm{A}}
\newcommand{\rmB}{\mathrm{B}}
\newcommand{\calT}{\mathcal{T}}
\newcommand{\sfA}{\mathsf{A}}
\newcommand{\sfB}{\mathsf{B}}
\newcommand{\sfC}{\mathsf{C}}
\newcommand{\sfD}{\mathsf{D}}
\newcommand{\sfE}{\mathsf{E}}
\newcommand{\sfF}{\mathsf{F}}
\newcommand{\sfG}{\mathsf{G}}
\newcommand{\frakL}{\mathfrak{L}}
\newcommand{\K}{\mathrm{K}}
\newcommand{\rmT}{\mathrm{T}}
\newcommand{\bfv}{\mathbf{v}}
\newcommand{\bfg}{\mathbf{g}}
\newcommand{\frakV}{\mathfrak{V}}
\newcommand{\bfn}{\mathbf{n}}
\renewcommand{\o}{\mathfrak{o}}
\newcommand{\bbDelta}{\amsmathbb{}}

\newcommand{\aff}{\mathrm{aff}}
\newcommand{\ft}{\mathrm{ft}} %finite type
\newcommand{\fp}{\mathrm{fp}} %finite presentation
\newcommand{\fr}{\mathrm{fr}} %free
\newcommand{\tft}{\mathrm{tft}} %topologically finite type
\newcommand{\tfp}{\mathrm{tfp}} %topologically finite presentation
\newcommand{\tfr}{\mathrm{tfr}} %topologically free
\newcommand{\aft}{\mathrm{aft}}
\newcommand{\lft}{\mathrm{lft}}
\newcommand{\laft}{\mathrm{laft}}
\newcommand{\cpt}{\mathrm{cpt}}
\newcommand{\cproj}{\mathrm{cproj}}
\newcommand{\qc}{\mathrm{qc}}
\newcommand{\qs}{\mathrm{qs}}
\newcommand{\lcmpt}{\mathrm{lcmpt}}
\newcommand{\red}{\mathrm{red}}
\newcommand{\fin}{\mathrm{fin}}
\newcommand{\fd}{\mathrm{fd}} %finite-dimensional
\newcommand{\gen}{\mathrm{gen}}
\newcommand{\petit}{\mathrm{petit}}
\newcommand{\gros}{\mathrm{gros}}
\newcommand{\loc}{\mathrm{loc}}
\newcommand{\glob}{\mathrm{glob}}
%\newcommand{\ringed}{\mathrm{ringed}}
%\newcommand{\qcoh}{\mathrm{qcoh}}
\newcommand{\cl}{\mathrm{cl}}
\newcommand{\et}{\mathrm{\acute{e}t}}
\newcommand{\fet}{\mathrm{f\acute{e}t}}
\newcommand{\profet}{\mathrm{prof\acute{e}t}}
\newcommand{\proet}{\mathrm{pro\acute{e}t}}
\newcommand{\Zar}{\mathrm{Zar}}
\newcommand{\fppf}{\mathrm{fppf}}
\newcommand{\fpqc}{\mathrm{fpqc}}
\newcommand{\orig}{\mathrm{orig}} %overconvergent topology
\newcommand{\smooth}{\mathrm{sm}}
\newcommand{\sh}{\mathrm{sh}}
\newcommand{\op}{\mathrm{op}}
\newcommand{\cop}{\mathrm{cop}}
\newcommand{\open}{\mathrm{open}}
\newcommand{\closed}{\mathrm{closed}}
\newcommand{\geom}{\mathrm{geom}}
\newcommand{\alg}{\mathrm{alg}}
\newcommand{\sober}{\mathrm{sober}}
\newcommand{\dR}{\mathrm{dR}}
\newcommand{\rad}{\mathfrak{rad}}
\newcommand{\discrete}{\mathrm{discrete}}
%\newcommand{\add}{\mathrm{add}}
%\newcommand{\lin}{\mathrm{lin}}
\newcommand{\Krull}{\mathrm{Krull}}
\newcommand{\qis}{\mathrm{qis}} %quasi-isomorphism
\newcommand{\ho}{\mathrm{ho}} %homotopy equivalence
\newcommand{\sep}{\mathrm{sep}}
\newcommand{\unr}{\mathrm{unr}}
\newcommand{\tame}{\mathrm{tame}}
\newcommand{\wild}{\mathrm{wild}}
\newcommand{\nil}{\mathrm{nil}}
\newcommand{\defm}{\mathrm{defm}}
\newcommand{\Art}{\mathrm{Art}}
\newcommand{\Noeth}{\mathrm{Noeth}}
\newcommand{\affd}{\mathrm{affd}}
%\newcommand{\adic}{\mathrm{adic}}
\newcommand{\pre}{\mathrm{pre}}
\newcommand{\coperf}{\mathrm{coperf}}
\newcommand{\perf}{\mathrm{perf}}
\newcommand{\perfd}{\mathrm{perfd}}
\newcommand{\rat}{\mathrm{rat}}
\newcommand{\cont}{\mathrm{cont}}
\newcommand{\dg}{\mathrm{dg}}
\newcommand{\almost}{\mathrm{a}}
%\newcommand{\stab}{\mathrm{stab}}
\newcommand{\heart}{\heartsuit}
\newcommand{\proj}{\mathrm{proj}}
\newcommand{\qproj}{\mathrm{qproj}}
\newcommand{\pd}{\mathrm{pd}}
\newcommand{\crys}{\mathrm{crys}}
\newcommand{\prisma}{\mathrm{prisma}}
\newcommand{\FF}{\mathrm{FF}}
\newcommand{\sph}{\mathrm{sph}}
\newcommand{\lax}{\mathrm{lax}}
\newcommand{\weak}{\mathrm{weak}}
\newcommand{\strict}{\mathrm{strict}}
\newcommand{\mon}{\mathrm{mon}}
\newcommand{\sym}{\mathrm{sym}}
\newcommand{\lisse}{\mathrm{lisse}}
\newcommand{\an}{\mathrm{an}}
\newcommand{\ad}{\mathrm{ad}}
\newcommand{\sch}{\mathrm{sch}}
\newcommand{\rig}{\mathrm{rig}}
\newcommand{\pol}{\mathrm{pol}}
\newcommand{\plat}{\mathrm{flat}}
\newcommand{\proper}{\mathrm{proper}}
\newcommand{\compl}{\mathrm{compl}}
\newcommand{\non}{\mathrm{non}}
\newcommand{\access}{\mathrm{access}}
\newcommand{\comp}{\mathrm{comp}}
\newcommand{\tstructure}{\mathrm{t}} %t-structures
\newcommand{\pure}{\mathrm{pure}} %pure motives
\newcommand{\mixed}{\mathrm{mixed}} %mixed motives
\newcommand{\num}{\mathrm{num}} %numerical motives
\newcommand{\ess}{\mathrm{ess}}
\newcommand{\topological}{\mathrm{top}}
\newcommand{\convex}{\mathrm{cvx}}
\newcommand{\locconvex}{\mathrm{lcvx}}
\newcommand{\ab}{\mathrm{ab}} %abelian extensions
\newcommand{\inj}{\mathrm{inj}}
\newcommand{\surj}{\mathrm{surj}} %coverage on sets generated by surjections
\newcommand{\eff}{\mathrm{eff}} %effective Cartier divisors
\newcommand{\Weil}{\mathrm{Weil}} %weil divisors
\newcommand{\lex}{\mathrm{lex}}
\newcommand{\rex}{\mathrm{rex}}
\newcommand{\AR}{\mathrm{A\-R}}
\newcommand{\cons}{\mathrm{c}} %constructible sheaves
\newcommand{\tor}{\mathrm{tor}} %tor dimension
\newcommand{\connected}{\mathrm{connected}}
\newcommand{\cg}{\mathrm{cg}} %compactly generated
\newcommand{\nilp}{\mathrm{nilp}}
\newcommand{\isg}{\mathrm{isg}} %isogenous
\newcommand{\qisg}{\mathrm{qisg}} %quasi-isogenous
\newcommand{\irr}{\mathrm{irr}} %irreducible represenations
\newcommand{\simple}{\mathrm{simple}} %simple objects
\newcommand{\indecomp}{\mathrm{indecomp}}
\newcommand{\preproj}{\mathrm{preproj}}
\newcommand{\preinj}{\mathrm{preinj}}
\newcommand{\reg}{\mathrm{reg}}
\newcommand{\semisimple}{\mathrm{ss}}
\newcommand{\integrable}{\mathrm{int}}
\newcommand{\s}{\mathfrak{s}}

%prism custom command
\usepackage{relsize}
\usepackage[bbgreekl]{mathbbol}
\usepackage{amsfonts}
\DeclareSymbolFontAlphabet{\mathbb}{AMSb} %to ensure that the meaning of \mathbb does not change
\DeclareSymbolFontAlphabet{\mathbbl}{bbold}
\newcommand{\prism}{{\mathlarger{\mathbbl{\Delta}}}}

\begin{document}

    \title{The moduli stack of principal \texorpdfstring{$G$}{}-bundles}
    
    \author{Dat Minh Ha}
    \maketitle
    
    \begin{abstract}
        
    \end{abstract}
    
    {
      \hypersetup{} 
      %\dominitoc
      \tableofcontents %sort sections alphabetically
    }

    \listoftodos

    \section{Introduction}

    \section{Quotient stacks, classifying stacks, and principal bundles}
        \begin{convention}
            Fix a base scheme $S$ and consider a Grothendieck topology $\tau$ on $\Sch_{/S}$, and for technical reasons, we require that $\tau$ is \textit{no finer than the fpqc topology}. For brevity, let us denote the resulting site by $S_{\tau}$. 

            Write $\Sh(S_{\tau})$ for the sheaf topos on $S_{\tau}$, and $\Stk(S_{\tau}) := \Grpd(\Sh(S_{\tau}))$ for the $(2, 1)$-category of $\Grpd$-valued stacks on $S_{\tau}$. 

            Eventually, we will specialise to the case where $S := \Spec k$ where $k$ is an algebraically closed field of characteristic $0$.
        \end{convention}

        \begin{convention}
            Throughout, let $G$ be a group object of $S_{\tau}$ (guaranteed to exist because $S_{\tau}$ has finite products). Eventually, we will specialise to $G$ being affine, but not yet. We remark also that it does not really matter whether $G$ is a general group $S$-scheme or a group object of the underlying category of the small site $S_{\tau}$, since in practice, we will always be choosing the topology $\tau$ so that $G$ would be well-defined. For instance, algebraic groups are tautologically of finite type, so as long as the covering morphisms of $\tau$ are of finite type, e.g. $\tau = \et, \fppf$, we will be fine.

            \todo[inline]{I changed this convention slightly to account for some descent-theoretic subtleties. Does not really impact the larger content though.} 
        \end{convention}
        \begin{remark}
            If $G$ is smooth (e.g. when $S$ is the spectrum of a field of characteristic $0$) then on can take $\tau$ to be the \'etale topology. If not, then one will have to work with a topology at least as fine as the fppf topology.
        \end{remark}

        \subsection{Constructing \texorpdfstring{$\Bun_G$}{}}
            \begin{definition}[Equivariance] \label{def: equivariance}
                Let $Z, Z'$ be $S$-schemes with actions of $G$, say $\alpha: G \x_S Z \to Z$ and $\alpha': G \x_S Z' \to Z'$. A morphism $f: Z \to Z'$ is said to be \textbf{$G$-equivariant} if and only if the following diagram commutes:
                    $$
                        \begin{tikzcd}
                    	{G \x_S Z} & {G \x_S Z'} \\
                    	Z & {Z'}
                    	\arrow["{\id_G \x_S f}", from=1-1, to=1-2]
                    	\arrow["\alpha"', shift right=2, from=1-1, to=2-1]
                    	\arrow["{\pr_2}", shift left=2, from=1-1, to=2-1]
                    	\arrow["{\pr_2}", shift left=2, from=1-2, to=2-2]
                    	\arrow["{\alpha'}"', shift right=2, from=1-2, to=2-2]
                    	\arrow["f", from=2-1, to=2-2]
                        \end{tikzcd}
                    $$
            \end{definition}
            \begin{definition}[Principal bundles and quotient stacks] \label{def: principal_bundles_and_quotient_stacks}
                Let $Z$ be an $S$-scheme with an action by $G$, say $\alpha: G \x_S Z \to Z$ such that there exist a $\tau$-covering $\{f_i: U_i \to Z\}_{i \in I}$ such that one has pullback diagrams as follows for all $i \in I$:
                    $$
                        \begin{tikzcd}
                    	{G \x_S U_i} & {G \x_S Z} \\
                    	{U_i} & Z
                    	\arrow[from=1-1, to=1-2]
                    	\arrow["{\pr_1}"', shift right=2, from=1-1, to=2-1]
                    	\arrow["{\pr_2}", shift left=2, from=1-1, to=2-1]
                    	\arrow["\lrcorner"{anchor=center, pos=0.125}, draw=none, from=1-1, to=2-2]
                    	\arrow["{\pr_2}", shift left=2, from=1-2, to=2-2]
                    	\arrow["\alpha"', shift right=2, from=1-2, to=2-2]
                    	\arrow["{\exists f_i}", from=2-1, to=2-2]
                        \end{tikzcd}
                    $$
                In this case, the diagram:
                    $$
                        \begin{tikzcd}
                    	{G \x_S Z} & Z
                    	\arrow["\alpha", shift left=2, from=1-1, to=1-2]
                    	\arrow["{\pr_2}"', shift right=2, from=1-1, to=1-2]
                        \end{tikzcd}
                    $$
                shall be called a \textbf{principal $G$-bundle} over $Z$ and the pullback diagrams are called \textbf{local trivialisations}; often, should it be cumbersome to specify the $G$-action, one would often write principal $G$-bundles over $Z$ as $G$-equivariant morphisms $P \to Z$ and have it implicitly understood that the underlying scheme of $P$ is isomorphic to $G \x_S Z$. The same diagram can also be shown to be a groupoid object\footnote{In fact, an internal equivalence relation!} of $\Sh(S_{\tau})$, and by regarding it as such, one shall get the \textbf{quotient stack}:
                    $$[Z /_{\alpha} G]$$
                Its functor of points is given by:
                    $$[Z /_{\alpha} G](T) := \left\{ \text{principal $G$-bundles $P \to T$ and $G$-equivariant isomorphisms $T \to Z$} \right\}$$
                for all $S$-schemes $T$, i.e. \say{points} of $[Z/\_{\alpha}G]$ are principal $G$-bundles over the $G$-scheme $Z$.
            \end{definition}
            \begin{remark}
                For definition \ref{def: principal_bundles_and_quotient_stacks} to make sense, we need to remind ourselves of the fact that if $G \in \Ob(\Grp)$ is an abstract group (for a moment) acting via two actions on the same set $T$, say yielding two $G$-sets $P, P'$, then any $G$-set morphism $P \to P'$ will necessarily be bijective. Using this, we can show that two principal $G$-bundles with the same base scheme can only have invertible $G$-equivariant morphisms between them. 
            \end{remark}
            \begin{remark}
                The definition works should we replace $S_{\tau}$ by any $(2, 1)$-site with enough finite pullbacks. Note how we do not require the existence of any colimit for the definition of quotient stacks. 
            \end{remark}
            \begin{example}[Group extensions as principal bundles]
                In essence, we are thinking of principal bundles in the same way that we might think about group extensions. Recall that an extension $\hat{G}$ of a group $G$ (for the moment, not necessarily the same as our group $S$-scheme) by another group $H$ is a short exact sequence in $\Grp$ as follows:
                    $$1 \to H \to \hat{G} \xrightarrow[]{\pi} G \to 1$$
                \textit{A priori}, there is an action of $G$ on $H$, and the underlying set of $\hat{G}$ is nothing but $H \x G$, say by $\rho: G \x H \to H$. As such, we have formed the following principal bundle in $\Sets$:
                    $$
                        \begin{tikzcd}
                    	{G \x H} & H
                    	\arrow["\rho", shift left=2, from=1-1, to=1-2]
                    	\arrow["{\pr_2}"', shift right=2, from=1-1, to=1-2]
                        \end{tikzcd}
                    $$
                Note that this is doing nothing but realising $H$ as a certain $G$-set (in fact, principal $G$-bundles in $\Sets$ are nothing but $G$-sets!). We can also form the quotient stack/groupoid:
                    $$[H /_{\rho} G]$$
                whose functor of points is given by:
                    $$[H /_{\rho} G](T) := \left\{ \text{principal $G$-bundles $P \to T$ and $G$-set morphisms $T \to H$} \right\}$$
                for all sets $T$.

                The same argument can be applied to say that extensions of group $S$-schemes $G$ give rise to principal $G$-bundles. One caveat is that affine group schemes might have non-affine extensions (e.g. \todo{What's an example ?}) and conversely, non-affine group schemes may have affine extensions (e.g. according to Chevalley's Theorem, given any algebraic group $G$ over a perfect field $k$, there shall exist a \textit{unique} normal affine algebraic closed subgroup $H < G$ along with an abelian variety $A$ such that $1 \to H \to G \to A \to 1$ is a group $k$-scheme extension, and recall that abelian varieties are projective by definition and hence not affine).
            \end{example}
            
            One advantage of this definition over the usual one (cf. e.g. \cite[\href{https://stacks.math.columbia.edu/tag/044O}{Tag 044O}]{stacks}) is that it makes it clear that quotient stacks satisfy descent and that the existence of local trivialisations is hard-wired in the defining diagrams, which makes it easy to deal with such trivialisations using categorical procedures like taking pullbacks.
            
            \begin{example}[Classifying stacks and universal principal bundles] \label{example: classifying_stacks}
                Since $S$ is terminal in $S_{\et}$, $G$ acts on it trivially. The \textbf{classifying stack} of $G$, often denoted by $BG$, is nothing but the quotient stacks $[S/G]$. For any $T \in \Ob(S_{\tau})$, its $T$-points (i.e. objects of $BG(T)$) are principal $G$-bundles $P \to T$ along with $G$-equivariant morphisms $P \to T$. If $T$ is furthermore acted on trivially by $G$ then we will have that:
                    $$
                        \begin{aligned}
                            BG(T) & \cong \{ \text{principal $G$-bundles $P \to T$ and $G$-equivariant morphisms $T \to S$} \}
                            \\
                            & \cong \{ \text{principal $G$-bundles $P \to T$} \}
                        \end{aligned}
                    $$
                In other words, this is telling us that, for all $S$-schemes $T$ acted on trivially by $G$, all principal $G$-bundles $P \to T$ on $T$, there shall exist a morphism $\pt: S \to BG$ giving rise to a $(2, 1)$-pullback diagram in $\Stk(S_{\tau})$ as follows:
                    $$
                        \begin{tikzcd}
                    	P & S \\
                    	T & BG
                    	\arrow[from=1-1, to=1-2]
                    	\arrow["\forall"', from=1-1, to=2-1]
                    	\arrow["{(2, 1)}"{description}, "\lrcorner"{anchor=center, pos=0.125}, draw=none, from=1-1, to=2-2]
                    	\arrow["\exists \pt", from=1-2, to=2-2]
                    	\arrow["{\exists!}", dashed, from=2-1, to=2-2]
                        \end{tikzcd}
                    $$
                where $P \to S$ is the structural morphism. In this sense, one may regard any morphism of stacks:
                    $$\pt: S \to BG$$
                as a \textbf{universal princial $G$-bundle on $S$}, as any princial $G$-bundle $P \to T$ on an $S$-scheme $T$ is obtained by pullback $\pt: S \to BG$. Strictly speaking, this is an abuse of terminology, but it is nevertheless a useful way to think about principal $G$-bundles.
            \end{example}
            \begin{remark}
                 Definition \ref{def: principal_bundles_and_quotient_stacks} and example \ref{example: classifying_stacks} actually remain true when we replace $Z$ and $T$ by arbitrary prestacks fibred in groupoids over $T_{\tau}$. There are some necessary but obvious modifications to be made, and there are some $2$-categorical subtleties to worry about, but these are mostly abstract-nonsensical.
            \end{remark}
    
            The discussion in example \ref{example: classifying_stacks} leads to the following definition. Recall that for any prestacks $\scrX, \scrY$ fibred in groupoids over $\Sch_{/S}$, the mapping stack $\Maps_S(\scrX, \scrY)$ satisfies the following:
                $$\Maps_S(\scrX, \scrY)(T) := \Maps_S( T, \Maps_S(\scrX, \scrY) ) \cong \Maps_T(\scrX \x_S T, \scrY) \cong \Maps_T(\scrX \x_S T, \scrY \x_S T)$$
            for all prestacks $T$ in groupoids over $\Sch_{/S}$, where the first equivalence holds because $\Maps_S(-, -)$ is an internal Hom and the second equivalence holds thanks to the universal property of $(2, 1)$-pullbacks.
            \begin{definition}[Moduli stack of principal bundles] \label{def: moduli_stack_of_principal_bundles}
                The \textbf{moduli stack of principal $G$-bundles} $P \to X$ on a fixed $S$-scheme $X$ is given as the following mapping stack over $S$:
                    $$\Bun_G(X) := \Maps_S( X, BG )$$
            \end{definition}
            
            Before moving on, let us try to understand the definition. Firstly, consider the following for all $S$-scheme $X$, which holds because $\Maps_S(-, -)$ is an internal Hom:
                $$\Maps_S( X, \Maps_S( S, BG ) ) \cong \Maps_S( X \x_S S, BG ) \cong \Maps_S( X, BG ) = \Bun_G(X)$$
            Next, notice that $\Maps_S( X, \Maps_S( S, BG )  )$ is equivalent to the groupoid of $(2, 1)$-pullback squares:
                $$
                    \begin{tikzcd}
                        P & S \\
                        X & BG
                        \arrow[from=1-1, to=1-2]
                        \arrow["\forall"', from=1-1, to=2-1]
                        \arrow["{(2, 1)}"{description}, "\lrcorner"{anchor=center, pos=0.125}, draw=none, from=1-1, to=2-2]
                        \arrow["\exists \pt", from=1-2, to=2-2]
                        \arrow["{\exists!}", dashed, from=2-1, to=2-2]
                    \end{tikzcd}
                $$
            where $P \to X$ are principal $G$-bundles on the $S$-scheme $X$. At the same time, we know from example \ref{example: classifying_stacks} that any such bundle gives rise to such a pullback diagram. Therefore, it makes sense to regard $\Bun_G(X)$ as the moduli stack of principal $G$-bundles on $X$.
            
            \begin{example}
                When $X \cong S$, we have:
                    $$\Bun_G(S) \cong \Maps_S( S, BG )$$
                which is indeed the groupoid of principal $G$-bundles on $S$, per the content of example \ref{example: classifying_stacks}.
            \end{example}
            \begin{example}[$\Bun_{\GL_n}$ classifies rank-$n$ vector bundles] \label{example: Bun_GL_n_classifies_rank_n_vector_bundles}
                A very nice fact is that $\Bun_{\GL_n}(X)$ is isomorphic as an $S$-stack to the moduli stack of rank-$n$ vector bundles on $X$. This is useful for proving that, when $X$ is flat, of finite-presentation, and projective over $S$ and $G$ is smooth over $S$, the stack $\Bun_G(X)$ will in fact be algebraic (in fact, it will also be smooth!). The proof usually involves firstly proving that $\Bun_{\GL_n}(X)$ is algebraic, and then using some kind of change-of-group result.
    
                But back to the current discussion. Firstly, let us consider the $S$-stack $\Vect_n(S)$\footnote{We leave it as an exercise to the reader to fill in the details of the construction of this stack and to prove that as a category fibred in groupoids over $S_{\tau}$, it indeed satisfies descent as long as $\tau$ is no finer than the fpqc topology.} given as follows for all $S$-schemes $T$:
                    $$\Vect_n(S)(T) := \{ \text{locally free $\scrO_T$-modules of rank $n$ and $\scrO_T$-linear isomorphisms} \}$$
                It can be shown (see \cite[Lemma 4.1.1]{wang_algebraicity_of_Bun_G}) that there is an isomorphism of $S$-stacks:
                    $$\Vect_n(S) \to B\GL_n$$
                given for each $T \in \Ob(S_{\tau})$ (i.e. given as a natural transformation) by:
                    $$\scrE \mapsto \Isom_{\scrO_T}( \scrO_T^{\oplus n}, \scrE )$$
                where $\Isom_{\scrO_T}(-, -)$ is as in \cite[\href{https://stacks.math.columbia.edu/tag/08K7}{Tag 08K7}]{stacks} (see \cite[\href{https://stacks.math.columbia.edu/tag/0D3T}{Tag 0D3T}]{stacks} in particular for a description of $\Isom_{\scrO_T}(\scrF, \scrF')$ as a geometric object for any $\scrF, \scrF' \in \Ob(\QCoh(T))$); this is well-defined because any $\Isom_{\scrO_T}( \scrO_T^{\oplus n}, \scrE )$ has a natural $\GL_n \x_S T$-action, as isomorphisms of finite-dimensional vector spaces are given by invertible matrices. From this, we get that:
                    $$
                        \begin{aligned}
                            \Bun_{\GL_n}(X) & := \Maps_S(X, B\GL_n)
                            \\
                            & \cong \Maps_S(X, \Vect_n(S))
                            \\
                            & = \{ \text{locally free $\scrO_X$-modules of rank $n$ and $\scrO_X$-linear isomorphisms} \}
                        \end{aligned}
                    $$
                which means that $\Bun_{\GL_n}(X)$ is the moduli stack of rank-$n$ vector bundles on $X$.
            \end{example}
            
            Now, as a matter of fact, for any \textit{algebraic} $S$-stack $\scrY$, it makes sense to consider the moduli stack of rank-$n$ vector bundles on $\scrY$:
                $$\Vect_n(\scrY)(T) := \{ \text{locally free $\scrO_{\scrY}$-modules of rank $n$ and $\scrO_{\scrY}$-linear isomorphisms} \}$$
            with quasi-coherent modules over the structure sheaf of an algebraic stack being understood in the sense of \cite[\href{https://stacks.math.columbia.edu/tag/06WU}{Tag 06WU}]{stacks}; in particular, this means that it makes sense to speak of:
                $$\Bun_{\GL_n}(\scrY)$$
            as the moduli space of rank-$n$ vector bundles on $\scrY$. In light of example \ref{example: Bun_GL_n_classifies_rank_n_vector_bundles}, let us suppose that $\Bun_G(X)$ - and hence $\Bun_G(X) \x_S X$, since schemes are instances of algebraic stacks - is an algebraic stack (which can happen) and then consider the following:
                $$
                    \begin{aligned}
                        \Bun_{\GL_n}(\Bun_G(X) \x_S X) & = \Maps_S( \Bun_G(X) \x_S X, B\GL_n )
                        \\
                        & \cong \Maps_S(\Bun_G(X), \Maps_S(X, B\GL_n))
                        \\
                        & \cong \Maps_S(\Bun_G(X), \Bun_{\GL_n}(X))
                    \end{aligned}
                $$
            Vector bundles of rank $n$ on the algebraic stack $\scrY := \Bun_G(X) \x_S X$ are thus the same as morphisms $\Bun_G(X) \to \Bun_{\GL_n}(X)$, and when $G = \GL_n$, one has also a \textbf{universal vector bundle} (of rank $n$):
                $$\scrE_0$$
            on $\Bun_{\GL_n}(X) \x_S X$, corresponding to any representative of the isomorphism class of $\id_{\Bun_{\GL_n}(X)}$. This is universal in the sense that, because for any prestack $Y$ fibred in groupoids over $S$ an any $S$-scheme $T$, the groupoid $\Maps_S(Y, Y)(T)$ is monoidal with respect to compositions, for every:
                $$\scrE \in \Ob( \Bun_{\GL_n}(\scrY)(T) )$$
            there exists some:
                $$\varphi \in \Ob( \Maps_S(\Bun_{\GL_n}(X), \Bun_{\GL_n}(X)) )$$
            such that:
                $$\scrE \cong \varphi^* \scrE_0$$
            \todo[inline]{Needs more details}

        \subsection{Algebraicity}
            As eluded to above, the strategy is to prove firstly that, given enough hypotheses on $X$ (which turn out to be not at all restrictive in practice), the stack $\Bun_{\GL_n}(X)$ will be algebraic, and then secondly to use some kind of change-of-group argument to vary $G$. 

            \begin{convention}
                From now on, let $k$ be a field, let:
                    $$S := \Spec k$$
                and let us assume that:
                    $$G$$
                is an algebraic group over $\Spec k$, i.e. a finite-type group scheme over $\Spec k$. Also, when not specified, all geometric constructions such as schemes, stacks, etc. will be over $\Spec k$. Depending on the situation, we might also write:
                    $$\pt := \Spec k$$
                
                Recall also that, over fields of characteristic $0$, algebraic groups are automatically smooth (cf. \cite[\href{https://stacks.math.columbia.edu/tag/047N}{Tag 047N}]{stacks}).
            \end{convention}

            \begin{lemma}[Classifying stacks are algebraic] \label{lemma: classifying_stacks_are_algebraic}
                (Cf. \cite[Lemmas 2.5.1 and 2.5.2]{wang_algebraicity_of_Bun_G})
                \begin{enumerate}
                    \item Let $G, G'$ be algebraic groups acting trivially on the point scheme $\pt$. There is a canonical isomorphism:
                        $$BG \x_{\Spec k} BG' \xrightarrow[]{\cong} B(G \x_{\Spec k} G')$$
                    that when regarded as a natural transformation and evaluated at any $S \in \Ob((\Spec k)_{\tau})$, takes in a pair $(P, P')$ - conssiting of a principal $G$-bundle $P \to S$ and a principal $G'$-bundle $P' \to S$ - and sends it to the principal $G \x_{\Spec k} G'$-bundle $P \x_S P'$.
                    \item The diagonal morphism:
                        $$\Delta_{BG}: BG \to BG \x_{\Spec k} BG$$
                    is representable by schemes and furthermore, affine. 

                    In particular, we see thus that the classifying stack $BG$ is algebraic.
                \end{enumerate}
            \end{lemma}
                \begin{proof}
                    \begin{enumerate}
                        \item 
                        \item 
                    \end{enumerate}
                \end{proof}
            \begin{proposition}[Quotient stacks are algebraic]
                (Cf. \cite[Theorem 2.0.2]{wang_algebraicity_of_Bun_G}) Let $Z$ be any scheme with a $G$-action.
                \begin{enumerate}
                    \item The quotient stack $[Z/G]$ is algebraic.
                    \item The diagonal morphism:
                        $$\Delta_{[Z/G]}: [Z/G] \to [Z/G] \x_{\Spec k} [Z/G]$$
                    is representable by schemes and separated. 
                    \item If $Z$ is qs (respectively, separated), then $\Delta_{[Z/G]}$ will be qc (respectively, affine) moreover.
                \end{enumerate}
            \end{proposition}
                \begin{proof}
                    \begin{enumerate}
                        \item 
                        \item 
                        \item 
                    \end{enumerate}
                \end{proof}

    \section{Global geometry of \texorpdfstring{$\Bun_G$}{} of a curve}
        \subsection{Level structures and automorphic realisation}

        \subsection{Smoothness of \texorpdfstring{$\Bun_G$}{} of a curve}

    \section{Local geometry of \texorpdfstring{$\Bun_G$}{} of a curve}
    
    \addcontentsline{toc}{section}{References}
    \printbibliography

\end{document}