\input{article preambles}

\setcounter{section}{-1}

\input{commands}

\begin{document}

    \title{$\Bun_G$ is a smooth algebraic stack}
    
    \author{Dat Minh Ha}
    \maketitle
    
    \begin{abstract}
        
    \end{abstract}
    
    {
      \hypersetup{} 
      %\dominitoc
      \tableofcontents %sort sections alphabetically
    }

    \section{Introduction}

    \section{Quotient stacks}
        \begin{convention}
            Fix a base scheme $S$. If $\tau$ is a topology on the category $\Sch_{/S}$ then we shall write $S_{\tau}$ for the corresponding small site of $S$-schemes; e.g. $S_{\fppf}$ means the site whose objects are fppf morphisms $t: T \to S$ and whose morphisms are the commutative triangles:
                $$
                    \begin{tikzcd}
                	T && {T'} \\
                	& S
                	\arrow["t"', from=1-1, to=2-2]
                	\arrow["{t'}", from=1-3, to=2-2]
                	\arrow[from=1-1, to=1-3]
                    \end{tikzcd}
                $$
            and on this site, the topology is given by jointly surjective families of fppf morphisms $\{f_i: T_i \to S\}_{i \in I}$.
        \end{convention}
    
        \begin{definition}[Equivariant morphisms] \label{def: equivariant_morphisms}
            Suppose that $X, Y$ are $S$-schemes acted upon by the same group $S$-scheme $G$ via actions $\alpha_X: X \x_S G \to X, \alpha_Y: Y \x_S G \to Y$. A morphisms of $S$-schemes:
                $$f: X \to Y$$
            is said to be \textbf{$G$-equivariant} if and only if it induces the following morphism of internal groupoids in $\Sch_{/S}$:
                $$
                    \begin{tikzcd}
                	{X \x_S G} & X \\
                	{Y \x_S G} & Y
                	\arrow["f", from=1-2, to=2-2]
                	\arrow["{f \x \id_G}"', from=1-1, to=2-1]
                	\arrow["{\pr_1}"', shift right=2, from=1-1, to=1-2]
                	\arrow["{\pr_1}"', shift right=2, from=2-1, to=2-2]
                	\arrow["{\alpha_X}", shift left=2, from=1-1, to=1-2]
                	\arrow["{\alpha_Y}", shift left=2, from=2-1, to=2-2]
                    \end{tikzcd}
                $$
        \end{definition}
        \begin{remark}
            Equivalently, a morphism of $S$-schemes $f: X \to Y$, both acted upon by some group $S$-scheme $G$, is $G$-equivariant if and only if it induces a morphism of fppf sheaves (of sets) $\bar{f}: X/G \to Y/G$ fitting into the following commutative square in $\Sh(S_{\fppf})$:
                $$
                    \begin{tikzcd}
                	X & Y \\
                	{X/G} & {Y/G}
                	\arrow["{\coeq(\alpha_X, \pr_1)}"', from=1-1, to=2-1]
                	\arrow["{\bar{f}}", dashed, from=2-1, to=2-2]
                	\arrow["f", from=1-1, to=1-2]
                	\arrow["{\coeq(\alpha_Y, \pr_1)}", from=1-2, to=2-2]
                    \end{tikzcd}
                $$
            The equivalence with definition \ref{def: equivariant_morphisms} is guaranteed by the universal property of coequalisers. 
        \end{remark}
        \begin{definition}[Invariant morphisms] \label{def: invariant_morphisms}
            Suppose that $X, Y$ are $S$-schemes acted upon by the same group $S$-scheme $G$ via actions $\alpha_X: X \x_S G \to X, \alpha_Y: Y \x_S G \to Y$. A morphism of $S$-schemes:
                $$f: X \to Y$$
            is then \textbf{$G$-invariant} if and only if it is $G$-equivariant and there is a lifting $X/G \to Y$ fitting into the following commutative diagram in $\Sh(S_{\fppf})$:
                $$
                    \begin{tikzcd}
                	X & Y \\
                	{X/G} & {Y/G}
                	\arrow["{\coeq(\alpha_X, \pr_1)}"', from=1-1, to=2-1]
                	\arrow["{\bar{f}}", from=2-1, to=2-2]
                	\arrow["f", from=1-1, to=1-2]
                	\arrow["{\coeq(\alpha_Y, \pr_1)}", from=1-2, to=2-2]
                	\arrow[dashed, from=2-1, to=1-2]
                    \end{tikzcd}
                $$
        \end{definition}
        \begin{proposition}[Compositions of equivariant/invariant morphisms] \label{prop: compositions_of_equivariant_and_of_invariant_morphisms}
            Let $G$ be a group $S$-scheme. Then, compositions of $G$-equivariant (respectively, $G$-invariant) morphisms of $S$-schemes are once more $G$-equivariant (respectively, $G$-invariant).
        \end{proposition}
        \begin{proposition}[Base-changing equivariant/invariant morphisms] \label{prop: base_changes_of_equivariant_and_of_invariant_morphisms}
            Let $G$ be a group $S$-scheme and fix a $G$-equivariant (respectively, $G$-invariant) morphism of $S$-schemes:
                $$f: X \to Y$$
            Then, for all $T \in \Ob(S_{\fppf})$, the base-changed morphism:
                $$f \x \id_T: X_T \to Y_T$$
            will be $G_T$-equivariant (respectively, $G_T$-invariant). 
        \end{proposition}

        \begin{definition}[Quotient prestacks] \label{def: quotient_prestacks}
            Suppose that $G$ is a group $S$-scheme acting on some $S$-scheme $Z$ via $\alpha_Z: Z \x_S G \to Z$. One can then form the \textbf{quotient prestack} $[Z/_{\alpha}G]$ as the category fibred in groupoids over $\Sch_{/S}$ whose sections over objects $T \in \Ob(\Sch_{/S})$ are the groupoids\footnote{... internal to the category of sets.}:
                $$
                    \begin{tikzcd}
                	{Z(T) \x G(T)} & {Z(T)}
                	\arrow["{\pr_1}"', shift right=2, from=1-1, to=1-2]
                	\arrow["{\alpha_Z(T)}", shift left=2, from=1-1, to=1-2]
                    \end{tikzcd}
                $$
        \end{definition}
        \begin{example}[Classifying stacks of group schemes]
            Let $G$ be a group $S$-scheme acting trivially on the terminal object $S \in \Ob(\Sch_{/S})$. The quotient prestack $[S/G]$ is then the prestack over objects $T \in \Ob(\Sch_{/S})$ are the groupoids:
                $$
                    \begin{tikzcd}
                	{\{*\} \x G(T)} & {\{*\}}
                	\arrow["{\pr_1}"', shift right=2, from=1-1, to=1-2]
                	\arrow["{\alpha_S(T)}", shift left=2, from=1-1, to=1-2]
                    \end{tikzcd}
                $$
            (note that $S(T) \cong \{*\}$), which are nothing but the groupoids with a unqiue object and with elements of $G(T)$ (and compositions of such elements) as morphisms. 
        \end{example}
        \begin{remark}[Functoriality of quotient prestacks]
            Let $G$ be a group $S$-scheme.
        
            From proposition \ref{prop: compositions_of_equivariant_and_of_invariant_morphisms}, one sees that any composition:
                $$X \to Y \to Z$$
            of morphisms of $S$-schemes acted upon by $G$ induces a composition of $1$-morphisms of stacks:
                $$[X/G] \to [Y/G] \to [Z/G]$$
            and note that such compositions are $2$-associative. From proposition \ref{prop: base_changes_of_equivariant_and_of_invariant_morphisms}, one sees that given any $S$-scheme $X$ acted upon by $G$ and any arbitrary $S$-scheme $T$, there is a $2$-pullback:
                $$
                    \begin{tikzcd}
                	{[X_T/G_T]} & {[X/G]} \\
                	T & S
                	\arrow[from=1-2, to=2-2]
                	\arrow[from=2-1, to=2-2]
                	\arrow[from=1-1, to=2-1]
                	\arrow[from=1-1, to=1-2]
                	\arrow["2"{description}, "\lrcorner"{anchor=center, pos=0.125}, draw=none, from=1-1, to=2-2]
                    \end{tikzcd}
                $$
        \end{remark}
        \begin{remark}[Change of groups]
            Fix an $S$-scheme $Z$ acted upon by two group $S$-schemes $G, H$ via $\alpha_{Z, G}: Z \x_S G \to Z, \alpha_{Z, H}: Z \x_S H \to Z$. Suppose also that there is a group $S$-scheme homomorphism:
                $$\varphi: G \to H$$
            Such a homomorphism induces the following commutative diagram:
                $$
                    \begin{tikzcd}
                	{Z \x_S G} & Z \\
                	{Z \x_S H} & Z
                	\arrow["{\pr_1}"', shift right=2, from=1-1, to=1-2]
                	\arrow["{\alpha_{Z, G}}", shift left=2, from=1-1, to=1-2]
                	\arrow["{\pr_1}"', shift right=2, from=2-1, to=2-2]
                	\arrow["{\alpha_{Z, H}}", shift left=2, from=2-1, to=2-2]
                	\arrow["{\id_Z \x \varphi}"', from=1-1, to=2-1]
                	\arrow["{\id_Z}", from=1-2, to=2-2]
                    \end{tikzcd}
                $$
            which, when viewed as a morphism of groupoids internal to $\Sch_{/S}$, tells us that there is a resulting $1$-morphism of prestacks:
                $$[Z/G] \to [Z/H]$$
        \end{remark}

    \section{Mapping stacks and \texorpdfstring{$\Bun_G$}{}}
        \begin{definition}[Mapping prestacks] \label{def: mapping_prestacks}
            Suppose that $X$ is an $S$-scheme and $\scrY$ is an $S$-prestack. Let us then define the \textbf{mapping prestack} $\Maps_S(X, \scrY)$ to be the category fibred in groupoids over $\Sch_{/S}$, whose section over each $T \in \Ob(\Sch_{/S})$ is the groupoid of $1$-morphisms $X_T \to \scrY$, i.e.:
                $$\Maps_S(X, \scrY)(T) := \scrY(X_T)$$
        \end{definition}
        \begin{definition}[$\Bun_G$] \label{def: Bun_G}
            Suppose that $G$ is a group $S$-scheme and $X$ is an $S$-scheme acted upon by $G$. The \textbf{moduli prestack of $G$-bundles} $\Bun_G(X)$ is then defined to be a mapping prestack as follows:
                $$\Bun_G(X) := \Maps_S(X, [S/G])$$
        \end{definition}
    
    \addcontentsline{toc}{section}{References}
    \printbibliography

\end{document}